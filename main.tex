% arara: lualatex: { shell: yes }
% arara: biber
% arara: nomencl
% arara: lualatex: { shell: yes }
% arara: lualatex: { shell: yes }

\documentclass[ngerman]{ttlab-qualify}
%%% mögliche Optionen:
% - ngerman
% - english
% - minted
% - algorithm
% - nomencl
% - nolibertine
%%% wähle den Studiengang (Auswahl erforderlich):
% - bsc2019
% - msc2019
% - mscbio2019
% - mscwirtschaft2019
% - seminar
\usepackage{hyperref}
\usepackage{setspace}
\usepackage{booktabs}
\usepackage{float}

\addbibresource{quellen.bib}
\newcounter{needcitationcounter}
\newcommand{\needcitation}{%
  \stepcounter{needcitationcounter}%
  \textcolor{red}{[\theneedcitationcounter]}%
}

\begin{document}
\titlehead{
  Leon Rohe\\
  7990263\\
  Informatik (BA)\\
  s1225702@stud.uni-frankfurt.de  
}
\subject{Bachelorarbeit}
\author{Leon Rohe}
\title{RTReconstruct}
\subtitle{Konzeption und Evaluation eines skallierbaren Systems für Echtzeit-3D-Rekonstruktionen}
\date{Abgabedatum: <Datum>}
\publishers{Institut für Informatik\\Text Technology Lab\\Prof. Dr. Alexander Mehler\\Ggf. Name des Zweitbetreuers}

\maketitle

\begin{titlepage}
    \centering
    \vspace*{2cm}
    
    {\Large \textbf{Abstract}}\\[1.5cm]
    
    \begin{singlespace}
        \noindent
        Your abstract text goes here. It should be concise and clearly summarize the goal, method, and findings of your work.\\
    \end{singlespace}
    
    \vfill
\end{titlepage}

\tableofcontents

\chapter{Einleitung}

\section{Motivation}
Während sich Virtuelle Realität (VR), weg von einer ursprünglich reinen Simulationsumgebung, hin zu einem Medium entwickelt, das reale und virtuelle Welten verschmelzen lässt, findet sie auch zunehmend in anderen Bereichen Anwendung. \needcitation So profitieren beispielsweise die Bereiche Fernzusammenarbeit, Bildung, Architektur und Industrie, im VR-Kontext, erheblich davon, dass reale Räume erfasst und in eine virtuelle Umgebung übertragen werden können. \needcitation Dies bietet die Möglichkeit für immersivere und glaubwürdigere Interaktionsszenarien, in denen physische und virtuelle Elemente nahtlos koexistieren können.

Eine der zentralen technischen Herausforderungen in dieser Richtung ist dabei jedoch die Echtzeit-Rekonstruktion dieser realen Umgebungen. Standardmäßige Photogrammetrie bietet eine theoretisch sehr attraktive Grundlage, da sie handelsübliche Kameratechnologien nutzt, um detaillierte 3D-Rekonstruktionen zu erstellen. In der Praxis sind die meisten photogrammetrischen Verfahren jedoch ausschließlich für Offline-Anwendungen konzipiert, da sie eine hohe Rechenleistung und lange Verarbeitungszeiten erfordern, was die Integration in interaktive Systeme erschwert. \needcitation

Gleichzeitig stehen aber auch im Bereich der Echtzeit-Photogrammetrie zahllose Ansätze zur Verfügung, seien es nun klassische Multi-View-Methoden über SLAM-Verfahren bis hin zu neuronalen Rekonstruktionsmodellen. \needcitation Diese Vielfalt erschwert eine einheitliche Einbindung in bestehende VR-Anwendungen: Modelle sind oft inkompatibel, schwer austauschbar und erfordern angepasste Kommunikationslogik zwischen Backend und VR-Frontend.

Der modulare, containerisierte System-Ansatz, bei dem die einzelnen Rekonstruktionsmodule vom eigentlichen Ablauf entkoppelt und über standardisierte Schnittstellen mit der VR-Anwendung kommunizieren, bietet eine mögliche Lösung. Mit diesem Ansatz könnten neue Techniken einfacher integriert, getestet und verglichen werden - ein klarer Vorteil für einen sich rasch weiterentwickelnden Forschungsbereich.

Bislang ist jedoch unklar, wie performant und zuverlässig ein solch modulares System für Echtzeit-VR-Anwendungen tatsächlich ist.

% Trotz dieser Potenziale ist bislang unklar, wie zuverlässig und performant ein solches modulares System in einer Echtzeit-VR-Umgebung tatsächlich arbeitet. Genau diese Fragestellung steht im Mittelpunkt dieser Arbeit: Es soll untersucht werden, in welchem Maße eine modulare, photogrammetriebasierte Echtzeit-Rekonstruktionspipeline die technischen und performativen Anforderungen moderner Virtual-Reality-Anwendungen erfüllen kann.

\section{Zielsetzung}
% Das Ziel dieser Arbeit ist daher die Entwicklung und Untersuchung einer modularen, photogrammetrie basierten Echtzeit-Rekonstruktionspipeline, die einfach in VR-Anwendungen integriert werden kann. Im Mittelpunkt steht dabei die Frage, inwieweit ein containerisiertes, modular aufgebautes Backend die Anforderungen an Performance und Zuverlässigkeit erfüllen kann, die in einer interaktiven VR-Umgebung bestehen.

% Zu diesem Zweck wird zunächst eine Systemarchitektur entworfen, die eine klare Trennung zwischen Steuerlogik, Kommunikationsschicht und den einzelnen Rekonstruktionsmodulen vorsieht. Durch den Einsatz von Docker-Containern sollen die Module voneinander isoliert, aber über standardisierte Schnittstellen austauschbar und erweiterbar bleiben. Anschließend wird ein Prototyp implementiert, der eine vollständige Kommunikationskette zwischen einem Unity-basierten VR-Frontend und dem modularen Backend realisiert.

Aus diesem Grund besteht das Ziel dieser Arbeit darin, eine modulare, photogrammetrie-basierte Echtzeit-Rekonstruktionspipeline zu entwickeln und zu untersuchen, die sich leicht in VR-Anwendungen integrieren lässt.  Die zentrale Fragestellung dreht sich um: Wie gut kann ein containerisiertes, modular aufgebautes Backend die Performance- und Zuverlässigkeitsanforderungen erfüllen, die eine interaktive VR-Umgebung stellt?

Um dies zu erreichen, wird zunächst eine Systemarchitektur erstellt, die Steuerlogik, Kommunikationsschicht und die einzelnen Rekonstruktionsmodule klar voneinander trennt.  Die Module sollen durch Docker-Container voneinander isoliert, jedoch über standardisierte Schnittstellen austauschbar und erweiterbar sein. Daraufhin wird ein Prototyp umgesetzt, der eine komplette Kommunikationskette zwischen einem Unity-basierten VR-Frontend und dem modularen Backend ermöglicht.

% Im weiteren Verlauf wird das System experimentell evaluiert, um Aufschluss über seine Leistungsfähigkeit im Echtzeitbetrieb zu gewinnen. Dabei werden Metriken wie Latenz, Durchsatz, Ressourcenauslastung und Stabilität untersucht, um die Effizienz und Belastbarkeit der Architektur zu bewerten. Ergänzend wird analysiert, wie aufwändig die Integration neuer Rekonstruktionsmodule ist und in welchem Maße das System die angestrebte Flexibilität tatsächlich ermöglicht.

% Die Arbeit konzentriert sich somit auf die architektonische und systemtechnische Perspektive der Echtzeit-Rekonstruktion. Ziel ist nicht die algorithmische Weiterentwicklung photogrammetrischer Verfahren selbst, sondern die Schaffung einer robusten und erweiterbaren Infrastruktur, die deren Einbindung in Virtual-Reality-Anwendungen vereinfacht. Durch die Konzeption, Umsetzung und Evaluation dieser Pipeline soll ein Beitrag zur Beurteilung geleistet werden, wie praktikabel modulare photogrammetriebasierte Rekonstruktionssysteme für den Einsatz in Echtzeit-VR-Szenarien tatsächlich sind.

Im weiteren Verlauf wird eine experimentelle Evaluierung des Systems vorgenommen, um seine Leistungsfähigkeit im Echtzeitbetrieb zu untersuchen. Zur Bewertung der Stabilität und Effizienz der Architektur werden Indikatoren wie Latenz, Durchsatz, Ressourcenauslastung und Stabilität untersucht. Darüber hinaus wird untersucht, wie aufwendig es ist, neue Rekonstruktionsmodule zu integrieren, und inwieweit das System die angestrebte Flexibilität tatsächlich bietet.

Die Arbeit legt ihren Fokus auf die systemtechnische und architektonische Sichtweise der Echtzeit-Rekonstruktion. Das Ziel ist nicht die algorithmische Weiterentwicklung photogrammetrischer Verfahren, sondern die Entwicklung einer robusten und erweiterbaren Infrastruktur, die deren Integration in Virtual-Reality-Anwendungen erleichtert. Diese Pipeline wurde konzipiert, realisiert und evaluiert, um zur Einschätzung der Praktikabilität modularer photogrammetriebasierter Rekonstruktionssysteme in Echtzeit-VR-Szenarien beizutragen.

\section{Aufbau der Arbeit}
Die vorliegende Arbeit ist in sieben Kapitel gegliedert, die systematisch vom theoretischen Hintergrund bis zur praktischen Umsetzung und Evaluation führen. Kapitel 2 legt die grundlegenden Konzepte dar, die für das Verständnis der Arbeit notwendig sind. Dazu gehören Virtual Reality, photogrammetrische 3D-Rekonstruktion, Container-Technologien und Docker, Kommunikationsprotokolle sowie die Entwicklungsumgebung Unity.

Kapitel 3 beschäftigt sich mit dem Stand der Technik. Hier werden bestehende Rekonstruktionspipelines und modulare Backend-Architekturen untersucht und kritisch bewertet. Zudem erfolgt eine Abgrenzung der Arbeit gegenüber existierenden Ansätzen, um die wissenschaftliche Relevanz der eigenen Entwicklung zu verdeutlichen.

In Kapitel 4 wird die Konzeption des Systems erläutert. Es beginnt mit der Anforderungsanalyse und geht anschließend auf die Systemarchitektur, das Schnittstellendesign und den Kommunikationsfluss zwischen Backend und VR-Frontend ein. Ein besonderer Schwerpunkt liegt auf dem Modularisierungskonzept, das die Integration verschiedener Rekonstruktionsmodelle erleichtert.

Kapitel 5 beschreibt die Implementierung der entwickelten Lösung. Sowohl die Backend- als auch die Frontend-Komponenten werden detailliert dargestellt, und die Integration in die bestehende Infrastruktur des Va.Si.Li-Labs wird erläutert.

Kapitel 6 widmet sich der Evaluation des Systems. Es werden die eingesetzte Methodik, die erhobenen Messdaten sowie die Analyse und Diskussion der Ergebnisse präsentiert. Die Evaluation untersucht sowohl die Performance und Zuverlässigkeit als auch die Erweiterbarkeit der modularen Architektur.

Abschließend fasst Kapitel 7 die zentralen Erkenntnisse der Arbeit zusammen, diskutiert Limitationen des entwickelten Systems und gibt einen Ausblick auf mögliche zukünftige Entwicklungen und Optimierungen.

\chapter{Grundlagen}
Dieses Kapitel bildet die technische und theoretische Grundlage der Arbeit. Es erläutert die wichtigsten Konzepte, auf denen das entwickelte System basiert und vermittelt somit das notwendige Verständnis für die folgenden Kapitel.

\section{Virtuelle Realität und Extended Reality}
Als Virtuelle Realität (VR) wird eine durch Computer generierte, interaktive Darstellung einer virtuellen, dreidimensionalen Umgebung bezeichnet. In dieser können Nutzer immersiv mit ihrer Umgebung interagieren. \needcitation Durch sogenannte Head-Mounted Displays (HMDs) wird dabei ein räumliches Verständnis geschaffen, indem den Nutzern ein stereoskopisches Bild, Head-Tracking und 3D-Audio geboten werden.

Die Weiterentwicklung der einfachen virtuellen Realität (VR) ist die sogenannte Extended Reality (XR). Sie vereint virtuelle und reale Elemente zu einem zusammenhängenden Erlebnis. Dies wird bei unterstützten HMDs durch eine Passthrough-Funktion ermöglicht. Dabei wird das Bild einer am HMD befestigten Kamera an den Nutzer ausgegeben und um virtuelle Elemente ergänzt. Der Grad der Immersion reicht dabei von einfacher Augmented Reality (AR), bei der die reale Umgebung nur um virtuelle Informationen ergänzt wird, bis hin zur Mixed Reality (MR), bei der eine plausible Interaktion zwischen realer und virtueller Welt möglich wird. \needcitation

\section{3D-Rekonstruktion}
3D-Rekonstruktion beschreibt das Verfahren, bei dem aus einer Reihe von Bildern eine möglichst genaue virtuelle 3D-Repräsentation der aufgenommenen Umgebung erstellt wird. Erweitert man die Eingabe für diese Algorithmen um weitere Informationen, wie zum Beispiel Tiefenkarten, Kameraparameter (Intrinsiken und Extrinsiken), so kann diese Rekonstruktion noch akkurater werden. \needcitation Ziel der 3D-Rekonstruktion besteht darin Geometrie, Textur und räumliche Struktur möglichst genau wiederzugeben. \needcitation

In der Forschung existieren für diese Zwecke unterschiedliche Ansätze, die sich in Bezug auf die benötigten Daten, Genauigkeit der Rekonstruktion und die Echtzeitfähigkeit unterscheiden.

\subsection{Photogrammetrie}
Die Photogrammetrie nutzt mehrere überlappende Bilder, um durch Merkmalsextraktion, Korrespondenzsuche und Triangulation eine Punktwolke der Szene zu berechnen. Aus dieser wird anschließend ein 3D-Modell generiert. Photogrammetrische Verfahren liefern extem detaillierte Ergebnisse, erfordern aber jedoch hohe Rechenleistung und sind meist für nur Offline-Prozesse geeignet. \needcitation

\subsection{SLAM-Verfahren}
Simultaneous Localization and Mapping (SLAM) bezeichnet einen Ansatz, bei dem ein System seine eigene Position bestimmt und gleichzeitig eine Karte der Umgebung erstellt. Der Einsatz von SLAM-Verfahren erfolgt vor allem in der Robotik und in Augmented-Reality-Anwendungen, da sie kontinuierlich in Echtzeit arbeiten. Sie liefern weniger detailreiche, aber dafür schnelle Rekonstruktionen, die besonders für dynamische Umgebungen geeignet sind. \needcitation

\subsection{Neuronale Rekonstruktion}
Jüngere Entwicklungen im Bereich der 3D-Rekonstruktion basieren auf neuronalen Netzen, die Bild- und Tiefendaten direkt in volumetrische oder strahlbasierte Darstellungen übersetzen. Besonders relevant sind sogenannte Neural Radiance Fields (NeRF), die Lichtverteilungen in einem 3D-Raum modellieren. Modelle wie SLAM3R kombinieren diese Ansätze mit GPU-beschleunigter Echtzeitverarbeitung. \needcitation

\subsection{Qualitätskriterien der 3D-Rekonstruktion}
Die Bewertung der Qualität einer 3D-Rekonstruktion kann aus verschiedenen Perspektiven erfolgen. Für die im Rahmen dieser Arbeit verfolgte Zielsetzung – die Integration und Echtzeit-Ausführung verschiedener Rekonstruktionsverfahren in einer VR-Umgebung – sind insbesondere Genauigkeit, Vollständigkeit, Latenz und Robustheit von zentraler Bedeutung.

Genauigkeit beschreibt, wie präzise die rekonstruierte Geometrie mit der realen Szene übereinstimmt. Sie ist entscheidend, um in der virtuellen Umgebung realistische und maßstabsgetreue Darstellungen zu ermöglichen. Ungenauigkeiten wirken sich unmittelbar auf die Wahrnehmung der räumlichen Tiefe und das Interaktionsverhalten im VR-System aus.

Vollständigkeit bezeichnet den Anteil der tatsächlichen Szene, der erfolgreich rekonstruiert wurde. Da die rekonstruierten Modelle in Echtzeit in die VR-Umgebung eingebettet werden, ist eine möglichst vollständige Erfassung der Umgebung wichtig, um Lücken und Artefakte zu vermeiden, die die Immersion des Nutzers stören könnten.

Latenz beschreibt die Zeitspanne zwischen der Erfassung der Eingangsdaten und dem Auftreten des aktualisierten 3D-Modells in der VR-Szene. Für interaktive Anwendungen ist eine geringe Latenz essenziell, um ein konsistentes Nutzererlebnis zu gewährleisten und Verzögerungen zwischen realer Bewegung und virtueller Rückmeldung zu minimieren.

Robustheit schließlich misst die Widerstandsfähigkeit des Systems gegenüber Störeinflüssen wie wechselnden Lichtverhältnissen, Bewegungsunschärfe oder dynamischen Objekten. Da das System in realen Einsatzszenarien nicht unter idealisierten Bedingungen arbeitet, ist Robustheit ein entscheidendes Kriterium für die praktische Nutzbarkeit der vorgeschlagenen Architektur.

\section{Systemische Grundlagen}

\subsection{Container-Technologie und Docker}
Container-Technologien ermöglichen es, Anwendungen mitsamt ihren Abhängigkeiten in isolierten Umgebungen auszuführen. Dadurch lassen sich Softwarekomponenten unabhängig vom Hostsystem betreiben, was die Reproduzierbarkeit und Portabilität von Anwendungen wesentlich verbessert. \needcitation Ein Container enthält alle zur Laufzeit benötigten Komponenten einer Anwendung, wie etwa Bibliotheken, Konfigurationsdateien und Laufzeitumgebungen. Dadurch wird sichergestellt, dass die Software in unterschiedlichen Umgebungen identisch funktioniert. Im Gegensatz zu virtuellen Maschinen teilen sich Container den Kernel des Host-Betriebssystems, wodurch sie deutlich ressourcenschonender sind. \needcitation Die bekannteste Implementierungen einer solchen Container-Technologie ist Docker. \needcitation

\subsection{WebSockets}
WebSockets sind ein Netzwerkprotokoll, das eine bidirektionale, dauerhafte Verbindung zwischen Client und Server ermöglicht. Im Gegensatz zu klassischen HTTP-Verbindungen, die nach jeder Anfrage beendet werden, erlaubt eine WebSocket-Verbindung den kontinuierlichen Austausch von Nachrichten in beide Richtungen. Dies macht sie besonders geeignet für Anwendungen mit Echtzeitanforderungen. \needcitation

Nach dem initialen Verbindungsaufbau über ein HTTP-Handshake wird die Verbindung auf das WebSocket-Protokoll umgestellt. Von diesem Zeitpunkt an können sowohl Client als auch Server asynchron Daten senden, ohne dass eine neue Anfrage notwendig ist. \needcitation

\section{Zielumgebung}
\subsection{Unity als VR-Plattform}
Die Game-Engine Unity dient als zentrale Laufzeit- und Entwicklungsumgebung für Virtual- und Extended-Reality-Anwendungen. Sie ermöglicht die Erstellung interaktiver, immersiver 3D-Szenen, die mit verschiedenen Head-Mounted Displays (HMDs) kompatibel sind. \needcitation

Unity bietet eine komponentenbasierte Architektur, bei der jedes Objekt durch sogenannte \textit{GameObjects} und deren \textit{Components} definiert wird. Dadurch lassen sich komplexe Systeme modular und flexibel strukturieren. Für Virtual-Reality-Anwendungen stehen in Unity spezialisierte XR-Plugins und SDKs zur Verfügung, die Head-Tracking, stereoskopisches Rendering, Controller-Interaktion und 3D-Audio unterstützen. \needcitation

Besonders relevant ist die enge Integration mit VR-Plattformen wie Meta Quest, HTC Vive oder Valve Index, wodurch plattformübergreifende Entwicklung möglich wird. Zudem erlaubt Unity durch seine Script-API in C\# die direkte Einbindung externer Systeme, beispielsweise über Netzwerkprotokolle wie WebSockets. \needcitation

\subsection{Das Va.Si.Li-Lab}
Das Va.Si.Li-Lab („Virtual and Simulation-Based Learning Laboratory“)  ist eine immersive Mehrbenutzer-VR-Umgebung der Goethe-Universität Frankfurt, die für simulationsbasiertes Lernen und interaktive Forschungsanwendungen entwickelt wurde. Es bietet eine hochgradig flexible Infrastruktur, in der virtuelle Lernszenarien, soziale Interaktionen und kollaborative Experimente in Echtzeit durchgeführt und analysiert werden können. \needcitation

Das Labor kombiniert Virtual-Reality-Technologien mit multimodaler Datenerfassung. Zu diesem Zweck werden diverse Eingabekanäle, wie etwa Sprache, Gestik, Blickrichtung sowie Objektinteraktionen, erfasst und zentral gespeichert. Die Architektur basiert auf Unity und dem Mehrbenutzer-Framework Ubiq, das es mehreren Teilnehmern gleichzeitig erlaubt, zu interagieren. \needcitation

Das Labor deckt ein breites Spektrum an Anwendungsszenarien ab: von schulischen und beruflichen Trainings über soziale Interaktionssimulationen bis hin zu wissenschaftlichen Untersuchungen von Lernprozessen. \needcitation

% \subsubsection*{Relevanz für diese Arbeit}
% Die Wahl des Va.Si.Li-Lab als Zielumgebung bietet mehrere Vorteile für die Integration und Evaluation des entwickelten Systems:
% \begin{itemize}
%     \item \textbf{Bestehende Infrastruktur:} Das Lab stellt bereits eine stabile Multiuser-VR-Plattform bereit, die eine schnelle Integration des Systems ermöglicht.
%     \item \textbf{Echtzeit-Interaktion:} Durch die bestehende Netz- und Trackinginfrastruktur können Live-Rekonstruktionsdaten direkt im VR-Kontext evaluiert werden.
%     \item \textbf{Erweiterbarkeit:} Die modulare Architektur des Labs erlaubt die Anbindung externer Container und Kommunikationsschnittstellen wie WebSockets.
%     \item \textbf{Forschungsorientierung:} Die Verbindung von technischer Innovation und simulationsbasiertem Lernen eröffnet neue Perspektiven für interaktive, datengestützte Forschungsanwendungen.
% \end{itemize}
\chapter{Stand der Technik}

Während Kapitel~2 die theoretischen und technischen Grundlagen der 3D-Rekonstruktion erläuterte,
rückt dieses Kapitel den aktuellen Forschungsstand VR-tauglicher Rekonstruktionssysteme in den Fokus.
\emph{VR-tauglich} bezeichnet dabei Verfahren, die Kameradaten während der Bewegung fortlaufend verarbeiten
und die Szene in einer laufenden Sitzung inkrementell aktualisieren; gemeint sind Rekonstruktionsupdates,
nicht die Display-Latenz des HMD.
Ziel ist es, etablierte End-to-End-Pipelines sowie ausgewählte Einzelverfahren systematisch einzuordnen
und ihre Relevanz für die in dieser Arbeit entwickelte modulare Architektur zu begründen.

\section{End-to-End-Rekonstruktionspipelines}

Die Entwicklung echtzeitfähiger 3D-Rekonstruktionssysteme wurde maßgeblich durch zwei Paradigmen geprägt:
RGB-D-basierte Ansätze, die auf Tiefensensoren aufbauen, und rein monokulare Verfahren, die ausschließlich
RGB-Bildströme verarbeiten.

\subsection{RGB-D-basierte Pipelines}

Frühe echtzeitfähige Systeme wie KinectFusion~\cite{kinectfusion} etablierten die volumetrische Fusion
von Tiefendaten mittels TSDF-Repräsentation. Diese Architektur ermöglichte erstmals die inkrementelle
Rekonstruktion während der Kamerabewegung und wurde in zahlreichen Arbeiten erweitert, etwa durch
dynamisches Hashing (VoxelHashing~\cite{voxelhashing}), elastische Oberflächenverfolgung
(DynamicFusion~\cite{dynamicfusion}) oder Multi-View-Konsistenz (BundleFusion~\cite{bundlefusion}).

Allen gemein ist die Annahme einer verfügbaren RGB-D-Kamera, die synchronisierte Farb- und Tiefendaten liefert.
Diese Hardware-Abhängigkeit schränkt die Anwendbarkeit auf VR-Headsets mit integrierten Tiefensensoren ein --
eine Voraussetzung, die bei gängigen Consumer-VR-Systemen wie der Meta Quest-Serie typischerweise nicht erfüllt ist.
Für Multi-User-VR-Szenarien bedeutet dies, dass entweder alle Teilnehmenden über identische, tiefenfähige Hardware
verfügen müssen oder RGB-D-Verfahren als Lösungsklasse ausscheiden.
Zudem sind viele dieser Systeme primär für Einzelnutzende konzipiert; eine kontinuierliche Verteilung
inkrementeller Rekonstruktionsupdates an mehrere synchronisierte Clients wird nicht nativ adressiert.

\subsection{Monokulare Pipelines}

Mit DTAM~\cite{dtam}, LSD-SLAM~\cite{lsdslam} und ORB-SLAM~\cite{orbslam} entstanden monokulare Systeme,
die Tiefe aus Kamerabewegung und photometrischer bzw.\ Feature-basierter Konsistenz ableiten.
Diese Verfahren sind hardwareseitig mit Standard-RGB-Kameras kompatibel und damit prinzipiell
für Consumer-VR-Headsets geeignet.

Allerdings wurden klassische Systeme häufig für Offline-Verarbeitung oder spezifische Anwendungsfälle
(z.\,B.\ robotische Navigation) entwickelt.
Eine Integration in interaktive VR-Umgebungen erfordert typischerweise Anpassungen an Daten-Pipelines,
da Eingabeformate, Tracking-Annahmen und Ausgaberepräsentationen eng an die jeweilige Implementierung gekoppelt sind.
Neuere lernbasierte Ansätze wie DeepVideoMVS~\cite{deepvideomvs} oder NICE-SLAM~\cite{niceslam} kombinieren
neuronale Tiefenschätzung mit Mapping-Komponenten, bleiben aber häufig an spezifische Frameworks und Datenformate gebunden.
Ein systematischer Vergleich mehrerer Verfahren unter identischen Bedingungen ist dadurch mit erheblichem
Integrationsaufwand verbunden.

\subsection{Limitationen für VR-Integration}

Trotz erheblicher Fortschritte auf Algorithmusseite teilen viele End-to-End-Pipelines zentrale Limitationen
für den Einsatz in Multi-User-VR-Umgebungen:

\begin{enumerate}
\item \textbf{Monolithische Architektur:}
Rekonstruktionslogik, Tracking und Ausgabeformate sind häufig fest in die jeweilige Pipeline integriert.
Ein Austausch einzelner Komponenten oder der parallele Betrieb mehrerer Modelle erfordert pro Verfahren
individuelle Code-Anpassungen, separate Laufzeitumgebungen und dedizierte Test-Setups.

\item \textbf{Fehlende Streaming-Infrastruktur:}
Viele Systeme sind für Batch-Verarbeitung oder Single-Client-Szenarien ausgelegt.
Eine kontinuierliche Verteilung inkrementeller Updates an mehrere synchronisierte VR-Clients während einer
laufenden Sitzung wird nicht nativ unterstützt; Rekonstruktionsergebnisse müssen oft exportiert und anschließend
in die VR-Engine importiert werden.

\item \textbf{Unzureichende Pose-Flexibilität:}
Verfahren mit externer Pose-Anforderung (z.\,B.\ TSDF-basierte volumetrische Rekonstruktion) und Verfahren mit
interner Pose-Schätzung (z.\,B.\ SLAM-basierte Ansätze) benötigen unterschiedliche Datenflüsse.
Viele bestehende Pipelines sind nicht darauf ausgelegt, beide Kategorien parallel zu unterstützen oder zwischen ihnen
zu wechseln, ohne Frontend- oder Pipeline-Logik anzupassen.
\end{enumerate}

Diese Einschränkungen motivieren die Entwicklung einer modularen, containerisierten Architektur, die unterschiedliche
Rekonstruktionsverfahren -- unabhängig von ihrer internen Implementierung -- über eine standardisierte Schnittstelle
in VR-Umgebungen integriert und dabei sowohl monolithische Kopplung als auch Hardware-Abhängigkeiten reduziert.

\section{Ausgewählte Rekonstruktionsverfahren}

Zur Evaluation der Architektur unter realistischen Heterogenitäten wurden vier Modelle integriert:
\textit{NeuralRecon}, \textit{VisFusion}, \textit{MASt3R-SLAM} und \textit{SLAM3R}.
Die Auswahl erfolgte nach folgenden Kriterien:
\begin{itemize}
    \item \textbf{Paradigmenvielfalt:} Abdeckung unterschiedlicher Rekonstruktionsansätze (volumetrisch, SLAM-basiert, feed-forward)
    \item \textbf{Pose-Heterogenität:} Verfahren mit externer Pose-Anforderung und solche mit interner Schätzung
    \item \textbf{Repräsentationsformate:} Sowohl geschlossene Oberflächenmodelle (Mesh) als auch Punktwolken
    \item \textbf{Aktualität und Verfügbarkeit:} Publikationen aus den Jahren 2021--2024 mit verfügbaren Open-Source-Implementierungen
    \item \textbf{GPU-Kompatibilität:} Echtzeitfähigkeit auf Consumer-Hardware (NVIDIA RTX 3090/4090)
\end{itemize}

Die vier Modelle wurden so kombiniert, dass sie die in dieser Arbeit relevanten Schnittstellen-Varianten abdecken:
volumetrische Mesh/TSDF-Ausgabe mit externen Posen (NeuralRecon, VisFusion) sowie punktbasierte, inkrementelle Rekonstruktion mit interner/impliziter Pose-Schätzung (MASt3R-SLAM, SLAM3R).
Damit kann die Architektur unter heterogenen Ausgabeformaten und Tracking-Annahmen evaluiert werden, ohne die Kommunikations- und Visualisierungsschnittstellen zu verändern.

\begin{table}[H]
\centering
\caption{Vergleich der integrierten Rekonstruktionsverfahren hinsichtlich Schnittstellenanforderungen und Ausgabedaten.}
\label{tab:model_comparison}
\begin{tabularx}{\linewidth}{@{}l l l l X@{}}
\toprule
\textbf{Verfahren} & \textbf{Paradigma} & \textbf{Pose-Annahme} & \textbf{Repräsentation} & \textbf{Output-Eignung in VR} \\
\midrule
NeuralRecon        & Volumetrisch & Extern & TSDF/Mesh  & Geschlossene Oberflächen, Interaktion/Collider möglich \\
VisFusion          & Volumetrisch & Extern & TSDF/Mesh  & Detailreiche Oberflächen, gut triangulierbar \\
MASt3R-SLAM        & Hybrid       & Intern (SLAM) & Punktwolke   & Inkrementell, detailreich, Meshing optional \\
SLAM3R             & Feed-forward & Intern/implizit & Punktwolke & Inkrementell, ohne explizite Pose-Optimierung \\
\bottomrule
\end{tabularx}
\end{table}

Bewusst nicht integriert wurden klassische Sparse-SLAM-Verfahren (z.\,B.\ ORB-SLAM ohne Dense-Mapping), da diese keine für VR-Visualisierung geeigneten dichten Repräsentationen liefern, sowie NeRF-basierte Verfahren, die aufgrund langer Trainingszeiten für Streaming-Szenarien ungeeignet sind.
Ebenfalls nicht integriert wurde 3D Gaussian Splatting (3DGS) trotz hoher Aktualität, da der Ansatz primär auf \emph{Real-Time Radiance Field Rendering} bzw.\ Novel-View-Synthesis ausgelegt ist und typischerweise eine optimierungsbasierte Szenenrepräsentation aus (vor-)kalibrierten Mehransichten erstellt, anstatt eine engine-nahe, explizite Geometrie-Repräsentation (z.\,B.\ Mesh/TSDF) für Interaktion, Kollision und Navigation bereitzustellen.

\subsection{Volumetrische Verfahren}

\textbf{NeuralRecon} rekonstruiert lokale TSDF-Fragmente anstatt pro Keyframe isolierte Tiefenkarten~\cite{sun2021neuralrecon}.
Das Verfahren nimmt RGB-Bilder mit extern bereitgestellten Posen entgegen, projiziert extrahierte Bildmerkmale
in ein sparsifiziertes Voxelvolumen und integriert diese über rekurrente Fusionsmechanismen (GRU).
Daraus entstehen glatte, gut triangulierbare Oberflächen, die als geschlossene Meshes ausgegeben werden und sich
für Kollisionsabfragen oder Navigation eignen.
Die Abhängigkeit von externen Posen erfordert robustes Tracking durch das VR-System, während die festgelegte
Voxelauflösung den erreichbaren Detailgrad limitiert.

\textbf{VisFusion} erweitert volumetrische Online-Rekonstruktion um explizite Modellierung von Sichtbarkeit während
der Feature-Fusion~\cite{gao2023visfusion}.
Wie NeuralRecon benötigt es extern bereitgestellte Posen und liefert TSDF-basierte Oberflächenrepräsentationen.
Im Vergleich zu Verfahren ohne Sichtbarkeitsprüfung nutzt VisFusion vorhergesagte Sichtbarkeitsgewichte sowie
\textit{ray-based sparsification}, um Details zu erhalten und konkurrierende Beobachtungen konsistenter zu integrieren.
Dies kann zu konsistenterer Fragmentfusion und besserer Detailerhaltung beitragen, verlagert jedoch zusätzliche Kosten
in die Laufzeit.

\subsection{Hybride SLAM-basierte Verfahren}

\textbf{MASt3R-SLAM} ist ein echtzeitfähiges monokulares Dense-SLAM-System, das klassische SLAM-Komponenten mit
einem lernbasierten 3D-Rekonstruktions- und Matching-Prior kombiniert~\cite{murai2025mast3rslam}.
Anders als volumetrische Verfahren schätzt es Kameraposen intern (u.\,a.\ über Feature-Matching und Bundle-Adjustment),
wodurch keine externe Tracking-Quelle benötigt wird.
Als Ergebnis entsteht eine dichte, farbige Punktwolkenrepräsentation, die inkrementell aktualisiert werden kann und
keine explizite Meshing-Stufe voraussetzt.
Der erhöhte GPU-Speicherbedarf durch transformerbasierte Komponenten kann bei begrenzten Ressourcen zu Engpässen führen.

\subsection{Feed-forward-Verfahren}

\textbf{SLAM3R} verfolgt einen feed-forward-basierten Ansatz zur dichten Rekonstruktion aus monokularen RGB-Videos~\cite{slam3r2024}.
Das System unterteilt die Verarbeitung in ein \textit{Image-to-Points}-Modul zur lokalen Rekonstruktion und ein
\textit{Local-to-World}-Modul zur inkrementellen globalen Registrierung~\cite{slam3r2024arxiv}.
Kameraposen werden dabei nicht explizit optimiert, sondern implizit durch gelernte Registrierung zwischen lokalen Punktkarten
abgeleitet.
Dies ermöglicht schnelle Verarbeitung ohne klassische Bundle-Adjustment-Schleifen, kann jedoch bei längeren oder komplexen
Kamerabewegungen zu akkumuliertem Drift führen, da ohne explizite globale Pose-Optimierung nachträgliche Korrekturen nur
begrenzt möglich sind.

\section{Forschungslücke und Abgrenzung}
\label{sec:gap}

Die dargestellten Limitationen bestehender End-to-End-Pipelines (Abschnitt~3.1) sowie die daraus resultierenden,
heterogenen Schnittstellen- und Ausgabeanforderungen der ausgewählten Modelle (Abschnitt~3.2, Tabelle~\ref{tab:model_comparison})
verdeutlichen eine Lücke an der Schnittstelle zwischen Forschungsprototypen und VR-Deployment.
Trotz erheblicher Fortschritte auf Algorithmusseite fehlt es an wiederverwendbarer Infrastruktur, die diese Verfahren in
laufende Multi-User-VR-Sitzungen integriert.

Konkret fehlen Infrastrukturen, die folgende Anforderungen \textit{gleichzeitig} erfüllen:

\begin{enumerate}
\item \textbf{Modellunabhängige Integration:}
Unterschiedliche Rekonstruktionsverfahren (volumetrisch, SLAM-basiert, feed-forward) sollen ohne Code-Anpassung in eine
gemeinsame Architektur eingebettet werden können.
Dies erfordert eine standardisierte Schnittstelle, die unabhängig von internen Frameworks und Tracking-Annahmen funktioniert.

\item \textbf{Kontinuierliches Streaming für Multi-User-VR:}
Ergebnisse müssen inkrementell in laufende VR-Szenen mit mehreren synchronisierten Clients gestreamt werden, während die
Rekonstruktion fortschreitet.
Batch-orientierte Pipelines (Aufnahme \(\rightarrow\) Offline-Verarbeitung \(\rightarrow\) Import) unterstützen dies nicht,
und Forschungsprototypen bieten selten standardisierte Streaming-Protokolle oder Multi-Client-Synchronisation.

\item \textbf{Heterogene Ausgabeformate:}
Die Architektur muss unterschiedliche 3D-Repräsentationen (TSDF/Mesh vs.\ Punktwolke) einheitlich visualisieren und dabei
formatspezifische Charakteristika (z.\,B.\ Chunking, LOD, inkrementelle Updates) berücksichtigen.

\item \textbf{Pose-Flexibilität:}
Systeme, die externe Posen benötigen, und solche mit interner/impliziter Schätzung sollen parallel betrieben werden können,
ohne Frontend-Logik oder Kommunikationsprotokoll anzupassen.
\end{enumerate}

Aktuelle Publikationen zu den evaluierten Modellen adressieren primär Rekonstruktionsqualität und algorithmische Verbesserungen,
nicht jedoch Deployment-Architekturen für VR-Integration oder den systematischen Vergleich heterogener Verfahren unter identischen Bedingungen.
Der praktische Integrationsaufwand (u.\,a.\ Anpassung von Datenformaten, Implementierung von Streaming-Logik, Synchronisation mehrerer Clients)
bleibt dadurch weitgehend den Anwendenden überlassen.

Das in dieser Arbeit entwickelte System \textit{RTReconstruct} adressiert diese Lücke durch eine containerisierte, modellunabhängige Architektur
mit standardisiertem Streaming-Protokoll und bildet die Grundlage für die in Kapitel~4 beschriebene Konzeption, Implementierung und Evaluation.

\chapter{Konzeption}

\section{Anforderungsanalyse}

\section{Systemarchitektur}

\section{Schnittstellendesign}

\section{Modularisierungskonzept}
\chapter{Implementierung}
Dieses Kapitel beschreibt die konkrete Umsetzung der in Kapitel 4 entworfenen Architektur. Es gliedert sich in einen Überblick zu Technologien und Laufzeitumgebung, die Backend- und Frontend-Realisierung, echtzeitkritische Pipeline-Aspekte, die Integration ins Va.Si.Li-Lab sowie reflektierte Designentscheidungen.

\section{Gesamtarchitektur \& Technologien}
\textbf{Ziel der Implementierung}

\noindent
Das System realisiert eine modulare Streaming-Pipeline, in der ein Unity-Client Bild- und Pose-Daten fensterbasiert an einen WebSocket-Router sendet. Der Router verteilt Fragmente an modell­spezifische Worker (z. B. SLAM3R, MASt3R, VisFusion), sammelt deren Ergebnisse (GLB/Punktwolke/Mesh) und publiziert sie an alle Clients derselben Szene. Die Ergebnisse werden zusätzlich persistiert. Die Router-Seite ist als FastAPI-Anwendung umgesetzt; modellseitig fungieren asynchrone WebSocket-Worker als Producer/Consumer. 

\medskip
\noindent
\textbf{Technologie-Stack}

\noindent
Backend: Python, FastAPI/Uvicorn, websockets, SQLite; Containerisierung via Docker Compose; GPU-fähige Worker-Container für Modelle (Neucon, VisFusion, SLAM3R, MASt3R). Frontend: Unity (C\#), NativeWebSocket, glTFast für GLB-Import, VFX-basierte Punktwolken-Darstellung. Die Compose-Orchestrierung startet den Router (Port 5000) und mehrere GPU-Worker; Quellverzeichnisse werden in die Container gemountet.

\section{Kommunikationsstandard}
Um eine effiziente WebSocket-Kommunikation zwischen dem Front- und Backend zu gewährleisten, werden einige standardisierte Byte-Frames definiert. Die wichtigsten Nachrichtentypen sind:
\begin{enumerate}
    \item ModelFragment (Client $\rightarrow$ Router $\rightarrow$ Modell)
    \par
    \noindent
    Header: `LEON', Version 1, Fenstergröße; dann Modellname und Szenenname. Es folgen Fensterweise:
    \begin{itemize}
        \item Bilddaten (JPEG-Bytes + Breite/Höhe)
        \item Intrinsics (Fx, Fy, Cx, Cy)
        \item Extrinsics (Kamera-Position xyz, Rotation xyzw)
    \end{itemize}
    Der Router erkennt Version 1, extrahiert Modell-/Szenennamen und legt das Fragment in die Queue des adressierten Modells.

    \item ModelResult (Modell $\rightarrow$ Router $\rightarrow$ Client)
    \par
    \noindent
    Header: LEON, Version 2, Szenenname; dann isPointcloud, Translation (xyz), Rotation (xyzw), Scale (xyz), gefolgt vom GLB-Payload. Der Router speichert pro Szene/Modell das jüngste Resultat und verteilt es an alle abonnierten Clients (bei bereits vorhandenen Ergebnissen direkt nach dem Handshake).

    \item (optional) TransformFragment (Client $\rightarrow$ Router)
    \par
    \noindent
    Header: LEON, Version 2, Szenenname; dann Translation (xyz), Rotation (Quaternion xyzw), Skalierung (xyz). Der Router aktualisiert damit die Pose aller bereits vorliegenden Ergebnisse der Szene und benachrichtigt verbundene Clients.
\end{enumerate}

\section{Backend}

\subsection{Architektur und Komponenten}

\textbf{Router (FastAPI) }

\noindent
Der Router stellt zwei WebSocket-Endpunkte bereit:
\begin{enumerate}
    \item \verb|/ws/client| für Unity-Clients (Handshake, Ergebnis-Fan-out)
    \item \verb|/ws/model/{model_name}| für Modell-Worker (Fragment-Zustellung, Ergebnis-Rückkanal)
\end{enumerate}
Der Router verwaltet außerdem den globalen Zustand von verbundenen Clients, welcher Szene diese zugeordnet sind, einer Eingabe-Queue für jedes Modell und eine Modelausgabe pro Modell pro Szene.

Beim Verbindungsaufbau mit dem Client sendet der Router einen Handshake als JSON mit einer Liste an aktuell verfügbaren Modellen.

\medskip
\noindent
\textbf{Modell-Worker}

\noindent
Ein Worker verbindet sich per WebSocket, lauscht auf eingehende Fragment, verarbeitet sie und sendet ein \texttt{ModelResult} zurück. Die abstrakte Basisklasse kapselt Verbindungs-/Sende-Logik und definiert die Funktionen \texttt{handle\_fragment(fragment)} sowie \texttt{send\_result(result)}.

\medskip
\noindent
\textbf{Persistenz}

\noindent
Ergebnisse werden in SQLite (db/results.db) gespeichert, inklusive Szene, Transform, Pointcloud-Flag und GLB-Blob; parallel protokolliert der Router je Modell CSV-Zeilen (input size, inference time, output size) zur späteren Auswertung.

\subsection{Datenfluss}
\begin{itemize}
    \item \textbf{Client-Endpoint} \verb|/ws/client|: Handshake (Rolle/Szene) → Modellliste senden → Empfang von LEON v1/v2 → v1 an Modell-Queue, v2 als Transform anwenden → Event-Signale an Szene-Abonnenten. Bereits vorhandene Ergebnisse der Szene werden unmittelbar gesendet.
    \item \textbf{Modell-Endpoint} \verb|/ws/model/{model_name}|:
Worker verbindet sich, erhält Fragmente aus der Queue, sendet nach Inferenz ModelResult v2 zurück; Router misst Latenz, schreibt CSV-Zeile, persistiert in SQLite und setzt Client-Events zur Auslieferung. Sondercodes b'0'/b'1' signalisieren Fehler bzw. „unwichtiges“ Ergebnis.
\end{itemize}

\subsection{Containerisierung}
Die docker-compose.yml baut Router und Worker, setzt GPU-Reservierungen und übergibt Modell-Namen und Server-URL (ws://router:5000/ws/model) als Umgebungsvariablen; Worker-Entry-Points verbinden sich selbsttätig.

\section{Frontend}

\subsection{Architektur und Komponenten}
Die Frontend-Architektur basiert auf vier zentralen Komponenten, die über definierte Schnittstellen miteinander interagieren.

\medskip

ICaptureDevice abstrahiert den hardwarespezifischen Sensorzugriff und ermöglicht plattformunabhängige Datenerfassung. Die Implementierung umfasst MetaQuestCaptureDevice für Passthrough-Kameras, SmartphoneCaptureDevice für ARFoundation-basierte mobile Geräte sowie Eval-Varianten zum Abspielen vorgezeichneter Pose-Sequenzen für reproduzierbare Tests. Alle Implementierungen liefern RGB-Bilder, Intrinsics (Brennweite, Hauptpunkt) und Extrinsics (Position, Rotation) in einheitlichem Format.

\medskip

IModelCollector verwaltet eingehende Frames in einem Sliding-Window-Puffer und entscheidet anhand von Schwellenwerten für Translation, Rotation und zeitlichen Abstand, ob ein Frame als Keyframe aufgenommen wird. Modellspezifische Varianten wie NeuralReconCollector und SLAM3RCollector definieren unterschiedliche Sampling-Strategien.

\medskip

ReconstructionClient implementiert die WebSocket-basierte Kommunikation mit dem Backend über NativeWebSocket. Nach dem Handshake sendet der Client Rolle und Szenenname und erhält die Liste verfügbarer Modelle zurück. Eine asynchrone Sendewarteschlange verwaltet die Nachrichtenserialisierung im Background-Thread, wobei Transform-Updates bevorzugt behandelt werden.

\medskip

RoomReconstructor verarbeitet eingehende Rekonstruktionsergebnisse thread-sicher und visualisiert sie in der Unity-Szene. Mesh-basierte Ergebnisse werden mit glTFast importiert und mittels MeshUtils.ChunkMesh() in kleinere Chunks unterteilt, Punktwolken werden über den Visual Effects Graph mit GPU-Buffern gerendert.

\subsection{Capture Pipeline}
Die Datenerfassung erfolgt frame-synchron als Coroutine, die mit Unity's Rendering-Zyklus koordiniert. Der ReconstructionManager führt CaptureLoop() aus, die nach WaitForEndOfFrame() Pose und Bilddaten zeitlich konsistent erfasst. Die Intrinsics werden basierend auf dem Verhältnis zwischen nativer und verwendeter Auflösung (640×480) skaliert, das Kamerabild wird in eine RenderTexture gerendert, per GPU-Readback kopiert und JPEG-kodiert. Die Keyframe-Selektion prüft mittels ShouldCollect(), ob räumliche oder zeitliche Schwellenwerte überschritten wurden.

\subsection{Clientseitige Kommunikation}
Das Frontend kommuniziert über eine dauerhafte WebSocket-Verbindung mit dem Router. Nach dem initialen Handshake werden Fragmente im binären LEON-v1-Format übertragen.
Merkmale der Übertragungsschicht:

\begin{itemize}
    \item Asynchrone Sendequeue zur Entkopplung der VR-Hauptschleife
    \item Background-Thread-Serialisierung für geringe Latenz
    \item Priorisierte Behandlung von Transform-Updates
\end{itemize}

Bereits vorhandene Rekonstruktionsergebnisse werden direkt nach Verbindungsaufbau zugestellt.

\subsection{Verarbeitung und Rendering rekonstruktiver Ergebnisse}
Rekonstruktionsergebnisse werden asynchron empfangen und thread-sicher in die Unity-Szene integriert. Das LEON Version 2-Protokoll enthält neben Header und Szenenname ein Transform (Translation, Rotation, Scale) sowie die GLB-Payload.

Alle Unity-API-Operationen werden über eine mainThreadActions-Queue gesammelt und im Update()-Zyklus abgearbeitet. Mesh-basierte Rekonstruktionen werden mit glTFast geladen und mittels MeshUtils.ChunkMesh() in ein $n\times n \times n$ Grid unterteilt, um Frustum Culling zu verbessern.

Punktwolken nutzen strukturierte GPU-Buffer für Positionen und Farben, die an den Visual Effect Graph übergeben werden. Nach dem Setzen der Buffer triggert Reinit() die GPU-seitige Aktualisierung.

\subsection{Geräteintegration}
Die Geräteintegration erfolgt über das ICaptureDevice-Interface, das hardwarespezifische Details abstrahiert. Das Frontend ist dennoch primär für Live-VR-Erfassung auf Meta Quest ausgelegt:

\begin{itemize}
    \item Passthrough-RGB-Kamera der Meta Quest
    \item Original-Headset-Pose aus XR-Tracking
    \item Unity-Render-Pipeline für Echtzeit-Anzeige
\end{itemize}

Weitere Sensorquellen (Smartphone, virtuelle Testkamera) werden über das gleiche Capture-Interface angebunden.

\section{Integration in das Va.Si.Li-Lab}
Die Integration erfolgte in die bestehende Unity-basierte Va.Si.Li-Lab-Umgebung, die eine mehrbenutzerfähige VR-Plattform für simulationsbasiertes Lernen und Forschung bereitstellt. Ziel war eine nahtlose Einbindung der Rekonstruktionspipeline, ohne bestehende Mehrbenutzer- und Interaktionsmechaniken zu beeinträchtigen.

\subsection{Zielumgebung und Anforderungen}
Das Va.Si.Li-Lab basiert auf Unity mit Ubiq als Mehrbenutzer-Framework und erfasst multimodale Interaktionen in Echtzeit. Für die Einbindung war erforderlich, die Rekonstruktionsdaten als eigenständige, netzwerkagnostische Szeneinhalte zu führen und die Frontend-Clientlogik strikt von Lab-spezifischen Synchronisationsdiensten zu entkoppeln. Dadurch bleibt die Rekonstruktion unabhängig von Session-Logik und kann in beliebige Lernszenarien eingebettet werden.

\subsection{Architekturankopplung}
Die Integration nutzt den bestehenden Szenen-Lebenszyklus des Labs und bindet den ReconstructionManager sowie RoomReconstructor als additive Komponenten in die Lab-Hauptszene ein. Die Kommunikation mit dem Backend erfolgt ausschließlich über die WebSocket-Schnittstelle des ReconstructionClient, der zu keiner Zeit Lab-interne Netzwerkpfade (z. B. Ubiq-Channels) verwendet, um Kollisionen mit dem Mehrbenutzer-Stack zu vermeiden. Resultate werden als reguläre Unity-GameObjects instanziiert und über Parent-Transforms an die Lab-Koordinaten verankert, wodurch Persistenz- und Analyse-Logik des Labs unberührt bleibt.

\subsection{Mehrbenutzer-Verhalten}
Die Rekonstruktionspipeline ist netzwerkagnostisch und backendgetrieben: Der Router publiziert Ergebnisse szenenweit, sodass mehrere Clients im selben Lab-Raum identische Resultate erhalten, ohne dass die Lab-Mehrbenutzer-Logik erweitert werden muss. Für den Fall, dass Clients zu unterschiedlichen Zeiten beitreten, liefert der Router das jüngste Ergebnis unmittelbar nach dem Handshake nach, wodurch ein konsistenter Zustand über die Sitzung hinweg gewährleistet bleibt.

\subsection{Deployment und Konfiguration}
Die Backend-Seite wird mittels Docker Compose im  bereitgestellt; Router und Worker-Container starten mit vordefinierten Modellnamen und GPU-Ressourcenzuteilung. Der Unity-Client wird mit Lab-spezifischen Endpunkten und Szenennamen konfiguriert; Umschalten zwischen Meta Quest Live-Capture und Evaluationsmodus erfolgt über das ICaptureDevice ohne Änderungen an der Lab-Szene. Bestehende Logging- und Persistenzmechanismen des Routers sichern Ergebnisse pro Szene und Modell, was Lab-seitig die nachträgliche Analyse ermöglicht, ohne zusätzliche Integrationsarbeit im Lab-Projekt.
\chapter{Evaluation}
\label{chap:evaluation} % Hinzugefügt, falls andere Kapitel darauf verweisen

Dieses Kapitel evaluiert das entwickelte RTReconstruct-System hinsichtlich seiner 
Echtzeitfähigkeit, Modularität und Rekonstruktionsqualität. Zunächst werden die 
Evaluationsziele definiert und die Testumgebung beschrieben. Anschließend erfolgt 
die Darstellung der gemessenen Performance-Metriken sowie der Rekonstruktionsqualität 
der integrierten Modelle. Das Kapitel schließt mit einer Diskussion der Ergebnisse 
im Kontext der definierten Anforderungen.

\section{Evaluationsziele}
\label{sec:eval_ziele}

Die Evaluation verfolgt drei zentrale Ziele, die direkt aus der in Abschnitt formulierten Forschungsfrage abgeleitet sind:

\begin{enumerate}
    \item \textbf{Funktionale Validierung}: Nachweis, dass die modulare Architektur 
    die definierten funktionalen Anforderungen erfüllt. Insbesondere wird geprüft, 
    ob verschiedene Rekonstruktionsmodelle parallel betrieben werden können und ob 
    die End-to-End Kommunikation zwischen VR-Frontend und Backend stabil funktioniert.
    
    \item \textbf{Echtzeitfähigkeit}: Bewertung, ob das System die für VR-Anwendungen 
    erforderlichen Performance-Anforderungen erfüllt. Dabei werden Latenz, Durchsatz 
    und Ressourcenauslastung als kritische Metriken untersucht.
    
    \item \textbf{Rekonstruktionsqualität}: Qualitative und -- soweit möglich -- 
    quantitative Bewertung der von den integrierten Modellen erzeugten 
    3D-Rekonstruktionen unter identischen Bedingungen.
\end{enumerate}

\section{Evaluationsmethodik}
Die Evaluationsmethodik beschreibt die Testumgebung, die verwendeten Testszenarien und die Messmethoden, die zur Erreichung der Evaluationsziele eingesetzt wurden. 

\subsection{Test- und Evaluationsumgebung}
\label{sec:testumgebung}

Um die \textbf{Reproduzierbarkeit} der Performance-Messungen und die \textbf{Vergleichbarkeit} der erzielten Ergebnisse zu gewährleisten, wurde die gesamte Evaluation in einer dedizierten und \textbf{kontrollierten Hard- und Software-Umgebung} durchgeführt. Die zentralen Komponenten und Spezifikationen dieser Umgebung sind in Tabelle~\ref{tab:hardware_and_software} zusammengefasst.

\begin{table}[H]
    \centering
    \label{tab:hardware_and_software}
    \caption{Spezifikationen der Hard- und Software-Umgebung}
    \begin{tabularx}{\textwidth}{l X}
        \toprule
        \textbf{Kategorie} & \textbf{Details und Spezifikationen} \\
        \midrule
        \multicolumn{2}{l}{\textbf{Hardware-Umgebung (Backend/Server)}} \\
        \midrule
        Backend-Server CPU & AMD Ryzen 9 5900X (12 Kerne, 24 Threads) \\
        GPU & NVIDIA GeForce GTX 1070 Ti, 8 GB VRAM \\
        VR-System (Frontend) & Meta Quest 3 \\
        Netzwerk & WiFi 6 Heimnetzwerk \\
        \midrule
        \multicolumn{2}{l}{\textbf{Software-Umgebung und Frameworks}} \\
        \midrule
        Betriebssystem & Ubuntu 22.04 LTS (Host) \\
        Containerisierung & Docker (\textit{[Version einfügen, z.B. 24.0.7]}) \\
        GPU-Unterstützung & NVIDIA Container Toolkit \\
        VR-Frontend & Unity (\textit{2022.3 LTS}) \\
        \bottomrule
    \end{tabularx}
\end{table}

\noindent
Alle Messungen erfolgten unter \textbf{kontrollierten Bedingungen}. Es wurde strikt darauf geachtet, dass während der Performance-Tests \textbf{keine weiteren rechenintensiven Hintergrundprozesse} liefen, um Verzerrungen zu minimieren.

\subsection{Testszenarien und Datensätze}

Für die Evaluation wurden fünf Testszenen mit unterschiedlichen Komplexitätsstufen 
konzipiert: drei virtuelle Szenen mit verfügbarem Ground-Truth zur quantitativen 
Bewertung sowie zwei reale Szenen aus einem typischen Alltagsumfeld zur Validierung der
Praxistauglichkeit. Eine kurze Übersicht über alle Testszenen findet sich in Tabelle~\ref{tab:test_scenes_overview}.

\begin{table}[H]
    \centering
    \label{tab:test_scenes_overview}
    \begin{tabularx}{\textwidth}{lcclX}
        \toprule
        \textbf{Szene} & \textbf{Größe (m)} & \textbf{Fragments} & \textbf{Frames} & \textbf{Merkmale} \\
        \midrule
        \multicolumn{5}{l}{\textbf{Virtuelle Szenen}} \\
        V1 -- Primitive & $5\times 5\times 5$ & 24 & 216 & Geometrische Grundformen, unifarbige Oberflächen \\
        V2 -- Schlafzimmer & $6\times 5\times 3$ & 44 & 396 & moderate Komplexität, Okklusionen \\
        V3 -- Mehrzweckraum & $10\times 5\times 3$ & 58 & 522 & Hohe Dichte, komplexe Geometrie \\
        \midrule
        \multicolumn{5}{l}{\textbf{Reale Szenen}} \\
        R1 & $4\times 4\times 3$ & [150] & [40] & \textit{Schlafzimmer}, Details, Schrägen, Okklusionen \\
        R2 & $6\times 5\times 3$ & [200] & [50] & \textit{Wohnzimmer}, Glas, Reflexionen, große Flächen \\
        \bottomrule
    \end{tabularx}
    \caption{Übersicht und Klassifikation der Testszenen}
\end{table}

\subsubsection{Virtuelle Szenen}

Die drei virtuellen Szenen wurden in Unity erstellt und ermöglichen durch verfügbare \textit{Ground-Truth-Meshes} eine quantitative Evaluation mittels F-Score. Die Szenen folgen einer progressiven Komplexitätssteigerung, um verschiedene Aspekte der Rekonstruktionsverfahren isoliert zu testen. Abbildung~\ref{fig:virtual_scenes} zeigt eine Übersicht aller drei Szenen.

\paragraph{Szene V1 -- Geometrische Primitive}
Szene V1 dient als Baseline-Test und enthält ausschließlich einfache geometrische Primitive (Quader, Pyramide, Zylinder, Kapsel) in einem hexagonalen Raum mit farbigen Wänden. Die unifarbigen, matten Oberflächen ohne Texturen ermöglichen die isolierte Bewertung fundamentaler Rekonstruktionsfähigkeiten: scharfe Kanten, gekrümmte Oberflächen und feature-arme Flächen.

\paragraph{Szene V2 -- Möbliertes Schlafzimmer}
Szene V2 repräsentiert einen möblierten Innenraum mittlerer Komplexität mit Doppelbett, Sessel, Sideboard, Wandbildern und Stehlampe. Diese Szene testet die Rekonstruktion komplexer Möbelgeometrie, das Verhalten bei Okklusionen, die Texturverarbeitung sowie die Detailerfassung kleiner Dekorationsobjekte.

\paragraph{Szene V3 -- Komplexer Mehrzweckraum}
Szene V3 stellt einen Stresstest für Skalierbarkeit und Detailtreue dar und simuliert einen multifunktionalen Raum mit Schlaf-, Wohn- und Arbeitsbereich. Die hohe Objektdichte mit zwei Betten, Esstisch, Stühlen und diversen Kleinobjekten erzeugt multiple Okklusionsebenen. Erwartet werden längere Inferenzzeiten, höhere GPU-Auslastung und potenzielle Artefakte bei geometrisch komplexen Strukturen und teilweise verdeckten Bereichen.

\begin{figure}[H]
    \centering
    \includegraphics[width=0.32\textwidth]{images/room00.png}
    \includegraphics[width=0.32\textwidth]{images/room01.png}
    \includegraphics[width=0.32\textwidth]{images/Room02.png}
    \caption{Übersicht der drei virtuellen Testszenen: V1 (Geometrische Primitive), V2 (Möbliertes Schlafzimmer), V3 (Komplexer Mehrzweckraum)}
    \label{fig:virtual_scenes}
\end{figure}

\subsubsection{Reale Szenen}

Die beiden realen Szenen wurden im Va.Si.Li-Lab aufgenommen und validieren die Praxistauglichkeit des Systems unter realen Bedingungen mit natürlichen Störfaktoren.

\paragraph{Szene R1 -- Schlafzimmer}

Szene R1 simuliert ein kleines, dicht möbliertes, privates Umfeld. Details, Schrägen, Okklusionen. Der Testfokus liegt auf der Robustheit gegenüber Textiloberflächen und diffuser Beleuchtung, welche die Rekonstruktion feiner Details und das Verhalten bei Oberflächenhomogenität überprüfen.

\paragraph{Szene R2 -- Wohnzimmer}

Szene R2 simuliert ein großes Wohnzimmer mit offener Gestaltung. Große Flächen, Glas, Reflexionen. Die Szene dient als Skalierbarkeits- und Materialstresstest. Im Fokus stehen die Handhabung großer, glänzender Flächen und Fenster, die Reflexionen verursachen, sowie repetitive Dekorelemente, welche die globale Konsistenz und Anfälligkeit für visuellen Drift testen.

\begin{figure}[H]
    \centering
    \includegraphics[width=0.49\textwidth]{images/room01.png}
    \includegraphics[width=0.49\textwidth]{images/room01.png}
    \caption{Übersicht der beiden realen Testszenen: R1 (Schlafzimmer), R2 (Wohnzimmer)}
    \label{fig:real_scenes}
\end{figure}

\subsection{Messverfahren und Metriken}

\subsubsection{Performance-Metriken}

\paragraph{Latenz}
Die End-to-End-Latenz misst die Zeitspanne zwischen dem Versenden eines Fragments 
durch das Unity-Frontend und der Visualisierung der aktualisierten Rekonstruktion 
im VR-Headset. Sie setzt sich aus folgenden Komponenten zusammen:
\begin{align}
    L_{\text{total}} = L_{\text{network}} + L_{\text{inference}} + L_{\text{render}}
\end{align}
wobei $L_{\text{network}}$ die Netzwerklatenz (Upload des Fragments und Download 
der Rekonstruktion), $L_{\text{inference}}$ die Modell-Inferenzzeit im Backend 
(GPU-Verarbeitung) und $L_{\text{render}}$ die Rendering-Zeit im Unity-Client 
(GLB-Import und Mesh-Visualisierung) bezeichnet.

\medskip
\noindent
Die Netzwerklatenz $L_{\text{network}} = L_{\text{upload}} + L_{\text{download}}$ 
wird nicht direkt durch Zeitstempel gemessen, sondern aus den erfassten Datenvolumina 
und der verfügbaren Netzwerkbandbreite berechnet: $L_{\text{upload}} = S_{\text{fragment}} / B_{\text{upload}}$ 
bzw. $L_{\text{download}} = S_{\text{result}} / B_{\text{download}}$. Hierbei 
bezeichnet $S_{\text{fragment}}$ die Fragmentgröße (Upload-Volumen pro Fragment) 
und $S_{\text{result}}$ die Resultgröße (Download-Volumen der Rekonstruktion). 
Diese Methodik ermöglicht eine infrastrukturunabhängige Bewertung der Dateneffizienz.

\medskip
\noindent
Die Messung der übrigen Latenzkomponenten erfolgte durch präzise Zeitstempel an 
den jeweiligen Übergangspunkten der Pipeline.

\paragraph{Durchsatz}
Der Durchsatz quantifiziert, wie viele Fragmente pro Sekunde durch das System 
verarbeitet werden können. Ein höherer Durchsatz ermöglicht häufigere Updates der 
Rekonstruktion und trägt zur Immersion bei. Gemessen wurde der Durchsatz auf 
Backend-Seite für jedes Worker-Modell separat.

\paragraph{Ressourcenauslastung}
Die GPU- und CPU-Auslastung wurde kontinuierlich während der Rekonstruktion 
aufgezeichnet. GPU-Utilization und GPU-Memory wurden via \texttt{nvidia-smi} 
erfasst, CPU-Auslastung und RAM-Verbrauch pro Container via Docker Stats. Diese 
Metriken ermöglichen die Bewertung der Ressourceneffizienz und geben Aufschluss 
über Engpässe im System.

\subsubsection{Qualitätsmetriken}

\paragraph{Quantitative Bewertung}
Für Szenen mit verfügbarem Ground-Truth-Mesh wurde der F-Score als kombinierte 
Metrik für Präzision und Recall berechnet:

\begin{align}
    \text{Precision} &= \frac{|\text{TP}|}{|\text{TP}| + |\text{FP}|} \\
    \text{Recall} &= \frac{|\text{TP}|}{|\text{TP}| + |\text{FN}|} \\
    \text{F-Score} &= 2 \cdot \frac{\text{Precision} \cdot \text{Recall}}{\text{Precision} + \text{Recall}}
\end{align}

\noindent
Rekonstruierte Punkte gelten als \textbf{True Positive (TP)}, wenn ihr Abstand zum 
Ground-Truth unter \textit{10} cm liegt, andernfalls als \textbf{False Positive (FP)}. 
\textbf{False Negatives (FN)} sind Ground-Truth-Punkte ohne entsprechenden 
rekonstruierten Punkt innerhalb des Schwellenwerts.

\medskip
\noindent
\textbf{Precision} misst die Genauigkeit der Rekonstruktion, indem sie den Anteil 
korrekt rekonstruierter Punkte angibt. \textbf{Recall} bewertet die 
Vollständigkeit und gibt an, wie viele Ground-Truth-Punkte erfasst wurden. 
Der \textbf{F-Score} kombiniert beide Metriken als harmonisches Mittel und liefert einen 
ausgewogenen Gesamtwert. Je näher der F-Score bei \textit{1.0} liegt, 
desto höher ist die Qualität der Rekonstruktion.

\paragraph{Qualitative Bewertung}
Die rekonstruierten Meshes wurden anhand folgender Kriterien bewertet:

\begin{itemize}
    \item \textbf{Vollständigkeit}: Wie viel Prozent der Szene wurde erfasst?
    \item \textbf{Detailtreue}: Sind feine Strukturen erkennbar?
    \item \textbf{Artefaktfreiheit}: Treten Löcher, Flimmern oder Fehlgeometrie auf?
    \item \textbf{Oberflächenqualität}: Glattheit und Konsistenz der Rekonstruktion
\end{itemize}

\noindent
Die Bewertung erfolgte durch visuelle Inspektion der Rekonstruktionen in Unity 
sowie durch exportierte Screenshots.

\section{Ergebnisse}
\label{sec:ergebnisse}

\subsection{Funktionale Validierung}

Die funktionale Validierung bestätigt, dass RTReconstruct alle definierten Kernfunktionalitäten erfüllt.

\paragraph{End-to-End-Kommunikation}
Die vollständige Kommunikationskette von der Fragmenterfassung im Unity-Client über 
die WebSocket-Verbindung zum Router bis zur Verteilung an die Worker-Container und 
zurück funktioniert stabil. In \textit{15} Testläufen über eine Gesamtdauer von \textit{6} Stunden 
traten \textit{0} Verbindungsabbrüche auf.

\paragraph{Parallele Modellausführung}
Alle vier integrierten Rekonstruktionsmodelle (NeuralRecon, VisFusion, MASt3R-SLAM, 
SLAM3R) konnten gleichzeitig betrieben werden. Die containerisierte Architektur 
ermöglichte eine vollständige Isolierung, sodass unterschiedliche Python- und 
PyTorch-Versionen parallel lauffähig waren. Einzig limitierender Faktor war die einzelne GPU,
die durch die Modelle gemeinsam genutzt wurde.

\paragraph{Multi-Szenen-Unterstützung}
Das System unterstützt die gleichzeitige Verarbeitung mehrerer Szenen. In Tests 
mit \textit{2} parallelen Szenen und \textit{4} verbundenen Clients blieb die Funktionalität 
erhalten. Die szenenspezifische Zuordnung der Rekonstruktionsergebnisse erfolgte 
fehlerfrei.

\paragraph{Visualisierung in VR}
Die über das Backend empfangenen Meshes wurden erfolgreich im Unity-Client 
visualisiert. Das in Kapitel 5 beschriebene Spatial 
Hashing ermöglichte eine performante Darstellung auch bei größeren Meshes und Punktwolken mit bis zu 100.000 Punkten.

\subsection{Performance-Analyse}

\subsubsection{Latenz}

Die Latenz stellt die zentrale Performance-Metrik für die Echtzeitfähigkeit des Systems dar. Im Folgenden wird zunächst die Gesamtlatenz über alle Testszenen und Modelle präsentiert, anschließend in ihre Komponenten zerlegt und abschließend durch die Analyse der Datenvolumina kontextualisiert.

\paragraph{Gesamtlatenz}

Zur Evaluierung der Systemperformance wurde die End-to-End-Latenz \\ \(L_{total}\) als Zeitspanne zwischen dem Absenden eines Fragments vom Client und dem Empfang der zugehörigen Rekonstruktion gemessen. Für jede Kombination aus Testszene und Rekonstruktionsmodell wurden drei unabhängige Testläufe durchgeführt, bei denen identische Eingabedaten verwendet wurden. Um dabei Verzerrungen durch Ressourcenkonflikte zu vermeiden, wurden die Modelle sequenziell im Einzelbetrieb getestet. Abbildung~\ref{fig:latency_boxplots} visualisiert die resultierenden Latenzverteilungen als Boxplots.

\medskip
\noindent
Die Darstellung zeigt auf der x-Achse die vier evaluierten Rekonstruktionsmodelle (NeuralRecon, VisFusion, MASt3R-SLAM, SLAM3R), während die y-Achse die gemessene Latenz in Millisekunden angibt. Pro Modell sind drei Boxplots dargestellt, die jeweils die Latenzverteilung eines Testlaufs repräsentieren. Die farbliche Kodierung kennzeichnet dabei denselben Testlauf über alle Modelle hinweg.

\begin{figure}[H]
    \begin{subfigure}{0.5\textwidth}
        \includegraphics[width=\linewidth]{images/room00_latency.png}
        \caption{Szene V1 -- Geometrische Primitive}
    \end{subfigure}
    \begin{subfigure}{0.5\textwidth}
        \includegraphics[width=\linewidth]{images/room01_latency.png}
        \caption{Szene V2 -- Möbliertes Schlafzimmer}
    \end{subfigure}
    \begin{subfigure}{0.5\textwidth}
        \includegraphics[width=\linewidth]{images/room02_latency.png}
        \caption{Szene V3 -- Komplexer Mehrzweckraum}
    \end{subfigure}
    \begin{subfigure}{0.5\textwidth}
        \includegraphics[width=\linewidth]{images/room00_latency.png}
        \caption{Szene R1 -- Schlafzimmer}
    \end{subfigure}
    \begin{subfigure}{\textwidth}
        \centering
        \includegraphics[width=0.49\linewidth]{images/room00_latency.png}
        \caption{Szene R2 -- Wohnzimmer}
    \end{subfigure}
    \caption{Gesamtlatenz \(L_{total}\) für alle Testszenen und Modelle über drei Testläufe. Die Boxen zeigen den Interquartilbereich (25.\,--\,75.\,Perzentil), die horizontale Linie den Median und die Whiskers den Wertebereich ohne Ausreißer.}
    \label{fig:latency_boxplots}
\end{figure}

\newpage
\noindent
Die Boxplots zeigen für alle Modelle und Szenen geringe Interquartilbereiche und minimale Ausreißer, was auf eine stabile und reproduzierbare Latenzcharakteristik des Systems hinweist. Aufgrund dieser geringen Varianz zwischen den drei Testläufen werden in den folgenden Analysen zur Zusammensetzung der Gesamtlatenz sowie zu Datenvolumina die Messwerte der drei Testläufe aggregiert dargestellt. Dies ermöglicht eine kompaktere Präsentation ohne relevanten Informationsverlust.

\medskip
\noindent
Die Abbildung zeigt deutliche Unterschiede in der Gesamtlatenz zwischen den Testszenen: Szene V1 weist die niedrigsten Latenzwerte auf, während die Latenz in den komplexeren Szenen V3, R1 und R2 ansteigt. Zudem variiert die Latenz zwischen den Modellen, wobei MAST3R durchgängig die höchsten Werte erreicht. Um die Ursachen dieser Variation zu identifizieren, wird die Gesamtlatenz im Folgenden in ihre Komponenten zerlegt.


\paragraph{Zusammensetzung der Gesamtlatenz}

Die beobachteten Latenzunterschiede zwischen den Testszenen lassen sich durch die Zerlegung der Gesamtlatenz in ihre konstituierenden Komponenten \(L_{\text{network}}\), \(L_{\text{inference}}\) und \(L_{\text{render}}\) analysieren. Diese Aufschlüsselung ermöglicht es, szenenabhängige Effekte auf die Inferenzzeit von fixen Overhead-Kosten der Netzwerkkommunikation und Rendering-Pipeline zu separieren. Abbildung~\ref{fig:latency_stacked_bar} visualisiert die resultierende Zusammensetzung für alle Modelle und Testszenen.

\begin{figure}[H]
    \centering
    \includegraphics[width=\textwidth]{images/latency_split.png}
    \caption{Zusammensetzung der Gesamtlatenz nach Komponenten für alle Rekonstruktionsmodelle, aufgeschlüsselt nach Testszene und gemittelt über die drei Testläufe.}
    \label{fig:latency_stacked_bar}
\end{figure}

\medskip
\noindent
Die Aufschlüsselung zeigt, dass \(L_{\text{inference}}\) den dominierenden Anteil der Gesamtlatenz ausmacht und zwischen den Szenen stark variiert. Der Anteil von \(L_{\text{network}}\) und \(L_{\text{render}}\) bleibt über die Szenen hinweg relativ konstant, nimmt jedoch prozentual mit steigender Szenenkomplexität ab. Die Netzwerklatenz \(L_{\text{network}}\) wird dabei maßgeblich durch die Größe der übertragenen Daten bestimmt, deren Quantifizierung im Folgenden dargestellt wird.


\paragraph{Fragment- und Ergebnisgrößen}

Um die Netzwerklatenz \(L_{\text{network}}\) zu kontextualisieren und die Bandbreitenanforderungen des Systems zu dokumentieren, wurden die durchschnittlichen Fragmentgrößen (Upload) und Ergebnisgrößen (Download) für alle Modelle gemessen. Tabelle~\ref{tab:data_volumes} zeigt die resultierenden Datenvolumina sowie die daraus berechnete genutzte Bandbreite.

\begin{table}[h]
    \centering
    \caption{Durchschnittliche Datenvolumina und genutzte Bandbreite aufgeschlüsselt nach Szene und Modell (gemittelt über drei Testläufe).}
    \label{tab:data_volumes}
    \resizebox{\textwidth}{!}{%
    \begin{tabular}{l|l|c|c|c|c}
        \toprule
        \textbf{Szene} & \textbf{Modell} & \makecell{\textbf{\O~$\text{Frag}_{\text{in}}$~[MB]}} & \makecell{\textbf{\O~$\text{Frag}_{\text{out}}$~[MB]}} & \makecell{\textbf{$\sum\text{Frag}_{\text{in}}$~[MB]}} & \makecell{\textbf{$\sum\text{Frag}_{\text{out}}$~[MB]}} \\
        \midrule
        \multirow{4}{*}{V1} 
            & NeuralRecon  & \multirow{4}{*}{0.82} & 1.83 & \multirow{4}{*}{19.57} & 44.04 \\
            & VisFusion    &                       & 2.73 &                        & 65.40 \\
            & MASt3R-SLAM  &                       & 1.60 &                        & 38.42 \\
            & SLAM3R       &                       & 1.60 &                        & 38.42 \\
        \midrule
        \multirow{4}{*}{V2} 
            & NeuralRecon  & \multirow{4}{*}{1.62} & 2.84 & \multirow{4}{*}{69.57} & 122.28 \\
            & VisFusion    &                      & 3.25  &                        & 139.85 \\
            & MASt3R-SLAM  &                      & 1.60  &                        & 68.60 \\
            & SLAM3R       &                      & 1.60  &                        & 68.84 \\
        \midrule
        \multirow{4}{*}{V3} 
            & NeuralRecon  & \multirow{4}{*}{1.56} & 3.60 & \multirow{4}{*}{88.70} & 204.93 \\
            & VisFusion    &                      & 4.24  &                        & 241.51 \\
            & MASt3R-SLAM  &                      & 1.59  &                        & 90.69 \\
            & SLAM3R       &                      & 1.60  &                        & 91.26 \\
        \midrule
        \multirow{4}{*}{R1} 
            & NeuralRecon  & \multirow{4}{*}{} &  & \multirow{4}{*}{} &  \\
            & VisFusion    &                      &  &                      &  \\
            & MASt3R-SLAM  &                      &  &                      &  \\
            & SLAM3R       &                      &  &                      &  \\
        \midrule
        \multirow{4}{*}{R2} 
            & NeuralRecon  & \multirow{4}{*}{} &  & \multirow{4}{*}{} &  \\
            & VisFusion    &                      &  &                      &  \\
            & MASt3R-SLAM  &                      &  &                      &  \\
            & SLAM3R       &                      &  &                      &  \\
        \bottomrule
    \end{tabular}%
    }
\end{table}


\medskip
\noindent
Die Tabelle zeigt, dass die Fragmentgrößen zwischen den Modellen variieren, wobei SLAM3R aufgrund seiner größeren Fenstergrößen die umfangreichsten Fragmente benötigt. Die Ergebnisgrößen unterscheiden sich ebenfalls deutlich: Volumetrische Verfahren (NeuralRecon, VisFusion) erzeugen größere Meshes im GLB-Format, während punktbasierte Modelle (MASt3R-SLAM, SLAM3R) kompaktere Punktwolken zurückliefern. Die Summe der genutzten Bandbreite liegt bei X.X~Mbps für den Upload und Y.Y~Mbps für den Download.

\medskip
\noindent
Die präsentierten Latenzmessungen bilden zusammen mit den Datenvolumina die Grundlage für die Bewertung der Echtzeitfähigkeit und Skalierbarkeit des Systems in Abschnitt~\ref{sec:diskussion}.

\subsubsection{Durchsatz}

Der Durchsatz ergibt sich für diese Evaluation aus der Anzahl der verarbeiteten Fragmente pro Szene, geteilt durch die mittlere End-To-End Latenz \(L_{total}\) über alle 3 Testläufe. Diese Metrik gibt an, wie viele Fragmente pro Sekunde durch das System verarbeitet werden können und ist ein Indikator für die Aktualisierungsrate der Rekonstruktion im VR-Frontend. Die folgende Tabelle \ref{tab:throughput_results} fasst diese gemessenen Durchsatzwerte für alle Modelle und Szenen zusammen.

\medskip
\begin{table}[h]
    \centering
    \label{tab:throughput_results}
    \begin{tabular}{lccccc}
        \toprule
        \textbf{Modell} & \textbf{V1} & \textbf{V2} & \textbf{V3} & \textbf{R1} & \textbf{R2}\\
        \midrule
        NeuralRecon     & 0.62 & 0.45 & 0.32 &  &  \\
        VisFusion       & 0.38 & 0.35 & 0.30 &  &  \\
        MASt3R-SLAM     & 0.13 & 0.12 & 0.12 &  &  \\
        SLAM3R          & 0.07 & 0.06 & 0.03 &  &  \\
        \bottomrule
    \end{tabular}
    \bigskip
    \caption{Durchsatz nach Modell (Fragmente pro Sekunde)}
\end{table}

\subsubsection{Ressourcenauslastung}

Die Ressourcenauslastung wurde sowohl für das Backend (Server-seitige Rekonstruktion) 
als auch für das Frontend (VR-Client-seitige Visualisierung) getrennt erfasst. Diese 
Trennung ermöglicht die Identifikation von Engpässen in der Pipeline und gibt 
Aufschluss darüber, welche Systemkomponente limitierend wirkt.

\paragraph{Backend-Ressourcen}

Die Backend-Ressourcenauslastung wurde kontinuierlich während der Rekonstruktionsläufe 
auf dem dedizierten Server (AMD Ryzen 9 5900X, NVIDIA GTX 1070 Ti) erfasst. Die 
Messungen umfassen GPU-Auslastung und GPU-Speicherverbrauch (erfasst mittels 
\texttt{nvidia-smi} in 1-Sekunden-Intervallen) sowie CPU- und RAM-Nutzung der 
containerisierten Komponenten (erfasst mittels \texttt{docker stats}).

\medskip
\noindent
\textbf{GPU-Ressourcen:} Tabelle~\ref{tab:gpu_resources} zeigt die durchschnittliche 
GPU-Utilization und den maximalen GPU-Speicherverbrauch während der Rekonstruktion, 
aufgeschlüsselt nach Modell und Testszene.

\begin{table}[h]
    \centering
    \caption{GPU-Auslastung und GPU-Speicherverbrauch während der Rekonstruktion}
    \label{tab:gpu_resources}
    \begin{tabular}{lcccccc}
        \toprule
        \textbf{Modell} & \textbf{Metrik} & \textbf{V1} & \textbf{V2} & \textbf{V3} & \textbf{R1} & \textbf{R2}\\
        \midrule
        \multirow{2}{*}{NeuralRecon} 
            & \O GPU-Utilization [\%] & \textit{44} & \textit{43} & \textit{47} & \textit{} & \textit{} \\
            & Max. VRAM [MB]          & \textit{3825} & \textit{4770} & \textit{7993} & \textit{} & \textit{} \\
        \midrule
        \multirow{2}{*}{VisFusion} 
            & \O GPU-Utilization [\%] & \textit{49} & \textit{53} & \textit{55} & \textit{} & \textit{} \\
            & Max. VRAM [MB]          & \textit{4877} & \textit{4599} & \textit{4418} & \textit{} & \textit{} \\
        \midrule
        \multirow{2}{*}{MASt3R-SLAM} 
            & \O GPU-Utilization [\%] & \textit{100} & \textit{100} & \textit{100} & \textit{} & \textit{} \\
            & Max. VRAM [MB]          & \textit{7969} & \textit{8016} & \textit{7984} & \textit{} & \textit{} \\
        \midrule
        \multirow{2}{*}{SLAM3R} 
            & \O GPU-Utilization [\%] & \textit{98.5} & \textit{99} & \textit{99} & \textit{} & \textit{} \\
            & Max. VRAM [MB]          & \textit{8022} & \textit{7986} & \textit{8012} & \textit{} & \textit{} \\
        \bottomrule
    \end{tabular}
\end{table}

\medskip
\noindent
\textbf{CPU- und RAM-Auslastung:} Die Ressourcennutzung der einzelnen Docker-Container 
ist in Tabelle~\ref{tab:container_resources} dargestellt. Die Werte zeigen die 
durchschnittliche CPU-Last und den RAM-Verbrauch für Router- und Worker-Container, 
gemittelt über alle Testszenen.

\begin{table}[h]
    \centering
    \caption{Durchschnittliche CPU- und RAM-Auslastung der Backend-Container}
    \label{tab:container_resources}
    \begin{tabular}{lcc}
        \toprule
        \textbf{Container} & \textbf{\O CPU-Auslastung [\%]} & \textbf{\O RAM-Verbrauch [MB]} \\
        \midrule
        Router                    & \textit{0.2} & \textit{200} \\
        \midrule
        Worker: NeuralRecon       & \textit{32} & \textit{2780} \\
        Worker: VisFusion         & \textit{17.25} & \textit{3015} \\
        Worker: MASt3R-SLAM       & \textit{14.35} & \textit{3466} \\
        Worker: SLAM3R            & \textit{15.37} & \textit{4070} \\
        \bottomrule
    \end{tabular}
\end{table}

\paragraph{Frontend-Ressourcen}

Die durchschnittliche Frame Rate des Unity-Clients wurde sowohl im Baseline-Betrieb 
(ohne aktive Rekonstruktion) als auch während der Rekonstruktions- und 
Visualisierungsphase gemessen. Tabelle~\ref{frame_rate} zeigt die Ergebnisse.

\begin{figure}[H]
    \centering
    \label{frame_rate}
    % \caption{FPS des Unity-Clients im Baseline-Betrieb und während der Rekonstruktion}
    \includegraphics[width=0.65\textwidth]{images/frontend_fps.png}
\end{figure}

\subsection{Rekonstruktionsqualität}

\subsubsection{Quantitative Bewertung}

Tabelle \ref{tab:fscore_all} zeigt die F-Score-Ergebnisse für alle Szenen mit verfügbarem Ground-Truth für einen Schwellenwert von \textit{10}cm.

\medskip
\noindent
Die dafür benötigten Daten wurden mit der quelloffenen Software CloudCompare gesammelt, indem zunächst die Cloud-to-Cloud Distance-Funktion verwendet wurde, um für jeden Punkt der Rekonstruktion den Abstand zum nächsten Ground-Truth-Punkt zu berechnen. Anschließend wurden die Punkte über \textit{Filter Points by Value} anhand des 10\,cm Schwellenwerts klassifiziert: Abstände \(\leq 10\)\,cm ergeben True Positives, größere Abstände False Positives. False Negatives wurden durch die umgekehrte Distanzberechnung von Ground-Truth zu Rekonstruktion ermittelt, woraus Precision, Recall und F-Score berechnet wurden.


\begin{table}[H]
    \centering
    \label{tab:fscore_all}
    \caption{F-Score-Ergebnisse nach Modell und Szene}
    \begin{tabular}{lccc|ccc|ccc}
        \toprule
        & \multicolumn{3}{c}{\textbf{V1}} & \multicolumn{3}{c}{\textbf{V2}} & \multicolumn{3}{c}{\textbf{V3}} \\
        \cmidrule(lr){2-4} \cmidrule(lr){5-7} \cmidrule(lr){8-10}
        \textbf{Modell} & \textbf{Prec.} & \textbf{Rec.} & \textbf{F-Score} & \textbf{Prec.} & \textbf{Rec.} & \textbf{F-Score} & \textbf{Prec.} & \textbf{Rec.} & \textbf{F-Score} \\
        \midrule
        NeuralRecon     & \textit{0.45} & \textit{0.39} & \textit{0.41} & \textit{0.69} & \textit{0.67} & \textit{0.68} & \textit{0.59} & \textit{0.59} & \textit{0.59} \\
        VisFusion       & \textit{0.57} & \textit{0.52} & \textit{0.54} & \textit{0.57} & \textit{0.65} & \textit{0.61} & \textit{0.57} & \textit{0.68} & \textit{0.62} \\
        SLAM3R          & \textit{0.66} & \textit{0.56} & \textit{0.61} & \textit{0.67} & \textit{0.59} & \textit{0.63} & \textit{0.41} & \textit{0.43} & \textit{0.42} \\
        MASt3R-SLAM     & \textit{0.52} & \textit{0.48} & \textit{0.50} & \textit{0.47} & \textit{0.53} & \textit{0.50} & \textit{0.43} & \textit{0.48} & \textit{0.45} \\
        \bottomrule
    \end{tabular}
\end{table}

\subsubsection{Qualitative Bewertung}

\paragraph{NeuralRecon}
NeuralRecon lieferte robuste Gesamtmodelle mit glatter Oberflächenqualität. Die Rekonstruktionen zeigten eine hohe globale Konsistenz durch die GRU-basierte TSDF-Fusion über mehrere Frames hinweg. Besonders in feature-armen Bereichen wie Wänden (Szene V2, V3) produzierte NeuralRecon saubere, artefaktfreie Oberflächen. Scharfe Kanten (Szene V1) wurden tendenziell durch die volumetrische Glättung abgerundet.

\paragraph{VisFusion}
VisFusion zeigte im direkten Vergleich zu NeuralRecon deutliche Verbesserungen in der Detailtreue. Die visibility-aware Merkmalsfusion ermöglichte eine bessere Handhabung von Okklusionen, was sich besonders in den komplexen Szenen V2 und V3 mit überlagerten Möbeln manifestierte. Die ray-based sparsification führte zu vollständigeren Rekonstruktionen mit erhaltenen feinen Strukturen wie Bilderrahmen und Lampendetails. Wandoberflächen wirkten kohärenter als bei NeuralRecon.

\paragraph{SLAM3R}
SLAM3R zeichnete sich durch sehr dichte Punktwolken aus, die direkt aus dem Image-to-Points-Netzwerk generiert wurden. Die feed-forward-Architektur ermöglichte die höchste Verarbeitungsgeschwindigkeit (20+ FPS). Allerdings war die Vollständigkeit geringer als bei volumetrischen Verfahren – insbesondere in schwer einsehbaren Bereichen wie unter Möbeln entstanden Lücken. Die Local-to-World-Registrierung zeigte in Szene V3 vereinzelt Inkonsistenzen bei der Ausrichtung aufeinanderfolgender Fragmente.

\paragraph{MASt3R-SLAM}
MASt3R-SLAM erzeugte detaillierte, dichte Punktwolken mit hoher lokaler Genauigkeit. Die transformer-basierte Feature-Extraktion ermöglichte robustes Matching auch bei texturarmen Oberflächen. In den realen Szenen R1 und R2 zeigte sich jedoch gelegentlich Drift bei längeren Trajektorien, erkennbar an leichten Verschiebungen zwischen Rekonstruktionsfragmenten verschiedener Zeitpunkte. Die punktbasierte Darstellung erreichte nicht die Oberflächenkohärenz der volumetrischen Verfahren.

\paragraph{Vergleichende Bewertung}
Tabelle \ref{tab:qualitative_comparison} fasst die qualitativen Beobachtungen zusammen und bewertet sie entweder negativ, neutral oder positiv gegnüber den anderen Modellen.

\begin{table}[h]
    \centering
    \label{tab:qualitative_comparison}
    \caption{Qualitative Bewertung der Rekonstruktionsmodelle}
    \begin{tabular}{lcccc}
        \toprule
        \textbf{Kriterium} & \textbf{NeuralRecon} & \textbf{VisFusion} & \textbf{MASt3R} & \textbf{SLAM3R} \\
        \midrule
        Vollständigkeit    & \textit{[+/o/-]} & \textit{[+/o/-]} & \textit{[+/o/-]} & \textit{[+/o/-]} \\
        Detailtreue        & \textit{[+/o/-]} & \textit{[+/o/-]} & \textit{[+/o/-]} & \textit{[+/o/-]} \\
        Artefaktfreiheit   & \textit{[+/o/-]} & \textit{[+/o/-]} & \textit{[+/o/-]} & \textit{[+/o/-]} \\
        Oberflächenqualität & \textit{[+/o/-]} & \textit{[+/o/-]} & \textit{[+/o/-]} & \textit{[+/o/-]} \\
        \bottomrule
    \end{tabular}
    \medskip
\end{table}

\subsection{Modularität und Systemstabilität}

\subsubsection{Integrationsaufwand neuer Modelle}

Um die Modularität der Architektur zu bewerten, wurde ein \textit{[fünftes Modell/Mock-Modell]} 
in das System integriert. Der Integrationsaufwand umfasste:

\begin{itemize}
    \item Erstellung eines Dockerfile: \textit{[X]} Zeilen Code
    \item Implementierung des Worker-Interfaces: \textit{[Y]} Zeilen Code
    \item Anpassung der docker-compose.yml: \textit{[Z]} Zeilen
    \item Gesamtdauer: \textit{[A]} Stunden
\end{itemize}

\noindent
Dies bestätigt, dass die einheitliche Schnittstelle den Integrationsaufwand 
erheblich reduziert.

\subsubsection{Stabilitätstests}

Das System wurde unter folgenden Fehlerszenarien getestet:

\paragraph{Container-Neustart während Rekonstruktion}
Ein Worker-Container wurde manuell gestoppt und nach \textit{[X]} Sekunden neu gestartet. 
Das System erkannte den Ausfall und \textit{[Beschreibung des Verhaltens]}. Die 
Wiederherstellungszeit betrug \textit{[Y]} Sekunden.

\paragraph{Netzwerkunterbrechung}
Die WiFi-Verbindung wurde für \textit{[X]} Sekunden unterbrochen. Das Frontend behielt 
das letzte Mesh \textit{[bei/zeigte Fehlermeldung]}. Nach Wiederherstellung der Verbindung 
konnte die Rekonstruktion \textit{[fortgesetzt/musste neu gestartet]} werden.

\paragraph{Gleichzeitige Last}
Bei parallelem Betrieb von \textit{[X]} Clients und \textit{[Y]} Szenen blieb die Performance 
\textit{[stabil/verschlechterte sich um Z\%]}. Die Latenz stieg auf durchschnittlich \textit{[A]} ms.

\section{Diskussion}
\label{sec:diskussion}

\subsection{Echtzeitfähigkeit}

Die gemessenen Latenzen von \textit{[Bereich]} zeigen, dass der Hauptanteil der Verzögerung 
auf die Modell-Inferenz entfällt (\textit{[X]}\%). Die WebSocket-Kommunikation und 
Container-Architektur verursachen nur einen geringen Overhead von \textit{[Y]}\%. Dies 
bestätigt, dass die gewählte Architektur keine signifikanten Performance-Einbußen 
gegenüber monolithischen Systemen mit sich bringt.

Die Unterschiede zwischen den Modellen lassen sich auf ihre unterschiedlichen 
Rekonstruktionsparadigmen zurückführen: Volumetrische Verfahren wie 
\textit{[NeuralRecon/VisFusion]} erzielen \textit{[schnellere/langsamere]} Inferenzzeiten aufgrund 
\textit{[Begründung: z.B. ihrer TSDF-basierten Fusion]}, während SLAM-basierte Ansätze 
\textit{[Charakteristik: z.B. durch Feature-Matching höhere Latenzen]} aufweisen.

Der gemessene Durchsatz von \textit{[X]} Fragmenten pro Sekunde ermöglicht \textit{[häufige/seltene]} 
Updates der Rekonstruktion. Die Frame Rate im Unity-Client von durchschnittlich 
\textit{[Y]} fps liegt \textit{[im akzeptablen/grenzwertigen]} Bereich für VR-Anwendungen.

\subsection{Modularität der Architektur}

Die erfolgreiche parallele Ausführung aller vier Modelle sowie der geringe 
Integrationsaufwand für neue Modelle (\textit{[X]} Stunden) belegen die hohe Modularität 
des Systems. Die containerisierte Architektur ermöglicht die Isolierung 
unterschiedlicher Laufzeitumgebungen, was die Erweiterbarkeit erheblich vereinfacht.

Die Stabilitätstests zeigen, dass das System \textit{[Szenario A wie Container-Neustarts]} 
robust handhabt. Bei \textit{[Szenario B wie Netzwerkunterbrechungen]} zeigten sich jedoch 
\textit{[Schwächen wie Datenverlust]}, die für produktive Einsätze adressiert werden müssten.

\subsection{Rekonstruktionsqualität im Vergleich}

Der Vergleich der Modelle zeigt erwartungsgemäß unterschiedliche Stärken: 
Volumetrische Verfahren lieferten \textit{[Charakteristik wie robuste Gesamtmodelle mit 
hoher Vollständigkeit]}, während SLAM-basierte Ansätze \textit{[Charakteristik wie feinere 
lokale Details bei geringerer globaler Konsistenz]} aufwiesen. Dies entspricht den 
in Kapitel \textbf{[REF: chap:stand\_der\_technik]} beschriebenen methodischen Unterschieden.

Die quantitativen Ergebnisse zeigen, dass \textit{[Modellname]} mit einem F-Score von \textit{[X]} 
die beste geometrische Genauigkeit erreicht, während \textit{[anderes Modell]} bei \textit{[Kriterium]} 
überzeugt. Die qualitative Bewertung unterstreicht, dass \textit{[Modellname]} für Szenarien 
mit \textit{[Merkmal]} besonders geeignet ist, während \textit{[anderes Modell]} in \textit{[anderen Szenarien]} 
Vorteile bietet.

\subsection{Eignung für VR-Anwendungen}

Die Evaluationsergebnisse zeigen, dass RTReconstruct die definierten funktionalen 
Anforderungen erfüllt und für interaktive VR-Anwendungen \textit{[grundsätzlich geeignet/
mit Einschränkungen nutzbar]} ist. Die Latenzwerte von \textit{[Bereich]} liegen \textit{[im/knapp 
über dem]} für VR akzeptablen Bereich. Die stabile Kommunikation und die erfolgreiche 
Multi-Szenen-Unterstützung bestätigen die Praxistauglichkeit der Architektur für 
das Va.Si.Li-Lab.

Die modulare Struktur ermöglicht es, je nach Anwendungsfall das optimale 
Rekonstruktionsverfahren auszuwählen: \textit{[Szenario A]} profitiert von \textit{[Modellname]}, 
während \textit{[Szenario B]} mit \textit{[anderem Modell]} bessere Ergebnisse erzielt.
\chapter{Zusammenfassung und Ausblick}

In dieser Arbeit wurde die Herausforderung adressiert, mehrere heterogene Echtzeit-3D-Rekonstruktionsverfahren in eine bestehende Virtual-Reality-Umgebung zu integrieren. Die zentrale Forschungsfrage lautete dabei: \textit{Wie gut eignet sich eine modulare, containerisierte Systemarchitektur zur Integration verschiedener Echtzeit-3D-Rekonstruktionsverfahren in eine bestehende Virtual-Reality-Umgebung?} Zur Beantwortung dieser Frage wurde das System RTReconstruct konzipiert, implementiert und experimentell evaluiert.

\section{Zusammenfassung der Ergebnisse}

\subsection{Konzeption und Implementierung}
Die entwickelte Architektur trennt konsequent zwischen einem Unity-basierten Frontend, einem zentralen Backend-Router und containerisierten Rekonstruktions-Workern. Das Frontend erfasst kontinuierlich Kamera- und Posendaten, bündelt diese zu zeitlichen Fragmenten und visualisiert die empfangenen 3D-Rekonstruktionen in Echtzeit, unabhängig von anderen Komponenten der Szene. Der Backend-Router empfängt und verteilt eingehende Fragmente über ein performantes (vgl.~\ref{tab:data_volumes}), binäres WebSocket-Protokoll asynchron an die zuständigen Worker und verwaltet die aktuellsten Rekonstruktionsergebnisse pro Szene und Modell. Jeder Worker kapselt ein Rekonstruktionsverfahren (NeuralRecon, VisFusion, MASt3R-SLAM, SLAM3R) in einem eigenständigen Docker-Container und kommuniziert über eine einheitliche Schnittstelle.

Die Implementierung zeigt, dass der Integrationsaufwand für neue Modelle mit durchschnittlich 47--66 Zeilen Dockerfile und 205--401 Zeilen Worker-Code gering ausfällt (vgl.~\ref{tab:integration_effort}) und dabei größtenteils modellunabhängig bleibt. Die Integration in das Va.Si.Li-Lab erfolgte als additive, nicht-invasive Erweiterung durch vorkonfigurierte Unity-Prefabs (ReconstructionManager, RoomReconstructor, ReconstructionClient), die alle erforderlichen Komponenten kapseln. Das System verhält sich als passiver, zuschaltbarer Dienst ohne Eingriff in die bestehende Mehrbenutzerlogik des Labs, da beide Systeme sich lediglich die gemeinsame Grundlage des Meta SDK zur Erfassung von Kamera- und Posendaten teilen. Ein einfaches Rollen- und Szenenkonzept (Host/Visitor) ermöglicht die clientseitige Verwaltung der Rekonstruktionsdaten, während die Mehrbenutzerumgebung des Labs vollständig funktionsfähig bleibt.

\subsection{Evaluationsergebnisse}

\subsubsection{Funktionale Validierung}
Die End-to-End-Kommunikation von der Fragmenterfassung im Unity-Client über die \\ WebSocket-Verbindung zum Router bis zur Verteilung an die Worker-Container und zurück funktionierte in 60 Testläufen über eine Gesamtdauer von 6 Stunden stabil mit 0~Verbindungsabbrüchen (vgl.~\ref{par:e2e_functional}). Alle vier integrierten Rekonstruktionsmodelle (NeuralRecon, VisFusion, MASt3R-SLAM, SLAM3R) konnten erfolgreich parallel betrieben werden, wobei die containerisierte Architektur eine vollständige Isolierung der Laufzeitumgebungen gewährleistete. Das System unterstützte die gleichzeitige Verarbeitung von 2~parallelen Szenen mit 4~verbundenen Clients ohne Funktionalitätsverlust, und die szenenspezifische Zuordnung der Rekonstruktionsergebnisse erfolgte fehlerfrei (vgl.~\ref{par:parallel_execution}). Die empfangenen Meshes und Punktwolken wurden durch Chunking und GPU-beschleunigtes Rendering performant im VR-Headset visualisiert.

\subsubsection{Echtzeitfähigkeit und Performance}
Die Performance-Evaluierung über fünf Testszenen (drei virtuelle, zwei reale) demonstriert deutliche modellspezifische Unterschiede. Volumetrische Verfahren erreichen Durchsatzraten von 0{,}30--0{,}62 Fragmenten pro Sekunde bei einer durchschnittlichen Gesamtlatenzen zwischen 2 und 5 Sekunden. SLAM-basierte Verfahren weisen mit 0{,}03--0{,}13 Fragmenten pro Sekunde einen viel geringeren Durchsatz auf, zeigen jedoch höhere Robustheit gegenüber SDK-Limitationen der Meta Quest 3 (fehlende Frame-Timestamps). Die Kommunikationslatenz (Netzwerk + Rendering) bleibt mit typisch unter 1000~ms gering, wodurch die Gesamtlatenz nahezu linear mit der Modell-Inferenzzeit skaliert (vgl.~\ref{fig:latency_stacked_bar}), und damit ein zentrales Erfolgskriterium modularer Architekturen erfüllt.

Auf Backendsseite blieb der Router mit durchschnittlich 20\% CPU-Last und rund 100~MB RAM-Verbrauch (vgl.~\ref{tab:container_resources}) praktisch unauffällig, während die GPU-Worker die verfügbaren Ressourcen deutlich mehr ausnutzten. Volumetrische Verfahren wie NeuralRecon und VisFusion belegten zwischen etwa 3{,}8~GB und 6{,}0~GB VRAM bei moderaten GPU-Utilization-Werten von 44--56\%, wohingegen die SLAM-basierten Ansätze MASt3R-SLAM und SLAM3R die NVIDIA GTX 1070 Ti mit 98--100\% GPU-Auslastung und nahezu voll belegten 8~GB VRAM dauerhaft an das hardwareseitige Limit brachten (vgl.~Tab.~\ref{tab:gpu_resources}). Dieses GPU-Bottleneck der SLAM-Verfahren begrenzt insbesondere den parallelen Betrieb mehrerer Modelle und erklärt die im Vergleich zu volumetrischen Methoden deutlich niedrigeren Durchsatzraten.

\subsubsection{Rekonstruktionsqualität}
Die vergleichende Qualitätsevaluation offenbarte eine ausgeprägte Leistungsumkehr zwischen virtuellen und realen Testumgebungen. In kontrollierten Unity-Szenen mit Ground-Truth erzielten volumetrische Verfahren höhere F-Scores als SLAM-Ansätze (vgl.~\ref{tab:fscore_all}), während sich diese Relation unter realen VR-Bedingungen mit der Meta Quest 3 umkehrte: SLAM-Verfahren zeigten visuell überlegene Vollständigkeit und Detailtreue, volumetrische Modelle hingegen fragmentierte Rekonstruktionen mit erheblichem Detailverlust (vgl. Abb.\\ \ref{fig:real_volumetric}, \ref{fig:real_slam}). Diese Diskrepanz ist primär auf SDK-bedingte Tracking-Ungenauigkeiten der Quest 3 Passthrough-API zurückzuführen - fehlende präzise Frame-Zeitstempel verursachen zeitliche Inkonsistenzen, die volumetrische TSDF-Fusion stark beeinträchtigen, während \\ SLAM-Verfahren durch interne Pose-Estimation robuster gegenüber diesen Hardwarelimitationen sind.

\subsection{Beantwortung der Forschungsfrage}
Die Ergebnisse der Evaluation zeigen, dass sich die entwickelte Architektur \textit{gut} bis \textit{sehr gut} für Forschungs- und Evaluationskontexte in VR-Umgebungen eignet. Dabei wurden drei zentrale Stärken validiert:

\begin{enumerate}
    \item \textbf{Modularität} \\
    Die erfolgreiche Integration von vier methodisch unterschiedlichen Rekonstruktionsverfahren mit heterogenen Framework-Abhängigkeiten und Repräsentationsformen \\ (vgl.~\ref{tab:model_comparison}) bestätigt die Containerisierung als praktikabel und skalierbar.
    
    \item \textbf{Vergleichbarkeit} \\
    Die Architektur ermöglicht systematische Vergleichsstudien unter identischen Bedingungen. Die identifizierte Hardware-Sensitivität volumetrischer Verfahren wäre ohne diese kontrollierte Umgebung kaum systematisch erfassbar gewesen.
    
    \item \textbf{Integrationsfähigkeit} \\
    Die additive Integration in das Va.Si.Li-Lab erfolgte ohne disruptive Eingriffe in die Mehrbenutzerlogik. RTReconstruct fungiert als optionaler, zuschaltbarer Dienst und erweitert die Plattform um räumlichen Kontext.
\end{enumerate}

Außerdem konnten Deployment-spezifische Trade-offs klar ermittelt werden. Volumetrische Verfahren eignen sich für kontrollierte Setups mit präzisen Kamera-Extrinsiken. SLAM\-Verfahren liefern hingegen trotz höherer Latenz robustere Ergebnisse mit Consumer-Hardware (Meta Quest 3).

\section{Limitationen}

Trotz erfolgreicher Validierung der Kernfunktionalitäten bestehen einige Einschränkungen:

\begin{itemize}
    \item \textbf{Hardware Limitationen} \\
    Die verwendete GPU (GTX 1070 Ti, 8\,GB VRAM) begrenzt die Anzahl parallel effektiv betreibbarer Worker und verhindert die Evaluation neuerer, speicherintensiver Modelle. Moderne Hardware (z.\,B. RTX 4090) könnte die Inferenzzeiten signifikant reduzieren.
    
    \item \textbf{Fehlende inkrementelle Updates} \\
    Das System überträgt vollständige Meshes und Punktwolken bei jedem Update, statt nur geometrische Differenzen. Dies führt zu redundanter Bandbreitennutzung, insbesondere bei volumetrischen Verfahren mit großen GLB-Modellen, und limitiert die praktische Update-Frequenz in drahtlosen VR-Szenarien unnötig.

    \item \textbf{Punktdichte} \\
    Das hardwarebedingte Limit von 100\,000 Punkten (Meta Quest 3) erzwingt einen Trade-off zwischen Raumgröße und Detailtreue: Während kleine Räume mit hoher Punktdichte rekonstruiert werden, sinkt die Auflösung bei großen Szenen, wodurch feine Details verloren gehen.
    
    \item \textbf{Maßstabsinkonsistenz} \\
    SLAM-basierte Verfahren erzeugen Rekonstruktionen in modellinternen Koordinatensystemen mit willkürlicher Skalierung. Die Integration in Unity erfordert derzeit manuelle Ausrichtung, da keine automatische Registration mit dem VR-Tracking-System implementiert wurde.
    
    \item \textbf{Fehlende Persistenz} \\
    Container-Neustarts führen zum Verlust aller Rekonstruktionszustände, da keine persistente Speicherung interner Modellzustände vorgesehen ist. Dies limitiert die Eignung für Langzeit-Rekonstruktionsszenarien.
    
    \item \textbf{SDK-Limitationen} \\
    Die fehlenden Frame-Timestamps und ungenauen Extrinsiken der Meta Quest 3 SDK beeinträchtigen volumetrische Verfahren systematisch. Diese Limitation ist aktuell API-Spezifisch und würde bei Einsatz präziserer Kamera-Systeme entfallen.
\end{itemize}

\section{Zukünftige Arbeiten}

Aufbauend auf den Ergebnissen und identifizierten Limitationen lassen sich Erweiterungen in vier Kategorien gliedern: Architekturoptimierungen zur Reduktion von Latenz und Ressourcenbedarf, funktionale Erweiterungen für verbesserte Nutzerinteraktion und Visualisierung, Modellintegration neuerer Rekonstruktionsverfahren sowie wissenschaftliche Perspektiven zur systematischen Evaluation und Nutzerstudien.

\subsection{Architekturoptimierungen}

\subsubsection{Inkrementelle Datenübertragung}
Statt vollständiger GLB-Modelle könnten differenzielle Updates nur geänderter Segmente übertragen werden, wodurch Latenz und Bandbreitenbedarf deutlich reduziert würden. Ansätze wie progressive Meshes oder Octree-basierte Partitionierung ermöglichen das selektive Nachladen sichtbarer oder veränderter Bereiche und würden flüssigere Szenenaktualisierungen in drahtlosen VR-Szenarien ermöglichen.

\subsubsection{Persistente Zustandsverwaltung}
Eine Checkpoint-Mechanik würde die Rekonstruktionszustände der Worker-Container zwischen Sessions speichern und deren Fortsetzung ermöglichen. Durch standardisierte Serialisierungsformate und Cloud-Storage könnten Rekonstruktionen versioniert und zwischen verschiedenen Sitzungen wiederhergestellt werden, was insbesondere für längerfristige \\ Dokumentations- und Analyseprojekte wertvoll wäre.

\subsubsection{Verteilte Rekonstruktion auf mehreren GPUs}
Eine Orchestrierung über mehrere GPU-Server würde echte Skalierbarkeit ermöglichen: Mehrere Szenen könnten parallel auf dedizierter Hardware rekonstruiert werden, ohne dass sich Worker-Prozesse GPU-Ressourcen teilen müssen. Dies würde insbesondere den gleichzeitigen Betrieb vieler VR-Clients ermöglichen.

\subsection{Funktionale Erweiterungen}

\subsubsection{Automatische Koordinaten-Registrierung}
Bei Session-Start könnten volumetrische und SLAM-basierte Rekonstruktionen parallel ausgeführt werden. Das volumetrische Verfahren liefert dabei automatisch die korrekte Skalierung und Ausrichtung, gegen die SLAM-Rekonstruktionen anschließend automatisch mittels ICP registriert werden könnten. Dies würde die manuelle Nachregistrierung obsolet machen und den Workflow erheblich vereinfachen.

\subsubsection{Adaptive Punktwolken-Visualisierung}
Eine dynamische Anpassung der Punktdichte basierend auf Kameradistanz und sichtbarem Bereich würde die 100\,000-Punkte-Grenze effektiver nutzen. Level-of-Detail-Mechanismen könnten nähere Regionen hochauflösend darstellen, während entfernte Bereiche mit geringerer Dichte visualisiert würden. Dies würde hochauflösende Rekonstruktionen auch auf der Meta Quest 3 praktikabel machen.

\subsubsection{Echtzeit-Nutzerinteraktion}
Interaktive Werkzeuge wie Annotationen, selektives Nachladen von Szenenregionen oder manuelle Geometriekorrektur würden kollaborative Rekonstruktions-Workflows ermöglichen. Dies wäre besonders für das Va.Si.Li-Lab wertvoll, wo Nutzer Szenen gemeinsam analysieren und dokumentieren könnten.

\subsection{Modellintegration}

\subsubsection{Gaussian Splatting}
Moderne Gaussian-Splatting-Verfahren könnten fotorealistischere und hochwertigere Rekonstruktionen mit effizienteren Rendering-Eigenschaften ermöglichen als aktuelle Mesh- oder Punktwolkenansätze. Für sie wäre allerdings eine neue Visualisierungs-Pipeline notwendig. Die native Unterstützung für viewabhängige Effekte (Reflexionen, Transparenzen) wäre besonders in komplexen realen Szenen von Vorteil.

\subsubsection{NeRF-basierte Verfahren}
Als alternative Rekonstruktionsansätze könnten moderne NeRF-Implementierungen (Instant-NGP, TensoRF) evaluiert werden. Ihr Vorteil liegt darin, dass sich bereits mit wenigen Aufnahmen hochwertige Ergebnisse erzielen lassen. Dies wäre für spontane VR-Szenarien wertvoll, bei denen eine schnelle Erfassung ohne lange Aufnahmephase erforderlich ist. Allerdings stellen die Trainingszeiten nach wie vor eine praktische Hürde für die Integration in Echtzeit dar.

\subsection{Wissenschaftliche Perspektiven}

\subsubsection{Systematische Benchmark-Suite}
Die Infrastruktur könnte zu einer Benchmark-Suite für VR-Rekonstruktionen ausgebaut werden. Dadurch wären faire Vergleiche zwischen Verfahren über verschiedene Hardware und Szenentypen hinweg möglich. So ließen sich Hardwarebedarf und spezifische Modell-Eignung systematisch ableiten.

\subsubsection{User Studies}
In Nutzerstudien könnte untersucht werden, wie Nutzer Rekonstruktionsartefakte wahrnehmen und welche Update-Frequenzen sie als natürlich empfinden. Auch die Akzeptanz verschiedener Repräsentationstypen (Mesh vs. Punktwolke) wäre ein interessantes Thema. Solche Erkenntnisse würden zu UX-Design-Richtlinien für Rekonstruktionssysteme in VR führen, die auf empirischen Daten basieren.

\section{Abschließende Bewertung}
Die Arbeit zeigt, dass containerisierte, modular aufgebaute Architekturen für die Echtzeit-3D-Rekonstruktion in VR technisch realisierbar, praktikabel und wissenschaftlich wertvoll sind. Das entwickelte System RTReconstruct stellt eine erweiterbare Plattform dar, die systematische Vergleiche heterogener Rekonstruktionsverfahren unter identischen Bedingungen ermöglicht. Eine Lücke, die bisherige monolithische End-to-End-Systeme nicht adressieren.

Die Ergebnisse bilden die Grundlage für weiterführende Forschungen zu interaktiven, verteilten Rekonstruktionssystemen in immersiven Lern- und Forschungsumgebungen. Ein Anwendungsfeld, das bislang wenig systematisch erforscht wurde.


\appendix

\printbibliography

\end{document}
