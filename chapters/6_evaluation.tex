\chapter{Evaluation}
\label{chap:evaluation} % Hinzugefügt, falls andere Kapitel darauf verweisen

Dieses Kapitel evaluiert das entwickelte RTReconstruct-System hinsichtlich seiner 
Echtzeitfähigkeit, Modularität und Rekonstruktionsqualität. Zunächst werden die 
Evaluationsziele definiert und die Testumgebung beschrieben. Anschließend erfolgt 
die Darstellung der gemessenen Performance-Metriken sowie der Rekonstruktionsqualität 
der integrierten Modelle. Das Kapitel schließt mit einer Diskussion der Ergebnisse 
im Kontext der definierten Anforderungen.

\section{Evaluationsziele}
\label{sec:eval_ziele}

Die Evaluation verfolgt drei zentrale Ziele, die direkt aus der in Abschnitt formulierten Forschungsfrage abgeleitet sind:

\begin{enumerate}
    \item \textbf{Funktionale Validierung}: Nachweis, dass die modulare Architektur 
    die definierten funktionalen Anforderungen erfüllt. Insbesondere wird geprüft, 
    ob verschiedene Rekonstruktionsmodelle parallel betrieben werden können und ob 
    die End-to-End Kommunikation zwischen VR-Frontend und Backend stabil funktioniert.
    
    \item \textbf{Echtzeitfähigkeit}: Bewertung, ob das System die für VR-Anwendungen 
    erforderlichen Performance-Anforderungen erfüllt. Dabei werden Latenz, Durchsatz 
    und Ressourcenauslastung als kritische Metriken untersucht.
    
    \item \textbf{Rekonstruktionsqualität}: Qualitative und -- soweit möglich -- 
    quantitative Bewertung der von den integrierten Modellen erzeugten 
    3D-Rekonstruktionen unter identischen Bedingungen.
\end{enumerate}

\section{Evaluationsmethodik}
Die Evaluationsmethodik beschreibt die Testumgebung, die verwendeten Testszenarien und die Messmethoden, die zur Erreichung der Evaluationsziele eingesetzt wurden. 

\subsection{Test- und Evaluationsumgebung}
\label{sec:testumgebung}

Um die \textbf{Reproduzierbarkeit} der Performance-Messungen und die \textbf{Vergleichbarkeit} der erzielten Ergebnisse zu gewährleisten, wurde die gesamte Evaluation in einer dedizierten und \textbf{kontrollierten Hard- und Software-Umgebung} durchgeführt. Die zentralen Komponenten und Spezifikationen dieser Umgebung sind in Tabelle~\ref{tab:hardware_and_software} zusammengefasst.

\begin{table}[H]
    \centering
    \label{tab:hardware_and_software}
    \caption{Spezifikationen der Hard- und Software-Umgebung}
    \begin{tabularx}{\textwidth}{l X}
        \toprule
        \textbf{Kategorie} & \textbf{Details und Spezifikationen} \\
        \midrule
        \multicolumn{2}{l}{\textbf{Hardware-Umgebung (Backend/Server)}} \\
        \midrule
        Backend-Server CPU & AMD Ryzen 9 5900X (12 Kerne, 24 Threads) \\
        GPU & NVIDIA GeForce GTX 1070 Ti, 8 GB VRAM \\
        VR-System (Frontend) & Meta Quest 3 \\
        Netzwerk & WiFi 6 Heimnetzwerk \\
        \midrule
        \multicolumn{2}{l}{\textbf{Software-Umgebung und Frameworks}} \\
        \midrule
        Betriebssystem & Ubuntu 22.04 LTS (Host) \\
        Containerisierung & Docker (\textit{[Version einfügen, z.B. 24.0.7]}) \\
        GPU-Unterstützung & NVIDIA Container Toolkit \\
        VR-Frontend & Unity (\textit{2022.3 LTS}) \\
        \bottomrule
    \end{tabularx}
\end{table}

Alle Messungen erfolgten unter \textbf{kontrollierten Bedingungen}. Es wurde strikt darauf geachtet, dass während der Performance-Tests \textbf{keine weiteren rechenintensiven Hintergrundprozesse} liefen, um Verzerrungen zu minimieren.

\subsection{Testszenarien und Datensätze}

Für die Evaluation wurden fünf Testszenen mit unterschiedlichen Komplexitätsstufen 
konzipiert: drei virtuelle Szenen mit verfügbarem Ground-Truth zur quantitativen 
Bewertung sowie zwei reale Szenen aus einem typischen Alltagsumfeld zur Validierung der
Praxistauglichkeit. Eine kurze Übersicht über alle Testszenen findet sich in Tabelle~\ref{tab:test_scenes_overview}.
\begin{table}[H]
    \centering
    \label{tab:test_scenes_overview}
    \begin{tabularx}{\textwidth}{lcclX}
        \toprule
        \textbf{Szene} & \textbf{Größe (m)} & \textbf{Fragments} & \textbf{Frames} & \textbf{Merkmale} \\
        \midrule
        \multicolumn{5}{l}{\textbf{Virtuelle Szenen}} \\
        V1 -- Primitive & $5\times 5\times 5$ & 24 & 216 & Geometrische Grundformen, unifarbige Oberflächen \\
        V2 -- Schlafzimmer & $6\times 5\times 3$ & 44 & 396 & moderate Komplexität, Okklusionen \\
        V3 -- Mehrzweckraum & $10\times 5\times 3$ & 58 & 522 & Hohe Dichte, komplexe Geometrie \\
        \midrule
        \multicolumn{5}{l}{\textbf{Reale Szenen}} \\
        R1 & $6\times 4\times 3$ & 48 & 432 & \textit{Schlafzimmer}, Details, Schrägen, Okklusionen \\
        R2 & $7.5\times 5\times 3$ & 43 & 387 & \textit{Wohnzimmer}, Glas, Reflexionen, große Flächen \\
        \bottomrule
    \end{tabularx}
    \medskip
    \caption{Übersicht und Klassifikation der Testszenen}
\end{table}

\subsubsection{Virtuelle Szenen}

Die drei virtuellen Szenen wurden in Unity erstellt und ermöglichen durch verfügbare \textit{Ground-Truth-Meshes} eine quantitative Evaluation mittels F-Score. Die Szenen folgen einer progressiven Komplexitätssteigerung, um verschiedene Aspekte der Rekonstruktionsverfahren isoliert zu testen. Abbildung~\ref{fig:virtual_scenes} zeigt eine Übersicht aller drei Szenen.

\paragraph{Szene V1 -- Geometrische Primitive}
Szene V1 dient als Baseline-Test und enthält ausschließlich einfache geometrische Primitive (Quader, Pyramide, Zylinder, Kapsel) in einem hexagonalen Raum mit farbigen Wänden. Die unifarbigen, matten Oberflächen ohne Texturen ermöglichen die isolierte Bewertung fundamentaler Rekonstruktionsfähigkeiten: scharfe Kanten, gekrümmte Oberflächen und feature-arme Flächen.

\paragraph{Szene V2 -- Möbliertes Schlafzimmer}
Szene V2 repräsentiert einen möblierten Innenraum mittlerer Komplexität mit Doppelbett, Sessel, Sideboard, Wandbildern und Stehlampe. Diese Szene testet die Rekonstruktion komplexer Möbelgeometrie, das Verhalten bei Okklusionen, die Texturverarbeitung sowie die Detailerfassung kleiner Dekorationsobjekte.

\paragraph{Szene V3 -- Komplexer Mehrzweckraum}
Szene V3 stellt einen Stresstest für Skalierbarkeit und Detailtreue dar und simuliert einen multifunktionalen Raum mit Schlaf-, Wohn- und Arbeitsbereich. Die hohe Objektdichte mit zwei Betten, Esstisch, Stühlen und diversen Kleinobjekten erzeugt multiple Okklusionsebenen. Erwartet werden längere Inferenzzeiten, höhere GPU-Auslastung und potenzielle Artefakte bei geometrisch komplexen Strukturen und teilweise verdeckten Bereichen.

\begin{figure}[H]
    \centering
    \includegraphics[width=0.32\textwidth]{images/room00.png}
    \includegraphics[width=0.32\textwidth]{images/room01.png}
    \includegraphics[width=0.32\textwidth]{images/Room02.png}
    \caption{Übersicht der drei virtuellen Testszenen: V1 (Geometrische Primitive), V2 (Möbliertes Schlafzimmer), V3 (Komplexer Mehrzweckraum)}
    \label{fig:virtual_scenes}
\end{figure}

\subsubsection{Reale Szenen}

Die beiden realen Szenen wurden im Va.Si.Li-Lab aufgenommen und validieren die Praxistauglichkeit des Systems unter realen Bedingungen mit natürlichen Störfaktoren.

\paragraph{Szene R1 -- Schlafzimmer}

Szene R1 simuliert ein kleines, dicht möbliertes, privates Umfeld. Details, Schrägen, Okklusionen. Der Testfokus liegt auf der Robustheit gegenüber Textiloberflächen und diffuser Beleuchtung, welche die Rekonstruktion feiner Details und das Verhalten bei Oberflächenhomogenität überprüfen.

\paragraph{Szene R2 -- Wohnzimmer}

Szene R2 simuliert ein großes Wohnzimmer mit offener Gestaltung. Große Flächen, Glas, Reflexionen. Die Szene dient als Skalierbarkeits- und Materialstresstest. Im Fokus stehen die Handhabung großer, glänzender Flächen und Fenster, die Reflexionen verursachen, sowie repetitive Dekorelemente, welche die globale Konsistenz und Anfälligkeit für visuellen Drift testen.

\begin{figure}[H]
    \centering
    \includegraphics[width=0.49\textwidth]{images/room03.jpg}
    \includegraphics[width=0.49\textwidth]{images/room04.jpg}
    \caption{Übersicht der beiden realen Testszenen: R1 (Schlafzimmer), R2 (Wohnzimmer)}
    \label{fig:real_scenes}
\end{figure}

\subsection{Messverfahren und Metriken}

\subsubsection{Performance-Metriken}

\paragraph{Latenz}
Die End-to-End-Latenz misst die Zeitspanne zwischen dem Versenden eines Fragments 
durch das Unity-Frontend und der Visualisierung der aktualisierten Rekonstruktion 
im VR-Headset. Sie setzt sich aus folgenden Komponenten zusammen:
\begin{align}
    L_{\text{total}} = L_{\text{network}} + L_{\text{inference}} + L_{\text{render}}
\end{align}
wobei $L_{\text{network}}$ die Netzwerklatenz (Upload des Fragments und Download 
der Rekonstruktion), $L_{\text{inference}}$ die Modell-Inferenzzeit im Backend 
(GPU-Verarbeitung) und $L_{\text{render}}$ die Rendering-Zeit im Unity-Client 
(GLB-Import und Mesh-Visualisierung) bezeichnet.


Die Netzwerklatenz $L_{\text{network}} = L_{\text{upload}} + L_{\text{download}}$ 
wird nicht direkt durch Zeitstempel gemessen, sondern aus den erfassten Datenvolumina 
und der verfügbaren Netzwerkbandbreite berechnet: $L_{\text{upload}} = S_{\text{fragment}} / B_{\text{upload}}$ 
bzw. $L_{\text{download}} = S_{\text{result}} / B_{\text{download}}$. Hierbei 
bezeichnet $S_{\text{fragment}}$ die Fragmentgröße (Upload-Volumen pro Fragment) 
und $S_{\text{result}}$ die Resultgröße (Download-Volumen der Rekonstruktion). 
Diese Methodik ermöglicht eine infrastrukturunabhängige Bewertung der Dateneffizienz.


Die Messung der übrigen Latenzkomponenten erfolgte durch präzise Zeitstempel an 
den jeweiligen Übergangspunkten der Pipeline.

\paragraph{Durchsatz}
Der Durchsatz quantifiziert, wie viele Fragmente pro Sekunde durch das System 
verarbeitet werden können. Ein höherer Durchsatz ermöglicht häufigere Updates der 
Rekonstruktion und trägt zur Immersion bei. Gemessen wurde der Durchsatz auf 
Backend-Seite für jedes Worker-Modell separat.

\paragraph{Ressourcenauslastung}
Die GPU- und CPU-Auslastung wurde kontinuierlich während der Rekonstruktion 
aufgezeichnet. GPU-Utilization und GPU-Memory wurden via \texttt{nvidia-smi} 
erfasst, CPU-Auslastung und RAM-Verbrauch pro Container via Docker Stats. Diese 
Metriken ermöglichen die Bewertung der Ressourceneffizienz und geben Aufschluss 
über Engpässe im System.

\subsubsection{Qualitätsmetriken}

\paragraph{Quantitative Bewertung}
Für Szenen mit verfügbarem Ground-Truth-Mesh wurde der F-Score als kombinierte 
Metrik für Präzision und Recall berechnet:

\begin{align}
    \text{Precision} &= \frac{|\text{TP}|}{|\text{TP}| + |\text{FP}|} \\
    \text{Recall} &= \frac{|\text{TP}|}{|\text{TP}| + |\text{FN}|} \\
    \text{F-Score} &= 2 \cdot \frac{\text{Precision} \cdot \text{Recall}}{\text{Precision} + \text{Recall}}
\end{align}

\noindent
Rekonstruierte Punkte gelten als \textbf{True Positive (TP)}, wenn ihr Abstand zum 
Ground-Truth unter \textit{10} cm liegt, andernfalls als \textbf{False Positive (FP)}. 
\textbf{False Negatives (FN)} sind Ground-Truth-Punkte ohne entsprechenden 
rekonstruierten Punkt innerhalb des Schwellenwerts.


\textbf{Precision} misst die Genauigkeit der Rekonstruktion, indem sie den Anteil 
korrekt rekonstruierter Punkte angibt. \textbf{Recall} bewertet die 
Vollständigkeit und gibt an, wie viele Ground-Truth-Punkte erfasst wurden. 
Der \textbf{F-Score} kombiniert beide Metriken als harmonisches Mittel und liefert einen 
ausgewogenen Gesamtwert. Je näher der F-Score bei \textit{1.0} liegt, 
desto höher ist die Qualität der Rekonstruktion.

\paragraph{Qualitative Bewertung}
Die rekonstruierten Meshes wurden anhand folgender Kriterien bewertet:

\begin{itemize}
    \item \textbf{Vollständigkeit}: Wie viel Prozent der Szene wurde erfasst?
    \item \textbf{Detailtreue}: Sind feine Strukturen erkennbar?
    \item \textbf{Artefaktfreiheit}: Treten Löcher, Flimmern oder Fehlgeometrie auf?
    \item \textbf{Oberflächenqualität}: Glattheit und Konsistenz der Rekonstruktion
\end{itemize}

\noindent
Die Bewertung erfolgte durch visuelle Inspektion der Rekonstruktionen in Unity 
sowie durch exportierte Screenshots.

\section{Ergebnisse}
\label{sec:ergebnisse}

\subsection{Funktionale Validierung}

Die funktionale Validierung bestätigt, dass RTReconstruct alle definierten Kernfunktionalitäten erfüllt.

\paragraph{End-to-End-Kommunikation}
Die vollständige Kommunikationskette von der Fragmenterfassung im Unity-Client über 
die WebSocket-Verbindung zum Router bis zur Verteilung an die Worker-Container und 
zurück funktioniert stabil. In \textit{15} Testläufen über eine Gesamtdauer von \textit{6} Stunden 
traten \textit{0} Verbindungsabbrüche auf.

\paragraph{Parallele Modellausführung}
Alle vier integrierten Rekonstruktionsmodelle (NeuralRecon, VisFusion, MASt3R-SLAM, 
SLAM3R) konnten gleichzeitig betrieben werden. Die containerisierte Architektur 
ermöglichte eine vollständige Isolierung, sodass unterschiedliche Python- und 
PyTorch-Versionen parallel lauffähig waren. Einzig limitierender Faktor war die einzelne GPU,
die durch die Modelle gemeinsam genutzt wurde.

\paragraph{Multi-Szenen-Unterstützung}
Das System unterstützt die gleichzeitige Verarbeitung mehrerer Szenen. In Tests 
mit \textit{2} parallelen Szenen und \textit{4} verbundenen Clients blieb die Funktionalität 
erhalten. Die szenenspezifische Zuordnung der Rekonstruktionsergebnisse erfolgte 
fehlerfrei.

\paragraph{Visualisierung in VR}
Die über das Backend empfangenen Meshes wurden erfolgreich im Unity-Client 
visualisiert. Das in Kapitel 5 beschriebene Spatial 
Hashing ermöglichte eine performante Darstellung auch bei größeren Meshes und Punktwolken mit bis zu 100.000 Punkten.

\subsection{Performance-Analyse}

\subsubsection{Latenz}

Die Latenz stellt die zentrale Performance-Metrik für die Echtzeitfähigkeit des Systems dar. Im Folgenden wird zunächst die Gesamtlatenz über alle Testszenen und Modelle präsentiert, anschließend in ihre Komponenten zerlegt und abschließend durch die Analyse der Datenvolumina kontextualisiert.

\paragraph{Gesamtlatenz}

Zur Evaluierung der Systemperformance wurde die End-to-End-Latenz \\ \(L_{total}\) als Zeitspanne zwischen dem Absenden eines Fragments vom Client und dem Empfang der zugehörigen Rekonstruktion gemessen. Für jede Kombination aus Testszene und Rekonstruktionsmodell wurden drei unabhängige Testläufe durchgeführt, bei denen identische Eingabedaten verwendet wurden. Um dabei Verzerrungen durch Ressourcenkonflikte zu vermeiden, wurden die Modelle sequenziell im Einzelbetrieb getestet. Abbildung~\ref{fig:latency_boxplots} visualisiert die resultierenden Latenzverteilungen als Boxplots.


Die Darstellung zeigt auf der x-Achse die vier evaluierten Rekonstruktionsmodelle (NeuralRecon, VisFusion, MASt3R-SLAM, SLAM3R), während die y-Achse die gemessene Latenz in Millisekunden angibt. Pro Modell sind drei Boxplots dargestellt, die jeweils die Latenzverteilung eines Testlaufs repräsentieren. Die farbliche Kodierung kennzeichnet dabei denselben Testlauf über alle Modelle hinweg.

\begin{figure}[H]
    \begin{subfigure}{0.5\textwidth}
        \includegraphics[width=\linewidth]{images/room00_latency.png}
        \caption{Szene V1 -- Geometrische Primitive}
    \end{subfigure}
    \begin{subfigure}{0.5\textwidth}
        \includegraphics[width=\linewidth]{images/room01_latency.png}
        \caption{Szene V2 -- Möbliertes Schlafzimmer}
    \end{subfigure}
    \begin{subfigure}{0.5\textwidth}
        \includegraphics[width=\linewidth]{images/room02_latency.png}
        \caption{Szene V3 -- Komplexer Mehrzweckraum}
    \end{subfigure}
    \begin{subfigure}{0.5\textwidth}
        \includegraphics[width=\linewidth]{images/room03_latency.png}
        \caption{Szene R1 -- Schlafzimmer}
    \end{subfigure}
    \begin{subfigure}{\textwidth}
        \centering
        \includegraphics[width=0.49\linewidth]{images/room04_latency.png}
        \caption{Szene R2 -- Wohnzimmer}
    \end{subfigure}
    \caption{Gesamtlatenz \(L_{total}\) für alle Testszenen und Modelle über drei Testläufe. Die Boxen zeigen den Interquartilbereich (25.\,--\,75.\,Perzentil), die horizontale Linie den Median und die Whiskers den Wertebereich ohne Ausreißer.}
    \label{fig:latency_boxplots}
\end{figure}

\newpage
\noindent
Die Boxplots zeigen für alle Modelle und Szenen geringe Interquartilbereiche und minimale Ausreißer, was auf eine stabile und reproduzierbare Latenzcharakteristik des Systems hinweist. Aufgrund dieser geringen Varianz zwischen den drei Testläufen werden in den folgenden Analysen zur Zusammensetzung der Gesamtlatenz sowie zu Datenvolumina die Messwerte der drei Testläufe aggregiert dargestellt. Dies ermöglicht eine kompaktere Präsentation ohne relevanten Informationsverlust.


Die Abbildung zeigt deutliche Unterschiede in der Gesamtlatenz zwischen den Testszenen: Szene V1 weist die niedrigsten Latenzwerte auf, während die Latenz in den komplexeren Szenen V3, R1 und R2 ansteigt. Zudem variiert die Latenz zwischen den Modellen, wobei MAST3R durchgängig die höchsten Werte erreicht. Um die Ursachen dieser Variation zu identifizieren, wird die Gesamtlatenz im Folgenden in ihre Komponenten zerlegt.


\paragraph{Zusammensetzung der Gesamtlatenz}

Die beobachteten Latenzunterschiede zwischen den Testszenen lassen sich durch die Zerlegung der Gesamtlatenz in ihre konstituierenden Komponenten \(L_{\text{network}}\), \(L_{\text{inference}}\) und \(L_{\text{render}}\) analysieren. Diese Aufschlüsselung ermöglicht es, szenenabhängige Effekte auf die Inferenzzeit von fixen Overhead-Kosten der Netzwerkkommunikation und Rendering-Pipeline zu separieren. Abbildung~\ref{fig:latency_stacked_bar} visualisiert die resultierende Zusammensetzung für alle Modelle und Testszenen.

\begin{figure}[H]
    \centering
    \includegraphics[width=\textwidth]{images/latency_split.png}
    \caption{Zusammensetzung der Gesamtlatenz nach Komponenten für alle Rekonstruktionsmodelle, aufgeschlüsselt nach Testszene und gemittelt über die drei Testläufe.}
    \label{fig:latency_stacked_bar}
\end{figure}


Die Aufschlüsselung zeigt, dass \(L_{\text{inference}}\) den dominierenden Anteil der Gesamtlatenz ausmacht und zwischen den Szenen stark variiert. Der Anteil von \(L_{\text{network}}\) und \(L_{\text{render}}\) bleibt über die Szenen hinweg relativ konstant, nimmt jedoch prozentual mit steigender Szenenkomplexität ab. Die Netzwerklatenz \(L_{\text{network}}\) wird dabei maßgeblich durch die Größe der übertragenen Daten bestimmt, deren Quantifizierung im Folgenden dargestellt wird.


\paragraph{Fragment- und Ergebnisgrößen}

Um die Netzwerklatenz \(L_{\text{network}}\) zu kontextualisieren und die Bandbreitenanforderungen des Systems zu dokumentieren, wurden die durchschnittlichen Fragmentgrößen (Upload) und Ergebnisgrößen (Download) für alle Modelle gemessen. Tabelle~\ref{tab:data_volumes} zeigt die resultierenden Datenvolumina sowie die daraus berechnete genutzte Bandbreite.

\begin{table}[h]
    \centering
    \caption{Durchschnittliche Datenvolumina und genutzte Bandbreite aufgeschlüsselt nach Szene und Modell (gemittelt über drei Testläufe).}
    \label{tab:data_volumes}
    \resizebox{\textwidth}{!}{%
    \begin{tabular}{l|l|c|c|c|c}
        \toprule
        \textbf{Szene} & \textbf{Modell} & \makecell{\textbf{\O~$\text{Frag}_{\text{in}}$~[MB]}} & \makecell{\textbf{\O~$\text{Frag}_{\text{out}}$~[MB]}} & \makecell{\textbf{$\sum\text{Frag}_{\text{in}}$~[MB]}} & \makecell{\textbf{$\sum\text{Frag}_{\text{out}}$~[MB]}} \\
        \midrule
        \multirow{4}{*}{V1} 
            & NeuralRecon  & \multirow{4}{*}{0.82} & 1.83 & \multirow{4}{*}{19.57} & 44.04 \\
            & VisFusion    &                       & 2.73 &                        & 65.40 \\
            & MASt3R-SLAM  &                       & 1.60 &                        & 38.42 \\
            & SLAM3R       &                       & 1.60 &                        & 38.42 \\
        \midrule
        \multirow{4}{*}{V2} 
            & NeuralRecon  & \multirow{4}{*}{1.62} & 2.84 & \multirow{4}{*}{69.57} & 122.28 \\
            & VisFusion    &                      & 3.25  &                        & 139.85 \\
            & MASt3R-SLAM  &                      & 1.60  &                        & 68.60 \\
            & SLAM3R       &                      & 1.60  &                        & 68.84 \\
        \midrule
        \multirow{4}{*}{V3} 
            & NeuralRecon  & \multirow{4}{*}{1.56} & 3.60 & \multirow{4}{*}{88.70} & 204.93 \\
            & VisFusion    &                      & 4.24  &                        & 241.51 \\
            & MASt3R-SLAM  &                      & 1.59  &                        & 90.69 \\
            & SLAM3R       &                      & 1.60  &                        & 91.26 \\
        \midrule
        \multirow{4}{*}{R1} 
            & NeuralRecon  & \multirow{4}{*}{1.63} & 2.19 & \multirow{4}{*}{78.50} & 105.29 \\
            & VisFusion    &                       & 2.58 &                       & 124.14 \\
            & MASt3R-SLAM  &                       & 1.60 &                       & 76.85 \\
            & SLAM3R       &                       & 1.60 &                       & 76.85 \\
        \midrule
        \multirow{4}{*}{R2} 
            & NeuralRecon  & \multirow{4}{*}{1.84} & 2.48 & \multirow{4}{*}{79.33} & 106.87 \\
            & VisFusion    &                       & 3.70 &                        & 159.19 \\
            & MASt3R-SLAM  &                       & 1.60 &                        & 68.84 \\
            & SLAM3R       &                       & 1.60 &                        & 68.84 \\
        \bottomrule
    \end{tabular}%
    }
\end{table}

Die Tabelle zeigt, dass die Fragmentgrößen zwischen den Modellen variieren, wobei SLAM3R aufgrund seiner größeren Fenstergrößen die umfangreichsten Fragmente benötigt. Die Ergebnisgrößen unterscheiden sich ebenfalls deutlich: Volumetrische Verfahren (NeuralRecon, VisFusion) erzeugen größere Meshes im GLB-Format, während punktbasierte Modelle (MASt3R-SLAM, SLAM3R) kompaktere Punktwolken zurückliefern. Die Summe der genutzten Bandbreite liegt bei X.X~Mbps für den Upload und Y.Y~Mbps für den Download.


Die präsentierten Latenzmessungen bilden zusammen mit den Datenvolumina die Grundlage für die Bewertung der Echtzeitfähigkeit und Skalierbarkeit des Systems in Abschnitt~\ref{sec:diskussion}.

\subsubsection{Durchsatz}

Der Durchsatz ergibt sich für diese Evaluation aus der Anzahl der verarbeiteten Fragmente pro Szene, geteilt durch die mittlere End-To-End Latenz \(L_{total}\) über alle 3 Testläufe. Diese Metrik gibt an, wie viele Fragmente pro Sekunde durch das System verarbeitet werden können und ist ein Indikator für die Aktualisierungsrate der Rekonstruktion im VR-Frontend. Die folgende Tabelle \ref{tab:throughput_results} fasst diese gemessenen Durchsatzwerte für alle Modelle und Szenen zusammen.

\begin{table}[h]
    \centering
    \label{tab:throughput_results}
    \begin{tabular}{lccccc}
        \toprule
        \textbf{Modell} & \textbf{V1} & \textbf{V2} & \textbf{V3} & \textbf{R1} & \textbf{R2}\\
        \midrule
        NeuralRecon     & 0.62 & 0.45 & 0.32 &  0.48 & 0.45 \\
        VisFusion       & 0.38 & 0.35 & 0.30 &  0.38 & 0.31 \\
        MASt3R-SLAM     & 0.13 & 0.12 & 0.12 &  0.12 & 0.12 \\
        SLAM3R          & 0.07 & 0.06 & 0.03 &  0.10 & 0.09 \\
        \bottomrule
    \end{tabular}
    \bigskip
    \caption{Durchsatz nach Modell (Fragmente pro Sekunde)}
\end{table}

\subsubsection{Ressourcenauslastung}

Die Ressourcenauslastung wurde sowohl für das Backend (Server-seitige Rekonstruktion) 
als auch für das Frontend (VR-Client-seitige Visualisierung) getrennt erfasst. Diese 
Trennung ermöglicht die Identifikation von Engpässen in der Pipeline und gibt 
Aufschluss darüber, welche Systemkomponente limitierend wirkt.

\paragraph{Backend-Ressourcen}

Die Backend-Ressourcenauslastung wurde kontinuierlich während der Rekonstruktionsläufe 
auf dem dedizierten Server (AMD Ryzen 9 5900X, NVIDIA GTX 1070 Ti) erfasst. Die 
Messungen umfassen GPU-Auslastung und GPU-Speicherverbrauch (erfasst mittels 
\texttt{nvidia-smi} in 1-Sekunden-Intervallen) sowie CPU- und RAM-Nutzung der 
containerisierten Komponenten (erfasst mittels \texttt{docker stats}).


\textbf{GPU-Ressourcen:} Tabelle~\ref{tab:gpu_resources} zeigt die durchschnittliche 
GPU-Utilization und den maximalen GPU-Speicherverbrauch während der Rekonstruktion, 
aufgeschlüsselt nach Modell und Testszene.

\begin{table}[h]
    \centering
    \caption{GPU-Auslastung und GPU-Speicherverbrauch während der Rekonstruktion}
    \label{tab:gpu_resources}
    \begin{tabular}{lcccccc}
        \toprule
        \textbf{Modell} & \textbf{Metrik} & \textbf{V1} & \textbf{V2} & \textbf{V3} & \textbf{R1} & \textbf{R2}\\
        \midrule
        \multirow{2}{*}{NeuralRecon} 
            & \O GPU-Utilization [\%] & \textit{44} & \textit{43} & \textit{47} & \textit{47} & \textit{48} \\
            & Max. VRAM [MB]          & \textit{3825} & \textit{4770} & \textit{7993} & \textit{6368} & \textit{5986} \\
        \midrule
        \multirow{2}{*}{VisFusion} 
            & \O GPU-Utilization [\%] & \textit{49} & \textit{53} & \textit{55} & \textit{56} & \textit{52} \\
            & Max. VRAM [MB]          & \textit{4877} & \textit{4599} & \textit{4418} & \textit{4406} & \textit{5984} \\
        \midrule
        \multirow{2}{*}{MASt3R-SLAM} 
            & \O GPU-Utilization [\%] & \textit{100} & \textit{100} & \textit{100} & \textit{100} & \textit{100} \\
            & Max. VRAM [MB]          & \textit{7969} & \textit{8016} & \textit{7984} & \textit{8014} & \textit{7966} \\
        \midrule
        \multirow{2}{*}{SLAM3R} 
            & \O GPU-Utilization [\%] & \textit{98.5} & \textit{99} & \textit{99} & \textit{98} & \textit{99} \\
            & Max. VRAM [MB]          & \textit{8022} & \textit{7986} & \textit{8012} & \textit{7637} & \textit{7921} \\
        \bottomrule
    \end{tabular}
\end{table}


\textbf{CPU- und RAM-Auslastung:} Die Ressourcennutzung der einzelnen Docker-Container 
ist in Tabelle~\ref{tab:container_resources} dargestellt. Die Werte zeigen die 
durchschnittliche CPU-Last und den RAM-Verbrauch für Router- und Worker-Container, 
gemittelt über alle Testszenen.

\begin{table}[h]
    \centering
    \caption{Durchschnittliche CPU- und RAM-Auslastung der Backend-Container}
    \label{tab:container_resources}
    \begin{tabular}{lcc}
        \toprule
        \textbf{Container} & \textbf{\O CPU-Auslastung [\%]} & \textbf{\O RAM-Verbrauch [MB]} \\
        \midrule
        Router                    & \textit{0.2} & \textit{200} \\
        \midrule
        Worker: NeuralRecon       & \textit{32} & \textit{2780} \\
        Worker: VisFusion         & \textit{17.25} & \textit{3015} \\
        Worker: MASt3R-SLAM       & \textit{14.35} & \textit{3466} \\
        Worker: SLAM3R            & \textit{15.37} & \textit{4070} \\
        \bottomrule
    \end{tabular}
\end{table}

\paragraph{Frontend-Ressourcen}

Die durchschnittliche Frame Rate des Unity-Clients wurde sowohl im Baseline-Betrieb 
(ohne aktive Rekonstruktion) als auch während der Rekonstruktions- und 
Visualisierungsphase gemessen. Tabelle~\ref{frame_rate} zeigt die Ergebnisse.

\begin{figure}[H]
    \centering
    \label{frame_rate}
    % \caption{FPS des Unity-Clients im Baseline-Betrieb und während der Rekonstruktion}
    \includegraphics[width=0.65\textwidth]{images/frontend_fps.png}
\end{figure}

\subsection{Rekonstruktionsqualität}
Die Rekonstruktionsqualität wird zunächst anhand der synthetischen Szenen V1--V3 mit 
Ground-Truth-Daten quantitativ und qualitativ bewertet. Anschließend erfolgt eine 
Untersuchung der Praxistauglichkeit unter realen VR-Bedingungen anhand der mit der 
Meta Quest 3 aufgenommenen Szenen R1 und R2.    

\subsubsection{Quantitative Bewertung}

Tabelle~\ref{tab:fscore_all} zeigt die F-Score-Ergebnisse für alle Szenen mit verfügbarem Ground-Truth für einen Schwellenwert von \textit{10}cm.

\begin{table}[H]
    \centering
    \label{tab:fscore_all}
    \caption{F-Score-Ergebnisse nach Modell und Szene}
    \begin{tabular}{lccc|ccc|ccc}
        \toprule
        & \multicolumn{3}{c}{\textbf{V1}} & \multicolumn{3}{c}{\textbf{V2}} & \multicolumn{3}{c}{\textbf{V3}} \\
        \cmidrule(lr){2-4} \cmidrule(lr){5-7} \cmidrule(lr){8-10}
        \textbf{Modell} & \textbf{Prec.} & \textbf{Rec.} & \textbf{F-Score} & \textbf{Prec.} & \textbf{Rec.} & \textbf{F-Score} & \textbf{Prec.} & \textbf{Rec.} & \textbf{F-Score} \\
        \midrule
        NeuralRecon     & \textit{0.45} & \textit{0.39} & \textit{0.41} & \textit{0.69} & \textit{0.67} & \textit{0.68} & \textit{0.59} & \textit{0.59} & \textit{0.59} \\
        VisFusion       & \textit{0.57} & \textit{0.52} & \textit{0.54} & \textit{0.57} & \textit{0.65} & \textit{0.61} & \textit{0.57} & \textit{0.68} & \textit{0.62} \\
        SLAM3R          & \textit{0.66} & \textit{0.56} & \textit{0.61} & \textit{0.67} & \textit{0.59} & \textit{0.63} & \textit{0.41} & \textit{0.43} & \textit{0.42} \\
        MASt3R-SLAM     & \textit{0.52} & \textit{0.48} & \textit{0.50} & \textit{0.47} & \textit{0.53} & \textit{0.50} & \textit{0.43} & \textit{0.48} & \textit{0.45} \\
        \bottomrule
    \end{tabular}
\end{table}

\noindent
Die Berechnung erfolgte in CloudCompare mittels Cloud-to-Cloud Distance: Punkte mit 
Abstand \(\leq 10\)\,cm zum nächsten Ground-Truth-Punkt wurden als True Positives 
klassifiziert, größere Abstände als False Positives. False Negatives wurden durch 
umgekehrte Distanzberechnung ermittelt. Da SLAM3R und MASt3R-SLAM keine globale 
Konsistenz gewährleisten, wurden deren Rekonstruktionen zuvor mittels ICP ausgerichtet.

\subsubsection{Qualitative Bewertung}

\paragraph{NeuralRecon}

Die Rekonstruktionen von NeuralRecon zeigten deutliche Lücken über alle Testszenarien hinweg. Besonders in Szene V1 wurden schräge Flächen vollständig ausgelassen, während horizontale und vertikale Strukturen zumindest teilweise erfasst wurden. In den komplexeren Szenen V2 und V3 setzte sich diese Lückenhaftigkeit fort. Neben den fehlenden Bereichen fielen auch vereinfachte Geometrien auf - feine Details gingen verloren, während gröbere Strukturen noch erkennbar blieben. Positiv hervorzuheben ist hingegen die Qualität der tatsächlich erfassten Bereiche: Die Meshes wiesen glatte, geschlossene Oberflächen auf und waren frei von Fehlgeometrien oder Flimmerartefakten. Die globale Koherenz blieb durchgängig erhalten.

\begin{figure}[H]
    \centering
    \begin{subfigure}{0.49\linewidth}
        \includegraphics[width=\linewidth]{images/EvaluationScreenshots/room00/gt00.png}
        \caption{Ground Truth}
    \end{subfigure}
    \begin{subfigure}{0.49\linewidth}
        \includegraphics[width=\linewidth]{images/EvaluationScreenshots/room00/neucon00.png}
        \caption{NeuralRecon Rekonstruktion}
    \end{subfigure}
    \caption{Lückenhafte Rekonstruktion von Szene V1 durch NeuralRecon. Schräge Flächen werden nicht erfasst, während horizontale und vertikale Strukturen teilweise rekonstruiert werden.}
    \label{fig:neuralrecon_v1}
\end{figure}

\begin{figure}[H]
    \centering
    \begin{subfigure}{0.49\linewidth}
        \includegraphics[width=\linewidth]{images/EvaluationScreenshots/room01/gt00.png}
        \caption{Ground Truth Detail}
    \end{subfigure}
    \begin{subfigure}{0.49\linewidth}
        \includegraphics[width=\linewidth]{images/EvaluationScreenshots/room01/neucon00.png}
        \caption{NeuralRecon Detail}
    \end{subfigure}
    \caption{Detailverlust bei NeuralRecon. Objekte wie die Stehlampe oder der TV-Schrank in Szene V2 gehen in der Rekonstruktion größtenteils verloren.}
    \label{fig:neuralrecon_detail}
\end{figure}

\paragraph{VisFusion}

VisFusion überzeugte mit einer spürbar höheren Vollständigkeit als NeuralRecon. Schräge Flächen in Szene V1, die zuvor komplett fehlten, wurden nun erfolgreich rekonstruiert, und insgesamt blieben weniger Bereiche lückenhaft. Die F-Scores stiegen über die Szenen hinweg kontinuierlich an (V1: 0.54, V2: 0.61, V3: 0.62), was auf eine robuste Performance auch bei zunehmender Komplexität hindeutet. Auch im Detail zeigte sich eine Verbesserung: Feinere Strukturen blieben besser erhalten, während gröbere Geometrien ähnlich zuverlässig erfasst wurden wie bei NeuralRecon. Die Meshes waren durchweg glatt, artefaktfrei und wiesen eine höhere geometrische Präzision in komplexen Bereichen auf. Die globale Konsistenz blieb über alle Szenen hinweg stabil.

\begin{figure}[H]
    \centering
    \begin{subfigure}{0.49\linewidth}
        \includegraphics[width=\linewidth]{images/EvaluationScreenshots/room00/neucon00.png}
        \caption{Neural Recon}
    \end{subfigure}
    \begin{subfigure}{0.49\linewidth}
        \includegraphics[width=\linewidth]{images/EvaluationScreenshots/room00/visfusion00.png}
        \caption{VisFusion}
    \end{subfigure}
    \caption{Vergleich zwischen NeuralRecon (links) und VisFusion (rechts) in Szene V1. VisFusion rekonstruiert die schrägen Flächen erfolgreich und zeigt eine deutlich höhere Vollständigkeit.}
    \label{fig:visfusion_comparison}
\end{figure}

\begin{figure}[H]
    \centering
    \begin{subfigure}{0.49\linewidth}
        \includegraphics[width=\linewidth]{images/EvaluationScreenshots/room01/visfusion00.png}
        \caption{Szene V2}
    \end{subfigure}
    \begin{subfigure}{0.49\linewidth}
        \includegraphics[width=\linewidth]{images/EvaluationScreenshots/room02/visfusion.png}
        \caption{Szene V3}
    \end{subfigure}
    \caption{VisFusion Rekonstruktionen der Szenen V2 (links) und V3 (rechts). Die Performance bleibt auch bei zunehmender Komplexität stabil. Objekte wie die Stehlampe oder der TV-Schrank werden detaillierter erfasst.}
    \label{fig:visfusion_scenes}
\end{figure}

\paragraph{SLAM3R}

SLAM3R verfolgte einen grundlegend anderen Ansatz und erzeugte dichte, farbige Punktwolken anstelle von Meshes. Bereits in der einfachen Szene V1 (F-Score: 0.66) zeigten sich deutliche Drift-Probleme, die zu Versätzen zwischen Rekonstruktionsabschnitten führten. Im Vergleich zu MASt3R-SLAM waren diese Probleme jedoch noch moderater ausgeprägt. Lokale Details blieben gut erhalten, und die realistische Farbdarstellung trug zur visuellen Qualität bei. In Szene V2 (F-Score: 0.67) verstärkten sich die Konsistenzprobleme bereits merklich. Mit zunehmender Szenenkomplexität verschlechterte sich die Performance dramatisch: In der komplexen Szene V3 brach sie vollständig ein (F-Score: 0.42). Hier akkumulierten sich Ungenauigkeiten massiv, das Verfahren verlor häufig den Anschluss zwischen aufeinanderfolgenden Rekonstruktionsabschnitten, was zu ausgeprägten Versätzen und Brüchen im Gesamtmodell führte. Die fehlende globale Ausrichtung machte eine manuelle Registrierung der Rekonstruktionen notwendig, um sie mit dem Ground Truth abzugleichen. Große Bereiche blieben fragmentiert und inkonsistent, die globale Kohärenz ging vollständig verloren.

\begin{figure}[H]
    \centering
    \begin{subfigure}{0.49\linewidth}
        \includegraphics[width=\linewidth]{images/EvaluationScreenshots/room00/gt00.png}
        \caption{SLAM3R Rekonstruktion}
    \end{subfigure}
    \begin{subfigure}{0.49\linewidth}
        \includegraphics[width=\linewidth]{images/EvaluationScreenshots/room00/slam3r00.png}
        \caption{SLAM3R Rekonstruktion}
    \end{subfigure}
    \caption{SLAM3R Rekonstruktion von Szene V1 mit erkennbaren Drift-Problemen. Trotz Versätzen bleiben lokale Details und Farbinformation gut erhalten.}
    \label{fig:slam3r_v1}
\end{figure}

\begin{figure}[H]
    \centering
    \begin{subfigure}{0.32\linewidth}
        \includegraphics[width=\linewidth]{images/EvaluationScreenshots/room00/slam3r00.png}
        \caption{Szene V1}
    \end{subfigure}
    \begin{subfigure}{0.32\linewidth}
        \includegraphics[width=\linewidth]{images/EvaluationScreenshots/room01/slam3r00.png}
        \caption{Szene V2}
    \end{subfigure}
    \begin{subfigure}{0.32\linewidth}
        \includegraphics[width=\linewidth]{images/EvaluationScreenshots/room02/slam3r.png}
        \caption{Szene V3}
    \end{subfigure}
    \caption{SLAM3R Performance-Vergleich über Szenen: V1 (links), V2 (Mitte), V3 (rechts). Deutlich erkennbar ist die zunehmende Fragmentierung und der Verlust globaler Kohärenz mit steigender Komplexität.}
    \label{fig:slam3r_scenes_progression}
\end{figure}


\paragraph{MASt3R-SLAM}

MASt3R-SLAM erzeugte ebenfalls Punktwolken und zeigte über alle Szenen hinweg konstante F-Scores (V1: 0.50, V2: 0.50, V3: 0.45). In der einfachen Szene V1 war das Verfahren jedoch SLAM3R deutlich unterlegen: Der Drift war stärker ausgeprägt, und die globale Kohärenz litt bereits in dieser unkomplizierten Umgebung. Auch Fragmente traten bereits in V1 häufiger auf als bei SLAM3R und beeinträchtigten die visuelle Qualität der Rekonstruktion. Mit zunehmender Szenenkomplexität zeigte sich jedoch ein anderes Bild: Während SLAM3R in V2 und besonders in V3 stark einbrach, blieb MASt3R-SLAM relativ stabil. In den komplexeren Szenen V2 und V3 übertraf es SLAM3R sowohl in der räumlichen Konsistenz als auch in der globalen Ausrichtung. Die Fragmentierung durch frei schwebende Punkte war zwar weiterhin vorhanden und trug zur Unübersichtlichkeit bei, jedoch deutlich moderater als bei SLAM3R in diesen Szenen. Rekonstruktionsabschnitte fügten sich besser zu einem Gesamtbild zusammen, auch wenn eine manuelle Registrierung weiterhin notwendig blieb, um die Rekonstruktionen mit dem Ground Truth abzugleichen. Die lokale Detailtreue mit Farbinformation war durchgängig gegeben, und die Performance blieb über die Szenen hinweg bemerkenswert konsistent, während SLAM3R zunehmend instabil wurde.

\begin{figure}[H]
    \centering
    \begin{subfigure}{0.49\linewidth}
        \includegraphics[width=\linewidth]{images/EvaluationScreenshots/room00/gt00.png}
        \caption{Ground Truth}
    \end{subfigure}
    \begin{subfigure}{0.49\linewidth}
        \includegraphics[width=\linewidth]{images/EvaluationScreenshots/room00/mast3r0.png}
        \caption{MASt3R-SLAM Rekonstruktion}
    \end{subfigure}
    \caption{MASt3R-SLAM Rekonstruktion von Szene V1 mit ausgeprägten Drift-Problemen. Die Versätze und frei schwebenden Punkte sind hier deutlich stärker als bei SLAM3R in derselben Szene.}
    \label{fig:mast3r_v1}
\end{figure}

\begin{figure}[H]
    \centering
    \begin{subfigure}{0.32\linewidth}
        \includegraphics[width=\linewidth]{images/EvaluationScreenshots/room00/mast3r0.png}
        \caption{Szene V1}
    \end{subfigure}
    \begin{subfigure}{0.32\linewidth}
        \includegraphics[width=\linewidth]{images/EvaluationScreenshots/room01/mast3r00.png}
        \caption{Szene V2}
    \end{subfigure}
    \begin{subfigure}{0.32\linewidth}
        \includegraphics[width=\linewidth]{images/EvaluationScreenshots/room02/mast3r.png}
        \caption{Szene V3}
    \end{subfigure}
    \caption{MASt3R-SLAM Performance-Vergleich über Szenen: V1 (links), V2 (Mitte), V3 (rechts). Im Gegensatz zu SLAM3R bleibt die Konsistenz über die Szenen hinweg relativ stabil.}
    \label{fig:mast3r_scenes_progression}
\end{figure}


\paragraph{Vergleichende Bewertung}
Tabelle \ref{tab:qualitative_comparison} fasst die qualitativen Beobachtungen zusammen und bewertet die Verfahren anhand einer fünfstufigen Skala (sehr gering, gering, mittel, hoch, sehr hoch) in Bezug auf die definierten Qualitätskriterien. Diese Bewertung erfolgte durch vergleichende visuelle Inspektion der Rekonstruktionen und ermöglicht eine differenzierte Einordnung der Stärken und Schwächen der einzelnen Verfahren.

\begin{table}[H]
    \centering
    \label{tab:qualitative_comparison}
    \begin{tabular}{lcccc}
        \toprule
        \textbf{Kriterium} & \textbf{NeuralRecon} & \textbf{VisFusion} & \textbf{SLAM3R} & \textbf{MASt3R-SLAM} \\
        \midrule
        Vollständigkeit & mittel & hoch & hoch & hoch \\
        Detailtreue & gering & mittel & sehr hoch & sehr hoch \\
        Artefaktfreiheit & sehr hoch & sehr hoch & sehr gering & gering \\
        Oberflächenqualität & sehr hoch & sehr hoch & mittel & mittel \\
        \bottomrule
    \end{tabular}
    \medskip
    \caption{Qualitative Bewertung der Rekonstruktionsverfahren im Vergleich}
\end{table}

\subsubsection{Rekonstruktion unter realen VR-Bedingungen}

Während die virtuellen Szenen V1--V3 die Rekonstruktionsqualität unter kontrollierten 
Bedingungen mit synthetischen, hochauflösenden Kameras evaluieren, dienen die realen 
Szenen R1 (Schlafzimmer) und R2 (Wohnzimmer) der Praxisvalidierung des Systems unter 
authentischen VR-Bedingungen mit der Meta Quest 3.

\paragraph{Charakteristika der Quest 3-Passthrough-Kamera}
Die für Tracking optimierten Pass\-through-Kameras der Quest 3 unterscheiden sich in 
mehreren Hardware-bedingten Eigenschaften von synthetischen Kameras. Sie weisen einen 
geringen Dynamikumfang auf, zeigen Rolling-Shutter-Effekte bei schnellen Kopfbewegungen 
und produzieren Bildrauschen bei schwacher Beleuchtung. Zudem ist der Kontrast reduziert.

\paragraph{Beobachtungen NeuralRecon}
NeuralRecon wies in den realen Szenen R1 und R2 die geringste Rekonstruktionsqualität 
aller evaluierten Modelle auf. Die erzeugten Rekonstruktionen zeigten erhebliche 
Diskontinuitäten in der Oberflächenrepräsentation, wobei feinere geometrische Details 
größtenteils nicht erfasst wurden. Die Rekonstruktionen beschränkten sich im Wesentlichen 
auf grobe Approximationen der Szenengeometrie.

\paragraph{Beobachtungen VisFusion}
VisFusion erzielte eine marginal verbesserte Rekonstruktionsqualität gegenüber NeuralRecon. 
Die Oberflächenrepräsentation wies weiterhin partielle Diskontinuitäten auf, wobei eine 
Erfassung gröberer geometrischer Strukturen erkennbar war. Die Vollständigkeit der 
Rekonstruktion übertraf NeuralRecon geringfügig, beschränkte sich jedoch ebenfalls 
primär auf die Erfassung makroskopischer Formmerkmale.

\begin{figure}[H]
    \centering
    \includegraphics[width=0.49\linewidth]{images/room03_neucon00.png}
    \includegraphics[width=0.49\linewidth]{images/room03_visfusion00.png}
    \caption{Rekonstruktionen der volumetrischen Verfahren in Szene R1. Links: NeuralRecon 
    mit erheblichen Diskontinuitäten. Rechts: VisFusion mit marginal verbesserter Vollständigkeit.}
    \label{fig:real_volumetric}
\end{figure}

\paragraph{Beobachtungen SLAM3R}
SLAM3R demonstrierte in den realen Szenen eine signifikant höhere Rekonstruktionsqualität 
als die volumetrischen Verfahren. Die Punktwolken-Repräsentation ermöglichte eine detaillierte 
Erfassung der Szenengeometrie mit deutlich reduzierter Diskontinuität gegenüber NeuralRecon 
und VisFusion. Bemerkenswert ist die verbesserte Performance gegenüber den synthetischen 
Szenen, die sich in reduzierter Artefaktbildung und beschleunigter Konvergenz manifestierte.

\paragraph{Beobachtungen MASt3R-SLAM}
MASt3R-SLAM erzielte die höchste Rekonstruktionsqualität in den realen Szenen. Die 
Rekonstruktionsergebnisse waren qualitativ vergleichbar mit SLAM3R, wiesen jedoch einen 
marginal erhöhten Detailgrad sowie reduzierte Artefaktbildung auf. Die Oberflächenrepräsentation 
zeigte sehr geringe Diskontinuität. Analog zu SLAM3R übertraf die Performance in den realen 
Szenen die Ergebnisse der synthetischen Evaluationen, wobei MASt3R-SLAM insgesamt die 
höchste Rekonstruktionsgüte in R1 und R2 erreichte.

\begin{figure}[H]
    \centering
    \includegraphics[width=0.49\linewidth]{images/room04_slam3r00.png}
    \includegraphics[width=0.49\linewidth]{images/room04_mast3r00.png}
    \caption{Rekonstruktionen der SLAM-basierten Verfahren in Szene R2. Links: SLAM3R 
    mit hohem Detailgrad. Rechts: MASt3R-SLAM mit höchster Rekonstruktionsgüte und reduzierter 
    Artefaktbildung.}
    \label{fig:real_slam}
\end{figure}

\paragraph{Vergleich synthetische vs. reale Szenen}
Der Vergleich zwischen den synthetischen Szenen V1--V3 und den realen Szenen R1--R2 
offenbart distinkte Performance-Charakteristika der unterschiedlichen Modellklassen. 
Die volumetrischen Verfahren NeuralRecon und VisFusion, welche in den synthetischen 
Szenen robuste und vollständige Rekonstruktionen generierten, zeigten in den realen 
Szenen eine signifikant reduzierte Rekonstruktionsqualität mit ausgeprägten 
Diskontinuitäten und verlorenem geometrischem Detailgrad. Konträr dazu wiesen die 
SLAM-basierten Verfahren SLAM3R und MASt3R-SLAM eine Performance-Inversion auf: 
Während diese in den synthetischen Szenen partiell mit Artefaktbildung und prolongierter 
Konvergenz konfrontiert waren, erzielten sie in den realen Szenen superiore 
Rekonstruktionsergebnisse mit erhöhtem Detailgrad und minimaler Diskontinuität. 
MASt3R-SLAM erreichte dabei die höchste Rekonstruktionsgüte in den realen VR-Aufnahmen.

\subsection{Modularität und Systemstabilität}

Die Modularität der entwickelten Architektur wird durch die erfolgreiche Integration 
der vier heterogenen Rekonstruktionsverfahren NeuralRecon, VisFusion, MASt3R-SLAM 
und SLAM3R unter Beweis gestellt. Dieser Abschnitt quantifiziert den Integrationsaufwand 
pro Modell und bewertet die Stabilität des Systems unter realen Betriebsbedingungen.

\subsubsection*{Integrationsaufwand}

Die Integration eines neuen Rekonstruktionsmodells in RTReconstruct erfordert drei 
zentrale Implementierungsschritte: die Erstellung eines modellspezifischen Docker-Containers 
mit allen benötigten Abhängigkeiten, die Implementierung einer Worker-Klasse als 
Unterklasse von \texttt{BaseReconstructionModel} sowie die Registrierung des neuen 
Dienstes in der \texttt{docker-compose.yml}. Tabelle~\ref{tab:integration_effort} 
fasst den durchschnittlichen Aufwand pro Modell zusammen.

\begin{table}[H]
\centering
\caption{Integrationsaufwand pro Rekonstruktionsmodell}
\label{tab:integration_effort}
\begin{tabular}{lcccc}
\toprule
\textbf{Komponente} & \textbf{NeuralRecon} & \textbf{VisFusion} & \textbf{MASt3R} & \textbf{SLAM3R} \\
\midrule
Dockerfile (Zeilen)           & 42  & 38  & 51  & 47  \\
Worker-Klasse (Zeilen)        & 156 & 148 & 189 & 203 \\
docker-compose.yml (Zeilen)   & 12  & 12  & 14  & 13  \\
\midrule
\textbf{Gesamt (Zeilen)}      & 210 & 198 & 254 & 263 \\
Geschätzter Zeitaufwand (h)   & 4--6 & 3--5 & 6--8 & 7--9 \\
\bottomrule
\end{tabular}
\end{table}

\noindent
Der Großteil des Implementierungsaufwands entfällt auf zwei Bereiche: Die Erstellung 
des Docker-Containers erfordert die korrekte Konfiguration von Base-Image, CUDA-Versionen, 
Python-Abhängigkeiten und modellspezifischen Bibliotheken. Besonders anspruchsvoll ist 
dabei die Kompatibilität zwischen PyTorch-, CUDA- und Treiber-Versionen, die für jedes 
Modell individuell aufgelöst werden muss. Die Implementierung der Worker-Klasse umfasst 
neben der Datenaufbereitung auch die Initialisierung des jeweiligen Rekonstruktionsmodells, 
das Laden von Modellgewichten sowie die Transformation der Ausgabe in das standardisierte 
\texttt{ModelResult}-Format.

Die Einbindung in die \texttt{docker-compose.yml} erfordert die Definition eines neuen 
Service-Eintrags, der den Build-Kontext, GPU-Ressourcen, Umgebungsvariablen (Modellname, 
Server-URL) sowie die Netzwerkanbindung an den zentralen Router spezifiziert. Dieser Schritt 
ist weitgehend standardisiert und kann durch Anpassung eines bestehenden Service-Templates 
erfolgen. Die Kommunikationslogik - Verbindungsaufbau zum Router, asynchroner Fragmentempfang 
und Ergebnisrücksendung - wird vollständig von der Basisklasse \texttt{BaseReconstructionModel} 
bereitgestellt und erfordert keine modellspezifische Anpassung. Dies reduziert die 
Komplexität der Integration erheblich.

\subsubsection*{Systemstabilität}

Die Stabilität des Systems wurde hinsichtlich der Isolierung von Fehlern einzelner 
Komponenten untersucht. Dabei wurden drei Aspekte betrachtet.

\paragraph{Fehlertoleranz bei fehlerhaften Rekonstruktionen}
Fehler innerhalb der Worker-Prozesse - etwa durch ungültige Eingabedaten, 
Speicherengpässe oder Modell-Inferenzfehler - führten nicht zu einem Absturz 
des zentralen Routers oder anderer Worker. Die fehlerhafte Rekonstruktionsanfrage 
wurde verworfen, während das System für andere Szenen und Modelle funktionsfähig blieb. 
Dies bestätigt die durch Containerisierung erreichte Isolation zwischen den Komponenten.

\paragraph{Verhalten bei Worker-Absturz}
Stürzt ein Worker-Container ab, wird dieser automatisch aus der Liste der verfügbaren 
Modelle entfernt. Der Router verarbeitet weiterhin Anfragen für die verbleibenden 
Modelle, und andere Clients bleiben unbeeinträchtigt. Ein manueller Neustart des 
abgestürzten Containers führt zur automatischen Wiederanmeldung beim Router, sodass 
das Modell anschließend erneut verfügbar ist.

\paragraph{Verbindungsabbruch zwischen Frontend und Backend}
Bei einem Verbindungsabbruch der WebSocket-Verbindung zwischen Frontend und Router 
ist ein manueller Neustart der Frontend-Anwendung erforderlich. Ein automatischer 
Reconnect-Mechanismus wurde nicht implementiert. Dies stellt eine Limitation der 
aktuellen Implementierung dar, die insbesondere bei mobilen VR-Anwendungen mit 
instabiler Netzwerkverbindung relevant werden kann.

\section{Diskussion}

Die vorliegende Arbeit untersuchte die Eignung einer modularen, containerisierten Systemarchitektur für die Integration verschiedener Echtzeit-3D-Rekonstruktionsverfahren in eine bestehende Virtual-Reality-Umgebung. Die in Kapitel~6.3 präsentierten Evaluationsergebnisse werden im Folgenden systematisch interpretiert und hinsichtlich ihrer praktischen Implikationen diskutiert. Die Diskussion gliedert sich in fünf Schwerpunkte: Echtzeitfähigkeit und Performance-Charakteristik (6.4.1), Modularität und Erweiterbarkeit der Architektur (6.4.2), Rekonstruktionsqualität und Generalisierbarkeit (6.4.3), Eignung für VR-Anwendungen (6.4.4) sowie die abschließende Beantwortung der Forschungsfrage (6.4.5).

\subsection{Echtzeitfähigkeit und Performance-Charakteristik}

Die Echtzeitfähigkeit des Systems wird maßgeblich durch die architektonische Entkopplung von Datenerfassung, Inferenz und Visualisierung bestimmt. Die Performance-Analyse zeigt, dass das entwickelte System diese Entkopplung erfolgreich realisiert, wobei sich jedoch deutliche Unterschiede zwischen den integrierten Rekonstruktionsverfahren manifestieren, die primär auf deren jeweilige algorithmische Komplexität zurückzuführen sind.

Die volumetrischen Verfahren NeuralRecon und VisFusion erreichten Durchsatzraten von 0,32--0,62 bzw. 0,30--0,38 Fragmenten pro Sekunde (vgl.~Tabelle~6.4), was für kontinuierliche Szenenaktualisierungen in VR-Anwendungen als praktikabel einzustufen ist. Im Kontrast dazu wiesen die SLAM-basierten Verfahren mit 0,03--0,13 Fragmenten pro Sekunde deutlich geringere Durchsatzraten auf. Diese Diskrepanz lässt sich durch die unterschiedlichen Rekonstruktionsparadigmen erklären: Während volumetrische Ansätze auf effiziente TSDF-Fusion in vordefinierten Voxelgittern optimiert sind, erfordern SLAM-basierte Verfahren rechenintensive Feature-Matching- und Bundle-Adjustment-Operationen über wachsende Keyframe-Mengen.

Bemerkenswert ist die hohe Stabilität der gemessenen Latenzen über alle Testläufe hinweg. Die Boxplot-Darstellungen (vgl.~Abbildung~6.3a--e) offenbaren minimale Interquartilbereiche mit nahezu identischen Medianwerten über drei unabhängige Messungen. Lediglich vereinzelte Ausreißer traten auf, die sich auf initiale Container-Warmlaufphasen oder Garbage-Collection-Events zurückführen lassen. Diese Konsistenz belegt, dass die asynchrone WebSocket-Architektur mit ihrem Fan-Out-Mechanismus auch bei heterogenen Inferenzzeiten eine robuste Datenstromverarbeitung gewährleistet. Die Entkopplung verhindert, dass langsamere Modelle die Datenerfassung im Frontend blockieren oder schnellere Modelle auf Eingabedaten warten müssen.

Die Zerlegung der End-to-End-Latenz in ihre konstituierenden Komponenten verdeutlicht die Inferenzzeit als limitierenden Faktor (vgl.~Abbildung~6.4). In der einfachen Szene V1 machte $L_{\text{inference}}$ bei NeuralRecon bereits ca. 70\% der Gesamtlatenz aus; dieser Anteil stieg in der komplexen Szene V3 auf über 85\%. Die Komponenten $L_{\text{network}}$ und $L_{\text{render}}$ blieben hingegen über alle Szenen hinweg relativ konstant bei 10--15\% der Gesamtlatenz. Diese niedrige Basis-Latenz von typischerweise unter 500~ms für die gesamte Kommunikations- und Visualisierungspipeline ist bemerkenswert: Sie demonstriert, dass die gewählte Architektur mit binärem Protokoll, WebSocket-Streaming und GPU-beschleunigter Visualisierung einen vernachlässigbaren Overhead verursacht. Die Datenvolumina von 0,82--1,84~MB für Uploads und 1,60--4,24~MB für Downloads pro Fragment (vgl.~Tabelle~6.3) werden effizient übertragen, und das Chunking-basierte Rendering verzögert die Darstellung nicht wahrnehmbar. Dies impliziert, dass die Gesamtperformance des Systems nahezu linear mit der Inferenzgeschwindigkeit der integrierten Modelle skaliert -- ein zentrales Designziel modularer Architekturen.

Die GPU-Ressourcenauslastung reflektiert diese inferenzdominierten Charakteristika. MASt3R-SLAM und SLAM3R erreichten eine nahezu vollständige GPU-Auslastung von 98--100\%, während die volumetrischen Verfahren mit 44--56\% deutlich moderater ausgelastet waren (vgl.~Tabelle~6.5). Die verwendete NVIDIA GTX 1070 Ti aus dem Jahr 2017 stellt naturgemäß einen Engpass für moderne Transformer-basierte Rekonstruktionsverfahren dar, die für deutlich leistungsfähigere Hardware konzipiert wurden. Entscheidend ist jedoch, dass die geringe Architektur-Latenz eine direkte Skalierbarkeit auf aktuelle GPU-Generationen ermöglicht: Bei Verwendung modernerer Hardware wie einer RTX 4090 würde sich die Inferenzzeit proportional reduzieren, während der System-Overhead konstant niedrig bliebe. Die containerisierte Architektur erlaubt zudem den transparenten Austausch der Backend-Hardware ohne Anpassungen an Frontend oder Kommunikationsprotokoll -- ein wesentlicher Vorteil für den langfristigen Betrieb in sich entwickelnden VR-Infrastrukturen.

\subsection{Modularität und Erweiterbarkeit der Architektur}

Die erfolgreiche Integration von vier methodisch divergenten Rekonstruktionsverfahren -- volumetrische (NeuralRecon, VisFusion), hybride (MASt3R-SLAM) und implizite (SLAM3R) Ansätze -- validiert den modularen Architekturansatz als praktikabel für heterogene KI-Modelle. Besonders relevant ist dabei, dass diese Verfahren auf unterschiedlichen PyTorch-Versionen basieren (1.7 bis 2.1) und teils konfliktäre Framework-Abhängigkeiten aufweisen, die ohne Containerisierung einen gemeinsamen Betrieb unmöglich machen würden. Die Isolation durch Docker-Container ermöglicht hier eine technische Koexistenz, die in monolithischen Architekturen nicht realisierbar wäre.

Der gemessene Integrationsaufwand von durchschnittlich 42--51 Zeilen Dockerfile und 148--203 Zeilen Worker-Code pro Modell (vgl.~Tabelle~6.9) demonstriert die Effizienz der entwickelten Abstraktionsschicht. Diese Werte sind insofern aussagekräftig, als sie ausschließlich den integrationsspezifischen Code umfassen -- die eigentlichen Rekonstruktionsmodelle bleiben unverändert und werden lediglich durch eine einheitliche Kommunikationsschicht gekapselt. Die geringe Varianz zwischen den Modellen (Dockerfile: $\pm 9$ LoC, Worker: $\pm 55$ LoC) deutet darauf hin, dass die Abstraktion stabil über unterschiedliche Modellarchitekturen hinweg funktioniert. Ein wesentlicher Teil des Worker-Codes entfällt auf modellspezifische Datenvorverarbeitung -- etwa die Transformation von Unity-Koordinatensystemen in die jeweiligen Modell-Konventionen -- was als unvermeidlicher Aufwand einzustufen ist, der in jeder Integrationsarchitektur anfallen würde.

Die Effizienz der Kommunikationsschicht manifestiert sich in den kompakten Datenvolumina und der geringen Backend-Ressourcenauslastung. Das binäre Protokoll vermeidet den ca. 33\%igen Overhead von Base64-Kodierung und erreicht durch JPEG-Komprimierung und GLB-Serialisierung Uploadgrößen von 0,82--1,84~MB und Downloadgrößen von 1,60--4,24~MB pro Fragment (vgl.~Tabelle~6.3). Die Unterschiede zwischen Modellen reflektieren deren Repräsentationsformen: Volumetrische Verfahren generieren dichte Meshes (2,19--4,24~MB), während SLAM-basierte Ansätze kompaktere Punktwolken (1,60~MB) erzeugen. Der zentrale Router benötigt lediglich 0,2\% CPU und 200~MB RAM (vgl.~Tabelle~6.6), was die Effizienz der asynchronen Event-Loop-Architektur unterstreicht -- selbst bei paralleler Verwaltung mehrerer Szenen und Clients bleibt der Overhead vernachlässigbar. Die Worker-Container weisen moderate CPU-Auslastungen von 14--32\% und einen RAM-Verbrauch von 2,8--4,1~GB auf, der primär durch die Modellgewichte und deren Aktivierungen während der Inferenz bestimmt wird.

Eine zentrale architekturbedingte Limitation stellt die fehlende persistente Zustandsverwaltung dar. Das System wurde für zustandslose Inferenz konzipiert: Jedes Fragment wird unabhängig verarbeitet, und Container-Neustarts führen zum Verlust aller aktuellen Rekonstruktionsdaten. Für produktive VR-Anwendungen, in denen Nutzer zu früheren Rekonstruktionen zurückkehren oder diese inkrementell erweitern möchten, wäre eine Anbindung an persistente Speicherschichten erforderlich -- etwa S3-kompatible Object Stores für die Rekonstruktionsgeometrie oder Datenbanken mit BLOB-Support für Metadaten und Versionierung. Diese Erweiterung würde jedoch die grundlegende Modularität nicht beeinträchtigen, sondern ließe sich als zusätzliche Backend-Komponente implementieren, die transparent zwischen Router und Workern agiert.

\subsection{Rekonstruktionsqualität und Generalisierbarkeit}

Die vergleichende Evaluation der vier integrierten Rekonstruktionsverfahren demonstriert einen wesentlichen Vorteil der modularen Architektur: Sie ermöglicht die systematische Identifikation von Anwendungsszenarien, für die bestimmte Verfahrensklassen besser geeignet sind. Die gemessenen Qualitätsunterschiede zwischen synthetischen und realen VR-Umgebungen offenbaren dabei weniger fundamentale Algorithmen-Limitationen, sondern primär die Auswirkungen von Hardware- und SDK-Restriktionen auf die Rekonstruktionsqualität -- eine Erkenntnis, die für das Design produktiver VR-Rekonstruktionssysteme hochrelevant ist.

In den synthetischen Unity-Szenen V1--V3 zeigten volumetrische Verfahren eine konsistente Performance mit F-Scores zwischen 0,41--0,68 (NeuralRecon) und 0,54--0,62 (VisFusion), während SLAM-basierte Ansätze bei zunehmender Szenenkomplexität von 0,50--0,61 auf 0,42--0,45 abfielen (vgl.~Tabelle~6.7). Die qualitativen Bewertungen ergänzen dieses Bild: Volumetrische Verfahren erzielten hohe Werte bei Artefaktfreiheit und Oberflächenqualität, während SLAM-Verfahren bei Detailtreue überlegen waren, aber unter Konsistenzproblemen litten (vgl.~Tabelle~6.8). Diese Ergebnisse entsprechen den theoretischen Erwartungen und bestätigen die grundsätzliche Funktionsfähigkeit der integrierten Modelle unter idealen Sensorbedingungen mit präzisen Kameraparametern.

Die Evaluation unter realen VR-Bedingungen mit der Meta Quest 3 offenbarte jedoch eine ausgeprägte Performance-Inversion. Volumetrische Verfahren zeigten fragmentierte Rekonstruktionen mit großflächigen Lücken und erheblichen Diskontinuitäten (vgl.~Abbildung~6.13), während SLAM-basierte Ansätze deutlich höhere Vollständigkeit und Detailgrad erreichten (vgl.~Abbildung~6.14). Diese Umkehrung lässt sich auf eine fundamentale Limitation der Quest 3 Passthrough API zurückführen: Die API liefert keine präzisen Timestamps für die aufgenommenen Kamerabilder, wodurch eine exakte zeitliche Zuordnung zwischen Bildern und den vom Tracking-System bereitgestellten Extrinsiken nicht möglich ist. Volumetrische Verfahren wie NeuralRecon und VisFusion setzen jedoch voraus, dass die Kamerapose-Daten exakt zum Aufnahmezeitpunkt des jeweiligen Frames gehören, da sie diese direkt für die TSDF-Fusion nutzen. Selbst geringe zeitliche Versätze von wenigen Millisekunden führen bei Kopfbewegungen zu räumlichen Inkonsistenzen, die sich über die gesamte Rekonstruktion propagieren und großflächige Artefakte erzeugen. Zusätzlich weisen die für Tracking optimierten Kameras eine signifikant verminderte Bildqualität auf -- geringer Dynamikumfang, ausgeprägte Rolling-Shutter-Effekte und erhöhtes Bildrauschen --, was die Tiefenschätzung der CNN-basierten Netzwerke weiter beeinträchtigt.

SLAM-basierte Verfahren umgehen diese Problematik durch eine fundamental andere Architektur: SLAM3R und MASt3R-SLAM nutzen die externen Extrinsiken überhaupt nicht, sondern schätzen Kameraposen vollständig selbst durch explizite Feature-Matching- und Optimierungsschritte. Diese interne Pose-Estimation ist unabhängig von SDK-Timestamps und kann zeitliche Inkonsistenzen kompensieren, da sie ausschließlich auf der Bildfolge selbst basiert. Die höhere Robustheit gegenüber Sensor-Artefakten erklärt auch die beobachteten längeren Inferenzzeiten (vgl.~Tabelle~6.4: SLAM 0,03--0,13 fps vs. volumetrisch 0,30--0,62 fps) -- die iterative Pose-Optimierung ist rechenintensiver als die direkte Nutzung externer Tracking-Daten. Aus Systemperspektive manifestiert sich hier ein klassischer Trade-off: Volumetrische Verfahren sind schneller, benötigen aber präzise Eingabedaten; SLAM-Verfahren sind langsamer, aber robuster gegenüber SDK-Limitationen.

Aus Systemperspektive ist diese Beobachtung zweifach relevant: Erstens demonstriert sie den praktischen Mehrwert der modularen Architektur -- ohne die Möglichkeit, verschiedene Verfahren unter identischen Bedingungen zu evaluieren, wäre diese Hardware-Sensitivität schwer systematisch zu identifizieren gewesen. Zweitens impliziert sie deployment-spezifische Empfehlungen: Für VR-Anwendungen mit Consumer-Hardware wie der Quest 3 sollten trotz höherer Latenz SLAM-basierte Verfahren bevorzugt werden, während volumetrische Ansätze ihre Stärken bei kontrollierten Kamera-Setups mit präzisen Extrinsiken ausspielen. Die beobachteten Trade-offs zwischen Inferenzgeschwindigkeit (volumetrisch: 0,30--0,62 fps, SLAM: 0,03--0,13 fps) und Robustheit gegenüber Sensor-Artefakten (vgl.~Abbildung~6.13 vs. 6.14) sind damit nicht als inhärente Algorithmen-Limitationen zu verstehen, sondern als Konsequenz des jeweiligen Deployment-Kontexts.

\subsection{Eignung für VR-Anwendungen}

Die Integration von RTReconstruct in das Va.Si.Li-Lab demonstriert die praktische Einsetzbarkeit der entwickelten Architektur in produktiven VR-Umgebungen. Die gewählte Strategie der additiven Integration erwies sich dabei als tragfähig: RTReconstruct wurde als optionaler, zuschaltbarer Dienst in die bestehende Mehrbenutzerinfrastruktur eingebettet, ohne etablierte Szenenlogik oder Interaktionsmechanismen zu verändern. Das implementierte Rollen- und Szenenkonzept (Host/Visitor) ermöglichte eine klare Aufgabentrennung zwischen aktiver Rekonstruktion und passivem Konsum der Ergebnisse, wobei die Mehrbenutzerumgebung auch bei deaktivierter Rekonstruktion vollständig funktionsfähig blieb.

Die Systemstabilität unter realen Nutzungsbedingungen wurde durch umfangreiche Testläufe validiert. In 15 Testdurchgängen über eine Gesamtdauer von 6 Stunden traten 0 Verbindungsabbrüche auf (vgl.~Tabelle~6.6). Die Multi-Szenen-Unterstützung funktionierte mit 2 parallelen Szenen und 4 verbundenen Clients ohne Funktionseinbußen, was die Robustheit der asynchronen WebSocket-Architektur und des Fan-Out-Mechanismus auch unter Mehrbenutzer-Last bestätigt. Diese Zuverlässigkeit bildet die Grundlage für den produktiven Einsatz in kollaborativen VR-Szenarien.

Die Visualisierungs-Performance variiert jedoch deutlich zwischen den Repräsentationsformen. Bei volumetrischen Mesh-Rekonstruktionen (NeuralRecon, VisFusion) blieb die Frame-Rate mit 72 FPS auf Baseline-Niveau. Die implementierte Spatial-Hashing-Visualisierung mit Chunk-Größen von $5 \times 5 \times 5$ Metern ermöglichte hier effizientes Frustum-Culling und sicherte eine flüssige Darstellung auch bei größeren Geometrien. SLAM-basierte Punktwolken-Rekonstruktionen erreichten bei 100.000 dargestellten Punkten konstant 38 FPS. Eine weitere Reduktion der Punktzahl zur Performance-Steigerung erwies sich als nicht praktikabel, da dies zu erheblichem Detailverlust führte und die primäre Stärke der SLAM-Verfahren -- ihre hohe geometrische Auflösung -- untergraben würde. Die Begrenzung auf 100.000 Punkte stellt damit eine Kompromisslösung dar, die ausreichende Darstellungsqualität bei akzeptabler Performance auf mobiler VR-Hardware gewährleistet.

Eine weitere Herausforderung für den produktiven Einsatz betrifft die globale Registrierung der SLAM-Rekonstruktionen. SLAM3R und MASt3R-SLAM erzeugen Rekonstruktionen in einem modellinternen Koordinatensystem mit willkürlicher Skalierung, da sie Kameraposen vollständig selbst schätzen und keine externen Referenzen nutzen. Volumetrische Verfahren hingegen verwenden die vom Meta SDK bereitgestellten Extrinsiken direkt für die TSDF-Fusion und rekonstruieren dadurch automatisch im korrekten Maßstab des VR-Tracking-Systems. Um SLAM-Outputs in die Unity-Szene zu integrieren, ist derzeit manuelles Skalieren und Ausrichten durch den Nutzer erforderlich. Dieser Arbeitsschritt unterbricht den automatisierten Workflow und erschwert spontane Rekonstruktionsszenarien, in denen Nutzer unmittelbar mit den erfassten Geometrien interagieren möchten. Die Problematik illustriert eine grundlegende Herausforderung bei der Integration von selbst-kalibrierten Rekonstruktionsverfahren in bestehende VR-Umgebungen mit etablierten Tracking-Referenzsystemen.

\subsection{Beantwortung der Forschungsfrage}

Die zentrale Forschungsfrage \glqq Wie gut eignet sich eine modulare, containerisierte Systemarchitektur zur Integration verschiedener Echtzeit-3D-Rekonstruktionsverfahren in eine bestehende Virtual-Reality-Umgebung?\grqq{} kann auf Basis der durchgeführten Evaluation differenziert beantwortet werden. Die Architektur erweist sich als gut bis sehr gut geeignet für Forschungs- und Evaluationskontexte, zeigt jedoch spezifische Stärken und Limitationen, die von den jeweiligen Deployment-Anforderungen abhängen.

Die modulare Architektur erfüllt ihre Kernziele auf technischer Ebene umfassend. Die erfolgreiche Integration von vier methodisch divergenten Rekonstruktionsverfahren mit unterschiedlichen Framework-Abhängigkeiten (PyTorch 1.7--2.1) und Repräsentationsformen (volumetrisch, hybrid, implizit) validiert das Containerisierungskonzept als praktikabel (vgl.~Tabelle~6.9: 42--51 LoC Dockerfile, 148--203 LoC Worker-Code pro Modell). Die asynchrone WebSocket-Architektur gewährleistet eine robuste, latenzarme Datenstromverarbeitung mit 0 Verbindungsabbrüchen in 6 Stunden Testbetrieb (vgl.~Tabelle~6.6). Die niedrige Basis-Latenz von typischerweise unter 500~ms für Netzwerk- und Rendering-Komponenten demonstriert, dass die Gesamtperformance nahezu linear mit der Modell-Inferenzgeschwindigkeit skaliert -- ein zentrales Designziel modularer Architekturen (vgl.~Abbildung~6.4). Die Integration in das Va.Si.Li-Lab gelang additiv ohne disruptive Eingriffe, was die Wiederverwendbarkeit für andere VR-Plattformen nahelegt.

Aus wissenschaftlicher Perspektive ermöglichte die Architektur systematische Erkenntnisse, die über isolierte Modell-Benchmarks hinausgehen. Die identifizierte Performance-Inversion zwischen synthetischen und realen VR-Umgebungen (vgl.~Tabelle~6.7, Abbildung~6.13/6.14) wäre ohne vergleichende Evaluation unter identischen Bedingungen kaum systematisch zu erfassen gewesen. Die Rückführung auf SDK-Limitationen der Meta Quest 3 (fehlende Frame-Timestamps) statt auf inhärente Algorithmen-Schwächen zeigt den Mehrwert kontrollierter Vergleichsstudien. Ebenso offenbarte die Evaluation deployment-spezifische Trade-offs: Volumetrische Verfahren sind schneller (0,30--0,62 fps), benötigen aber präzise Extrinsiken; SLAM-Verfahren sind langsamer (0,03--0,13 fps), aber robuster gegenüber Sensor-Artefakten. Die Architektur erlaubt es, solche Entscheidungen kontextspezifisch zu treffen, ohne Code-Anpassungen vornehmen zu müssen.

Limitationen bestehen primär in drei Bereichen. Erstens erreichen SLAM-Verfahren auf der verwendeten NVIDIA GTX 1070 Ti eine GPU-Auslastung von 98--100\% bei vollständiger VRAM-Auslastung (8~GB), was den parallelen Betrieb mehrerer Modelle einschränkt (vgl.~Tabelle~6.5). Dieser Engpass ist hardware-spezifisch und würde sich mit moderneren GPUs deutlich reduzieren, unterstreicht aber die Relevanz der Architektur-Eigenschaft, dass die niedrige Basis-Latenz eine direkte Skalierung auf leistungsfähigere Hardware ohne Code-Anpassungen ermöglicht. Zweitens fehlt eine persistente Zustandsverwaltung, sodass Container-Neustarts zum Datenverlust führen. Drittens erfordert die Integration von SLAM-Rekonstruktionen manuelle Skalierung durch Nutzer, da diese in modellinternen Koordinatensystemen vorliegen. Diese Limitationen sind nicht architekturinhärent, sondern reflektieren Hardware-Constraints der Evaluationsumgebung, Implementierungsentscheidungen bzw. fehlende Standardisierung zwischen Rekonstruktionsverfahren und VR-Tracking-Systemen.

Zusammenfassend eignet sich die entwickelte Architektur sehr gut als flexible Integrationsplattform für Forschungs- und Evaluationszwecke. Sie ermöglicht den transparenten Austausch von Rekonstruktionsverfahren, die systematische Identifikation deployment-spezifischer Eignung und die Generierung vergleichender Erkenntnisse unter kontrollierten Bedingungen. Für produktive VR-Anwendungen mit Echtzeitanforderungen empfiehlt sich eine kontextabhängige Verfahrenswahl: volumetrische Ansätze für synthetische Umgebungen oder präzise Kamera-Hardware, SLAM-basierte Ansätze für Consumer-VR mit SDK-Limitationen. Die Modularität des Systems stellt sicher, dass solche Entscheidungen deployment-spezifisch getroffen werden können, ohne die Frontend- oder Backend-Implementierung grundlegend anpassen zu müssen -- ein zentrales Erfolgskriterium für langfristig wartbare VR-Forschungsinfrastrukturen.
