\chapter{Stand der Technik}

Dieses Kapitel verortet die Arbeit im Forschungsstand der echtzeitfähigen 3D-Rekonstruktion und führt begründet in die vier integrierten Modelle NeuralRecon, VisFusion, MASt3R-SLAM und SLAM3R ein, deren Auswahl maßgeblich durch die Anforderungen an Echtzeit, Modularität und VR-Tauglichkeit im geplanten System motiviert ist.

\section{Überblick und Zielsetzung}

Echtzeit-3D-Rekonstruktion aus monokularen RGB-Videos hat sich von klassischen, überwiegend offline-orientierten SfM/Photogrammetrie-Pipelines hin zu neuronalen, streamingfähigen Verfahren entwickelt, die Geometrie fortlaufend in konsistenter Qualität liefern und damit für interaktive VR-Szenarien relevant sind. Für ein containerisiertes, modular integrierbares System sind Verfahren erforderlich, die kontinuierlich Eingabefragmente verarbeiten, latenzarm Resultate liefern und robuste Qualität in heterogenen Szenen erreichen, um im Unity-Frontend unmittelbar visualisiert werden zu können.

\section{NeuralRecon}

NeuralRecon gilt als ein früher Meilenstein für kohärente, echtzeitfähige 3D-Rekonstruktion aus monocularen Videos, indem es ein mehrstufiges, coarse-to-fine Netz einsetzt, das in einem spärlich besetzten voxelbasierten TSDF-Raum rekonstruiert. Kern ist eine sequentielle Feature-Fusion über GRU-Module entlang von Videofragmenten, wodurch zeitliche Konsistenz und lokale Kohärenz im TSDF verbessert werden und Frame-für-Frame-Drift reduziert wird. Die Architektur verarbeitet Fenster von Bildern mit bekannten Intrinsiken/Extrinsiken und priorisiert Rechenbudget auf aktive Voxelregionen, was die Laufzeit reduziert und die Nutzung in Streaming-Pipelines begünstigt.

Relevanz für diese Arbeit: Die fragmentbasierte, rekurrente Fusion passt zu einem WebSocket-Streaming-Setup mit fensterweiser Übertragung aus dem VR-Frontend, während die direkte TSDF-Ausgabe eine schnelle Konvertierung in Mesh/Punktwolke für die Unity-Visualisierung ermöglicht.

\section{VisFusion}

VisFusion erweitert volumetrische Online-Rekonstruktion um explizite Sichtbarkeitsmodellierung zwischen Mehransichten, wodurch Okklusionen, Foreshortening und feingranulare Oberflächen besser behandelt werden. Das Verfahren nutzt visibility-aware Feature-Fusion, residuale TSDF-Prädiktion über mehrere Skalen und strahlbasierte Fokussierung aktiver Regionen, was die Detailtreue und Konvergenzstabilität im Online-Setup verbessert. Durch die Verbindung aus Sichtbarkeitsinferenz und sparsamer Volumenaktualisierung behält VisFusion Echtzeitfähigkeit bei höherer geometrischer Präzision in komplexen Szenen bei.

Relevanz für diese Arbeit: VisFusion ergänzt NeuralRecon komplementär um robuste Sichtbarkeitsbehandlung und liefert damit einen wichtigen Kontrastpunkt in der Evaluation von Vollständigkeit und Detailgrad bei identischem Streaming- und Container-Setup.

\section{MASt3R-SLAM}

MASt3R-SLAM integriert dichte SLAM-Optimierung mit starken 3D-Priors aus MASt3R, insbesondere Two-View-Punktmaps und Confidence-Informationen, um Tracking und Mapping in Echtzeit zu stabilisieren. Durch GPU-beschleunigte Implementierung erreicht das System praxisnahe Frameraten und kombiniert lokale Dichte mit globaler Konsistenz, inklusive robustem Loop-Closure über effiziente visuelle Retrieval-Mechanismen. Die Nutzung der 3D-Priors verbessert sowohl Pose-Schätzung als auch dichte Rekonstruktion in herausfordernden Textur- und Bewegungsbedingungen.

Relevanz für diese Arbeit: Als SLAM-orientierter Vertreter liefert MASt3R-SLAM eine Pipeline, die besonders für kontinuierliche VR-Nutzerbewegungen geeignet ist, und dient als methodischer Gegenpol zu rein volumetrischen oder rein strahlbasierten Ansätzen in derselben modularen Backend-Architektur.

\section{SLAM3R}

SLAM3R verbindet strahlbasierte NeRF-Repräsentationen mit SLAM-Mechanismen und GPU-Beschleunigung, um photometrisch konsistente, dichte Rekonstruktionen in interaktiven Szenarien zu ermöglichen. Die strahlbasierte Modellierung erschließt hochqualitative Oberflächendetails und neuartige Sichten, während die SLAM-Kopplung Kameraposen und Szenenkohärenz im Online-Betrieb stabilisiert. Damit steht SLAM3R für das Paradigma der neuralen Strahlfelder im Vergleich zu TSDF-Volumen und dichten SLAM-Karten innerhalb derselben Systemintegration.

Relevanz für diese Arbeit: Als strahlbasierter Ansatz erweitert SLAM3R das Methodenspektrum und erlaubt eine systematische Gegenüberstellung von Volumen-, SLAM- und NeRF-Repräsentationen hinsichtlich Latenz, Qualität und Integrationsaufwand in der containerisierten Streaming-Pipeline.

\section{Begründung der Modellwahl}

\begin{itemize}
    \item Komplementäre Paradigmen: Die Auswahl deckt drei maßgebliche Denkrichtungen ab—volumetrische TSDF-Rekonstruktion (NeuralRecon, VisFusion), SLAM-gestützte dichte Karten mit Priors (MASt3R-SLAM) und strahlbasierte NeRF-Modelle (SLAM3R)—und ermöglicht so eine vergleichende Evaluation unter identischen Systembedingungen.
    \item Echtzeit- und Streaming-Tauglichkeit: Alle vier Modelle sind für kontinuierliche Verarbeitung mit latenzarmen Updates konzipiert oder dafür bekannt implementierbar zu sein, was die Integration in eine WebSocket-basierte, fragmentorientierte Pipeline mit sofortiger Rückführung ins Unity-Frontend ermöglicht.
    \item Integrierbarkeit und Modularität: Verfügbare Open-Source-Referenzen bzw. dokumentierte Pipelines begünstigen containerisierte Worker, standardisierte Eingaben (Bilder, Intrinsiken/Extrinsiken) und konsistente Ausgaben (TSDF/Mesh/Punktwolke oder strahlbasierte Darstellungen), was die Anforderungen des modularen Backend-Designs erfüllt.
    \item Evaluationsbreite: Die Modelle unterscheiden sich in Stärken bei Genauigkeit, Vollständigkeit, Latenz und Robustheit, was eine aussagekräftige Systembewertung entlang der in Kapitel 2 definierten Qualitätskriterien ermöglicht, ohne die Infrastruktur zu verändern.
\end{itemize}

\section{Einordnung in die Systemarchitektur}

Die Systemarchitektur entkoppelt VR-Frontend und Rekonstruktionsmodelle über WebSockets und containerisierte Worker, wodurch jeweils fensterbasierte Kamerabilder samt Posen/Inertialdaten an die Modelle gestreamt und Resultate als GLB/Punktwolken zurückübertragen werden. NeuralRecon und VisFusion profitieren von vorberechneten Intrinsiken/Extrinsiken und liefern volumetrische Oberflächen, während MASt3R-SLAM mit 3D-Priors die Pose- und Kartenqualität im Fluss stabilisiert und SLAM3R als strahlbasierter Dienst alternative Qualitäts-Latenz-Profile beisteuert. Die modulare Implementierung erlaubt parallele oder austauschbare Ausführung, wodurch die Modelle unter identischer Datenlogik verglichen und je nach Szenario dynamisch ausgewählt werden können.