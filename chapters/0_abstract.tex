\addchap*{\centering Abstract}

\begin{singlespace}
Die digitale Rekonstruktion realer Umgebungen in Virtual-Reality-Anwendungen ist vielversprechend, wird in der Praxis jedoch häufig durch Integrationsaufwand und heterogene Rekonstruktionspipelines erschwert.
Diese Arbeit untersucht, inwieweit eine modulare Systemarchitektur die Einbindung und den vergleichbaren Betrieb unterschiedlicher Verfahren zur fortlaufenden 3D-Rekonstruktion in einer bestehenden VR-Umgebung unterstützt.

Zu diesem Zweck wird ein System entworfen und prototypisch umgesetzt, das Daten aus der VR-Anwendung entgegennimmt, eine rekonstruktive Verarbeitung in austauschbaren Modulen ermöglicht und die entstehenden 3D-Ergebnisse wieder in die laufende Anwendung zurückführt.
Mithilfe einer experimentellen Evaluation werden die Funktionalität, das Laufzeitverhalten, die Stabilität und die Rekonstruktionsqualität in mehreren Szenarien untersucht.
\end{singlespace}

\cleardoublepage
