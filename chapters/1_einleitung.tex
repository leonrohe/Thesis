\chapter{Einleitung}

\section{Motivation}
Fortschritte in Hand-, Augen- und Mimik-Erkennung ermöglichen es uns heutzutage bei Begegnungen mit anderen Nutzern in der Virtuellen Realität (VR) viel expressiver und natürlicher zu kommunizieren, als früher. \needcitation Dennoch bleiben die meisten Begegnungen im VR-Kontext in einem gewissen Maß künstlich, da sie sich immer noch in virtuellen, von der Realität fernen, Räumen abspielen. Für viele Anwendungsbereiche ist dies jedoch hinderlich. So profitieren beispielsweise Bereiche wie Remote-Arbeit, Bildung, Architektur und Industrie, im VR-Kontext stark davon, dass sich Begegnungen und Interaktionen in realen, detailgetreuen Umgebungen abspielen. \needcitation

Eine der zentralen technischen Herausforderungen in diesem Bereich ist jedoch die Rekonstruktion dieser Umgebungen. Die herkömmliche Photogrammetrie bietet auf den ersten Blick eine optimale Lösung. Mithilfe gängiger Kameratechnologien erzeugt sie 3D-Modelle der aufgenommenen Umgebung. In der Praxis sind die meisten Photogrammetrie-Verfahren jedoch ausschließlich für Offline-Anwendungen ausgelegt, da sie ein hohes Maß an Rechenleistung und Verarbeitungszeiten benötigen. \needcitation Dadurch wird eine Integration in interaktive Systeme erschwert. Parallel dazu entstanden zahlreiche Verfahren zur Rekonstruktion in Echtzeit aus monokularen RGB-Sequenzen. \needcitation Ihre Vielfalt führt jedoch zu einem Fragmentierungsproblem: Die Modelle sind oft schwer austauschbar, erfordern spezialisierte Laufzeitumgebungen und angepasste Kommunikationsschnittstellen. Dadurch fehlt bislang ein allgemeines Framework, das verschiedene Ansätze interoperabel integriert.

Ein modularer, containerisierter Systemansatz bietet hierfür eine mögliche Lösung. Durch die Entkopplung von Modell-, Kommunikations- und Visualisierungsschicht kann die Integration heterogener Rekonstruktionsverfahren standardisiert werden. Dies würde nicht nur Vergleichbarkeit und Austauschbarkeit fördern, sondern auch eine nachhaltige Plattform für zukünftige Entwicklungen schaffen.

Bislang ist jedoch unklar, wie performant und zuverlässig ein solch modulares System für Echtzeit-VR-Anwendungen tatsächlich ist.

\section{Zielsetzung}
Ziel dieser Arbeit ist die Konzeption, Implementierung und Evaluation eines modularen Systems zur Echtzeit-3D-Rekonstruktion in einer bestehenden VR-Umgebung. Im Fokus steht dabei nicht die Weiterentwicklung einzelner Rekonstruktionsverfahren, sondern die Entwicklung einer skalierbaren Integrationsarchitektur, die verschiedene Modelle kapselt und über eine standardisierte Schnittstelle mit dem VR-Frontend kommuniziert. Als Frontend dient die Va.Si.Li-Lab-Umgebung des Text Technology Lab der Goethe-Universität Frankfurt, eine Unity-basierte, mehrbenutzerfähige VR-Plattform für simulationsbasiertes Lernen. \needcitation

Die zentrale Fragestellung der Arbeit ist dabei folgende:
\begin{itemize}
    \item Wie gut eignet sich eine modulare, containerisierte Systemarchitektur zur Echtzeit-Integration verschiedener 3D-Rekonstruktionsverfahren in eine Virtual-Reality Umgebung?
\end{itemize}

Zur Beantwortung dieser Forschungsfrage wird zunächst eine Systemarchitektur konzipiert, welche Steuerlogik, Kommunikationsschicht und die einzelnen Rekonstruktionsmodule klar voneinander trennt. Die einzelnen Module sind dabei vollständig voneinander isoliert, aber über standardisierte Schnittstellen austausch- und erweiterbar. Im Anschluss erfolgt die Implementierung eines Prototyps, welcher eine vollständige Kommunikationskette zwischen dem Unity-basierten Va.Si.Li.-Lab-Frontend und dem modularen Backend ermöglicht.

n einer experimentellen Evaluation wird anschließend die Leistungsfähigkeit des Systems unter Echtzeitbedingungen untersucht. Bewertet werden dabei (1) die technische Performance anhand von Latenz, Durchsatz und Ressourcenauslastung, (2) die Systemstabilität bei Verbindungsabbrüchen und Containerneustarts sowie (3) die Rekonstruktionsqualität anhand von Genauigkeit, Vollständigkeit und visueller Plausibilität.

\section{Aufbau der Arbeit}
% Die Arbeit ist dabei in sieben Kapitel gegliedert, welche vom theoretischen Hintergrund über Konzeption bis hin zur praktischen Umsetzung und Evaluation reichen. In Kapitel 2 werden dabei grundlegende Konzepte und Begriffe erklärt, welche für das Verständnis der darauf folgenden Kapitel benötigt werden. Dazu gehören beispielsweise Virtual Reality, 3D-Rekonstruktionsverfahren, Containerisierung, Kommunikationsprotokolle und die bereits existierende Unity Umgebung des Va.Si.Li-Lab.
% 
% Kapitel 3 behandelt den aktuellen Stand der Technik. Hier werden bestehende Pipelines für die 3D-Rekonstruktion und containerisierte Systeme aus Forschung und Entwicklung betrachtet. Außerdem wird die eigene Arbeit in diesen Kontext eingeordnet.
% 
% Kapitel 4 behandelt die Konzeption des entwickelten Systems. Es werden die Wahl der Systemarchitektur, das Design der Schnittstelle, der Kommunikationsfluss zwischen Backend und existierendem VR-Frontend und das Modularisierungskonzept erklärt.
% 
% In Kapitel 5 folgt die Implementierung des entworfenen Konzepts. Hier werden sowhol die Backend- als auch die Frontend-Umsetzung erläutert. Außerdem werden die benötigten Schritte zur Einbindung in die Va.Si.Li-Lab Umgebung gezeigt.
% 
% Kapitel 6 beschäftigt sich mit der Evaluation des Systems. Hier wird erklärt, welche Ziele mit der Evaluierung verfolgt werden, unter welchen Bedingungen sie wie durchgeführt wurden und welche Ergebnisse dabei erzielt wurden.
% 
% Abschließend fasst Kapitel 7 die zentralen Erkenntnisse der Arbeit zusammen, diskutiert Limitationen des entwickelten Systems und gibt einen Ausblick auf mögliche zukünftige Entwicklungen und Optimierungen.
% 