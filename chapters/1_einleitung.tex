\chapter{Einleitung}

\section{Motivation}
Fortschritte in Hand-, Augen- und Mimik-Erkennung ermöglichen es uns heutzutage bei Begegnungen mit anderen Nutzern in der Virtuellen Realität (VR) viel expressiver und natürlicher zu kommunizieren, als früher. \needcitation Dennoch bleiben die meisten Begegnungen im VR-Kontext zu einem gewissen Grad künstlich, da sie sich immer noch in künstlichen, von der Realität fernen, Räumen abspielen. Für viele Anwendungsbereiche ist dies jedoch hinderlich. So profitieren beispielsweise Bereiche wie Remote-Arbeit, Bildung, Architektur und Industrie, im VR-Kontext stark davon, dass sich Begegnungen und Interaktionen in realen, detailgetreuen Umgebungen abspielen. \needcitation

Eine der zentralen technischen Herausforderungen in diesem Bereich ist jedoch die Rekonstruktion dieser Umgebungen. Die herkömmliche Photogrammetrie bietet auf den ersten Blick eine optimale Lösung. Mithilfe gängiger Kameratechnologien erzeugt sie 3D-Modelle der aufgenommenen Umgebung. In der Praxis sind die meisten Photogrammetrie-Verfahren jedoch ausschließlich für Offline-Anwendungen ausgelegt, da sie ein hohes Maß an Rechenleistung und Verarbeitungszeiten benötigen. \needcitation Dadurch wird eine Integration in interaktive Systeme erschwert. Gleichzeitig gibt es aber auch im Bereich der Echtzeit-Photogrammetrie zahllose Ansätze. \needcitation Diese Vielfalt an unterschiedlichen Modellen erschwert eine einheitliche Einbindung dieser in bereits bestehende Systeme, denn sie sind oft inkompatibel, schwer austauschbar oder es wird eine angepasste Kommunikationslogik zwischen Backend und VR-Frontend erforderlich. \needcitation

Ein modularer, containerisierter System-Ansatz könnte für diese Problem Abhilfe beschaffen. Indem bereits existierende Rekonstruktionsmodelle von der Kommunikation zwischen Back- und Frontend entkoppelt werden, ermöglicht dieser Ansatz eine einfachere Einbindung dieser, was einfacherere Tests und Vergleiche ermöglicht - ein klarer Vorteil für einen sich rasch weiterentwickelnden Forschungsbereich.

Bislang ist jedoch unklar, wie performant und zuverlässig ein solch modulares System für Echtzeit-VR-Anwendungen tatsächlich ist.

\section{Zielsetzung}
Das Ziel dieser Arbeit ist deshalb die Konzeption, Integration und Evaluation eines solchen modularen Systems in ein bereits bestehendes VR-Frontend. Ziel ist nicht die Weiterentwicklung photogrammetrischer Verfahren. Das hierfür gewählte Fontend ist die Va.Si.Li-Lab Umgebung des Text-Technology-Labs der Goethe Universität Frankfurt. \needcitation

Die zentrale Fragestellung der Arbeit ist dabei folgende:
\begin{itemize}
    \item Wie gut eignet sich eine modulare, containerisierte Systemarchitektur zur Echtzeit-Integration verschiedener 3D-Rekonstruktionsverfahren in eine Virtual-Reality Umgebung?
\end{itemize}

Zur Beantwortung dieser Forschungsfrage wird zunächst eine Systemarchitektur konzipiert, welche Steuerlogik, Kommunikationsschicht und die einzelnen Rekonstruktionsmodule klar voneinander trennt. Die einzelnen Module sind dabei vollständig voneinander isoliert, aber über standardisierte Schnittstellen austausch- und erweiterbar. Im Anschluss erfolgt die Implementierung eines Prototyps, welcher eine vollständige Kommunikationskette zwischen dem Unity-basierten Va.Si.Li.-Lab-Frontend und dem modularen Backend ermöglicht.

In einer experimentellen Evaluation dann wird die Leistungsfähigkeit des Systems unter Echtzeitbedingungen untersucht. Bewertet werden dabei Latenz, Durchsatz, Ressourcenauslastung und Stabilität der Architektur sowie die Qualität der Rekonstruktionen in Bezug auf Vollständigkeit und Genauigkeit. Darüber hinaus wird geprüft, wie einfach sich neue Rekonstruktionsmodule integrieren lassen und ob das System die geforderte Flexibilität erfüllt.

\section{Aufbau der Arbeit}
% Die Arbeit ist dabei in sieben Kapitel gegliedert, welche vom theoretischen Hintergrund über Konzeption bis hin zur praktischen Umsetzung und Evaluation reichen. In Kapitel 2 werden dabei grundlegende Konzepte und Begriffe erklärt, welche für das Verständnis der darauf folgenden Kapitel benötigt werden. Dazu gehören beispielsweise Virtual Reality, 3D-Rekonstruktionsverfahren, Containerisierung, Kommunikationsprotokolle und die bereits existierende Unity Umgebung des Va.Si.Li-Lab.
% 
% Kapitel 3 behandelt den aktuellen Stand der Technik. Hier werden bestehende Pipelines für die 3D-Rekonstruktion und containerisierte Systeme aus Forschung und Entwicklung betrachtet. Außerdem wird die eigene Arbeit in diesen Kontext eingeordnet.
% 
% Kapitel 4 behandelt die Konzeption des entwickelten Systems. Es werden die Wahl der Systemarchitektur, das Design der Schnittstelle, der Kommunikationsfluss zwischen Backend und existierendem VR-Frontend und das Modularisierungskonzept erklärt.
% 
% In Kapitel 5 folgt die Implementierung des entworfenen Konzepts. Hier werden sowhol die Backend- als auch die Frontend-Umsetzung erläutert. Außerdem werden die benötigten Schritte zur Einbindung in die Va.Si.Li-Lab Umgebung gezeigt.
% 
% Kapitel 6 beschäftigt sich mit der Evaluation des Systems. Hier wird erklärt, welche Ziele mit der Evaluierung verfolgt werden, unter welchen Bedingungen sie wie durchgeführt wurden und welche Ergebnisse dabei erzielt wurden.
% 
% Abschließend fasst Kapitel 7 die zentralen Erkenntnisse der Arbeit zusammen, diskutiert Limitationen des entwickelten Systems und gibt einen Ausblick auf mögliche zukünftige Entwicklungen und Optimierungen.
% 