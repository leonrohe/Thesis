\chapter{Einleitung}

\section{Motivation}
Während sich Virtuelle Realität (VR), weg von einer ursprünglich reinen Simulationsumgebung, hin zu einem Medium entwickelt, das reale und virtuelle Welten verschmelzen lässt, findet sie auch zunehmend in anderen Bereichen Anwendung. \needcitation So profitieren beispielsweise die Bereiche Fernzusammenarbeit, Bildung, Architektur und Industrie, im VR-Kontext, erheblich davon, dass reale Räume erfasst und in eine virtuelle Umgebung übertragen werden können. \needcitation Dies bietet die Möglichkeit für immersivere und glaubwürdigere Interaktionsszenarien, in denen physische und virtuelle Elemente nahtlos koexistieren können.

Eine der zentralen technischen Herausforderungen in dieser Richtung ist dabei jedoch die Echtzeit-Rekonstruktion dieser realen Umgebungen. Standardmäßige Photogrammetrie bietet eine theoretisch sehr attraktive Grundlage, da sie handelsübliche Kameratechnologien nutzt, um detaillierte 3D-Rekonstruktionen zu erstellen. In der Praxis sind die meisten photogrammetrischen Verfahren jedoch ausschließlich für Offline-Anwendungen konzipiert, da sie eine hohe Rechenleistung und lange Verarbeitungszeiten erfordern, was die Integration in interaktive Systeme erschwert. \needcitation

Gleichzeitig stehen aber auch im Bereich der Echtzeit-Photogrammetrie zahllose Ansätze zur Verfügung, seien es nun klassische Multi-View-Methoden über SLAM-Verfahren bis hin zu neuronalen Rekonstruktionsmodellen. \needcitation Diese Vielfalt erschwert eine einheitliche Einbindung in bestehende VR-Anwendungen: Modelle sind oft inkompatibel, schwer austauschbar und erfordern angepasste Kommunikationslogik zwischen Backend und VR-Frontend.

Der modulare, containerisierte System-Ansatz, bei dem die einzelnen Rekonstruktionsmodule vom eigentlichen Ablauf entkoppelt und über standardisierte Schnittstellen mit der VR-Anwendung kommunizieren, bietet eine mögliche Lösung. Mit diesem Ansatz könnten neue Techniken einfacher integriert, getestet und verglichen werden - ein klarer Vorteil für einen sich rasch weiterentwickelnden Forschungsbereich.

Bislang ist jedoch unklar, wie performant und zuverlässig ein solch modulares System für Echtzeit-VR-Anwendungen tatsächlich ist.

% Trotz dieser Potenziale ist bislang unklar, wie zuverlässig und performant ein solches modulares System in einer Echtzeit-VR-Umgebung tatsächlich arbeitet. Genau diese Fragestellung steht im Mittelpunkt dieser Arbeit: Es soll untersucht werden, in welchem Maße eine modulare, photogrammetriebasierte Echtzeit-Rekonstruktionspipeline die technischen und performativen Anforderungen moderner Virtual-Reality-Anwendungen erfüllen kann.

\section{Zielsetzung}
% Das Ziel dieser Arbeit ist daher die Entwicklung und Untersuchung einer modularen, photogrammetrie basierten Echtzeit-Rekonstruktionspipeline, die einfach in VR-Anwendungen integriert werden kann. Im Mittelpunkt steht dabei die Frage, inwieweit ein containerisiertes, modular aufgebautes Backend die Anforderungen an Performance und Zuverlässigkeit erfüllen kann, die in einer interaktiven VR-Umgebung bestehen.

% Zu diesem Zweck wird zunächst eine Systemarchitektur entworfen, die eine klare Trennung zwischen Steuerlogik, Kommunikationsschicht und den einzelnen Rekonstruktionsmodulen vorsieht. Durch den Einsatz von Docker-Containern sollen die Module voneinander isoliert, aber über standardisierte Schnittstellen austauschbar und erweiterbar bleiben. Anschließend wird ein Prototyp implementiert, der eine vollständige Kommunikationskette zwischen einem Unity-basierten VR-Frontend und dem modularen Backend realisiert.

Aus diesem Grund besteht das Ziel dieser Arbeit darin, eine modulare, photogrammetrie-basierte Echtzeit-Rekonstruktionspipeline zu entwickeln und zu untersuchen, die sich leicht in VR-Anwendungen integrieren lässt.  Die zentrale Fragestellung dreht sich um: Wie gut kann ein containerisiertes, modular aufgebautes Backend die Performance- und Zuverlässigkeitsanforderungen erfüllen, die eine interaktive VR-Umgebung stellt?

Um dies zu erreichen, wird zunächst eine Systemarchitektur erstellt, die Steuerlogik, Kommunikationsschicht und die einzelnen Rekonstruktionsmodule klar voneinander trennt.  Die Module sollen durch Docker-Container voneinander isoliert, jedoch über standardisierte Schnittstellen austauschbar und erweiterbar sein. Daraufhin wird ein Prototyp umgesetzt, der eine komplette Kommunikationskette zwischen einem Unity-basierten VR-Frontend und dem modularen Backend ermöglicht.

% Im weiteren Verlauf wird das System experimentell evaluiert, um Aufschluss über seine Leistungsfähigkeit im Echtzeitbetrieb zu gewinnen. Dabei werden Metriken wie Latenz, Durchsatz, Ressourcenauslastung und Stabilität untersucht, um die Effizienz und Belastbarkeit der Architektur zu bewerten. Ergänzend wird analysiert, wie aufwändig die Integration neuer Rekonstruktionsmodule ist und in welchem Maße das System die angestrebte Flexibilität tatsächlich ermöglicht.

% Die Arbeit konzentriert sich somit auf die architektonische und systemtechnische Perspektive der Echtzeit-Rekonstruktion. Ziel ist nicht die algorithmische Weiterentwicklung photogrammetrischer Verfahren selbst, sondern die Schaffung einer robusten und erweiterbaren Infrastruktur, die deren Einbindung in Virtual-Reality-Anwendungen vereinfacht. Durch die Konzeption, Umsetzung und Evaluation dieser Pipeline soll ein Beitrag zur Beurteilung geleistet werden, wie praktikabel modulare photogrammetriebasierte Rekonstruktionssysteme für den Einsatz in Echtzeit-VR-Szenarien tatsächlich sind.

Im weiteren Verlauf wird eine experimentelle Evaluierung des Systems vorgenommen, um seine Leistungsfähigkeit im Echtzeitbetrieb zu untersuchen. Zur Bewertung der Stabilität und Effizienz der Architektur werden Indikatoren wie Latenz, Durchsatz, Ressourcenauslastung und Stabilität untersucht. Darüber hinaus wird untersucht, wie aufwendig es ist, neue Rekonstruktionsmodule zu integrieren, und inwieweit das System die angestrebte Flexibilität tatsächlich bietet.

Die Arbeit legt ihren Fokus auf die systemtechnische und architektonische Sichtweise der Echtzeit-Rekonstruktion. Das Ziel ist nicht die algorithmische Weiterentwicklung photogrammetrischer Verfahren, sondern die Entwicklung einer robusten und erweiterbaren Infrastruktur, die deren Integration in Virtual-Reality-Anwendungen erleichtert. Diese Pipeline wurde konzipiert, realisiert und evaluiert, um zur Einschätzung der Praktikabilität modularer photogrammetriebasierter Rekonstruktionssysteme in Echtzeit-VR-Szenarien beizutragen.

\section{Aufbau der Arbeit}
Die vorliegende Arbeit ist in sieben Kapitel gegliedert, die systematisch vom theoretischen Hintergrund bis zur praktischen Umsetzung und Evaluation führen. Kapitel 2 legt die grundlegenden Konzepte dar, die für das Verständnis der Arbeit notwendig sind. Dazu gehören Virtual Reality, photogrammetrische 3D-Rekonstruktion, Container-Technologien und Docker, Kommunikationsprotokolle sowie die Entwicklungsumgebung Unity.

Kapitel 3 beschäftigt sich mit dem Stand der Technik. Hier werden bestehende Rekonstruktionspipelines und modulare Backend-Architekturen untersucht und kritisch bewertet. Zudem erfolgt eine Abgrenzung der Arbeit gegenüber existierenden Ansätzen, um die wissenschaftliche Relevanz der eigenen Entwicklung zu verdeutlichen.

In Kapitel 4 wird die Konzeption des Systems erläutert. Es beginnt mit der Anforderungsanalyse und geht anschließend auf die Systemarchitektur, das Schnittstellendesign und den Kommunikationsfluss zwischen Backend und VR-Frontend ein. Ein besonderer Schwerpunkt liegt auf dem Modularisierungskonzept, das die Integration verschiedener Rekonstruktionsmodelle erleichtert.

Kapitel 5 beschreibt die Implementierung der entwickelten Lösung. Sowohl die Backend- als auch die Frontend-Komponenten werden detailliert dargestellt, und die Integration in die bestehende Infrastruktur des Va.Si.Li-Labs wird erläutert.

Kapitel 6 widmet sich der Evaluation des Systems. Es werden die eingesetzte Methodik, die erhobenen Messdaten sowie die Analyse und Diskussion der Ergebnisse präsentiert. Die Evaluation untersucht sowohl die Performance und Zuverlässigkeit als auch die Erweiterbarkeit der modularen Architektur.

Abschließend fasst Kapitel 7 die zentralen Erkenntnisse der Arbeit zusammen, diskutiert Limitationen des entwickelten Systems und gibt einen Ausblick auf mögliche zukünftige Entwicklungen und Optimierungen.
