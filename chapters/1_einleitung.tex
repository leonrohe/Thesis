\chapter{Einleitung}

\section{Motivation}
Virtuelle Realität (VR) bietet durch ihre immersiven Eigenschaften ein enormes Potenzial für realitätsnahe und interaktive Nutzererfahrungen. Fortschritte in der Hand-, Augen- und Gesichtsverfolgung haben die Natürlichkeit virtueller Interaktionen deutlich verbessert und ermöglichen eine intensivere soziale Präsenz in digitalen Umgebungen. Trotz dieser Entwicklungen bleiben die meisten virtuellen Begegnungsräume künstlich und vollständig vom physischen Umfeld der Nutzer entkoppelt.

Auch hochrealistische digitale Avatare – etwa MetaHumans – können die Kluft zwischen physischer und virtueller Realität nur begrenzt überbrücken. Der direkte Zugang zu den realen Umgebungen von Kommunikationspartnern, beispielsweise in virtuellen Meetings oder kollaborativen Szenarien, ist nach wie vor technisch anspruchsvoll und wenig verbreitet. Eine realitätsnahe Integration physischer Räume in virtuelle Umgebungen könnte jedoch Authentizität, Raumgefühl und soziale Nähe deutlich verbessern.

Ein vielversprechender Ansatz zur Lösung dieses Problems liegt in der Nutzung von Verfahren zur photogrammetrischen 3D-Rekonstruktion in Echtzeit. Diese Technologien ermöglichen es, reale Umgebungen dynamisch zu erfassen und direkt in VR-Anwendungen zu übertragen. Dabei stellt sich die Frage, welche Verfahren sich hinsichtlich Qualität, Performanz und Integration besonders für diesen Anwendungsfall eignen.

\section{Zielsetzung}
Diese Arbeit verfolgt das Ziel, zwei aktuelle Verfahren zur echtzeitbasierten photogrammetrischen 3D-Rekonstruktion hinsichtlich ihrer Eignung für die Integration realer Umgebungen in Virtual-Reality-Anwendungen zu evaluieren. Im Fokus steht ein systematischer Vergleich auf Basis technischer, qualitativer, robustheitsbezogener und praktischer Kriterien. Dabei werden Aspekte wie Echtzeitfähigkeit, Rekonstruktionsgenauigkeit, Robustheit gegenüber schwierigen Umgebungsbedingungen sowie Integrationsaufwand in bestehende XR-Frameworks berücksichtigt.

Die Untersuchung erfolgt anhand eines prototypischen Anwendungsszenarios in der bereits existierenden Va.Si.Li-Lab-Umgebung des TextTechnologyLabs. Ziel ist es, eine fundierte Einschätzung darüber zu gewinnen, welches der beiden Verfahren hinsichtlich Leistungsfähigkeit, Flexibilität und Anwendbarkeit in immersiven Echtzeitkontexten besser geeignet ist. Die Ergebnisse sollen insbesondere im Hinblick auf den Einsatz in multimodalen Lern- und Kommunikationsszenarien nutzbar sein.

Die Arbeit positioniert sich damit an der Schnittstelle von Echtzeit-3D-Rekonstruktion, immersiver Systemintegration und technischer Evaluationsmethodik. Sie zielt darauf ab, aktuelle Photogrammetrieansätze vergleichend zu analysieren und praxisrelevante Empfehlungen für ihren Einsatz in aktuellen, sowie auch zukünftigen VR-Umgebungen abzuleiten.


\section{Aufbau der Arbeit}
Die vorliegende Arbeit ist wie folgt strukturiert:
Kapitel 2 gibt einen Überblick über den aktuellen Stand der Wissenschaft und Technik im Bereich immersiver VR-Anwendungen und photogrammetrischer 3D-Rekonstruktion. Kapitel 3 erläutert die zugrunde liegenden technologischen Grundlagen, insbesondere Echtzeit-Rekonstruktionsverfahren, XR-Plattformen sowie relevante Hardwarekomponenten.

Kapitel 4 beschreibt die Integration der beiden ausgewählten Verfahren in ein prototypisches Anwendungsszenario. Kapitel 5 stellt die methodische Vorgehensweise zur vergleichenden Evaluation vor. Kapitel 6 präsentiert die Ergebnisse der Untersuchung und diskutiert sie im Hinblick auf die zentralen Evaluationskriterien. Kapitel 7 beleuchtet die Limitationen der Arbeit, bevor Kapitel 8 einen Ausblick auf weiterführende Forschungsperspektiven gibt.
